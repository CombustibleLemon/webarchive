% find this class at https://archive.danleonard.us/scholarship/coursework/coursework.cls
\documentclass{../../../coursework}

\title{Challenges Faced in the Termination of Persons with Disabilities from a Segregated Workshop in Mombasa, Kenya}
\subtitle{}
\shorttitle{Termination at a Segregated Workshop in Mombasa, Kenya}
\author{Daniel}{Glenn}{Leonard}
\newdate{date}{14}{12}{2018} % all MSID papers submitted 2018-12-14T16:28:20+03:00
\date{\displaydate{date}}
\course{MSID}{4003}{Community Engagement in the Global South}{University of Minnesota Twin-Cities}
\instructor{Dr}{Mohamud}{Abdi}{}{Jama}
\instructor{Dr}{Mohamed}{}{}{Karama}

\keywords{disability, poverty, Kenya, rehabilitation, community, development}
\addbibresource{disability.bib}
%\addbibresource[location=remote]{https://archive.danleonard.us/scholarship/coursework/minnesota/MSID/4003/disability.bib}
\baseurl{https://archive.danleonard.us/scholarship/coursework/minnesota/MSID/4003/disability.xhtml}

\begin{document}

\maketitle

\begin{abstract}
    The Association for the Physically Disabled of Kenya (APDK) provides
    rehabilitation programs for persons with disabilities as well as
    segregated labor opportunities. In Mombasa, Kenya, ADPK operates
    Bombolulu Workshops and Cultural Centre as a segregated housing
    complex, containing work programs in producing fair-trade jewelry and
    mobility aids. Ability to permanently reside at the facility is tied to
    continued employment at the organization, which for laborers dismissed in
    cost-cutting retrenchments in 2011 and 2012 has presented difficulties in
    living and housing. The objective of this study was to assess the effect
    of dismissal on quality-of-life in persons with disabilities resident at
    Bombolulu. The study was conducted via interviews with administration, one
    focus group discussion amongst terminated employees, and observation of
    facility operations over the course of six weeks. The sheltered workshop
    model was found to produce marked gains in quality-of-life, yet these
    effects are nullified upon employees' termination. The future of the
    sheltered workshop model is discussed.
\end{abstract}

\printkeywords

\section{Introduction}

Persons with disabilities (PWDs) have unique challenges in the labor market
and in their daily lives.

\subsection{Persons with disabilities in the labor force}

The disability status of an individual is a major predictor of one's labor
force participation, for a wide range of disabling symptoms
\parencite{Stern1989}. This has been exacerbated in the West under
contemporary neoliberal policies, which frequently require paid work as a
prerequisite to receiving state welfare benefits, rather than welfare being a
redistributive public service \parencite{Lister1998,Wilton2006}. This shift
been undertaken ostensibly to liberate persons with disabilities from the yoke
of government dependency, yet it has failed to provide accessible labor
opportunities for these people to enter \parencite{Wilton2006}. As a result,
persons with disabilities have experienced increased necessity to compete for
employment.

In Kenya, persons with disabilities are widely excluded from the labor market.
Of 1,697 persons surveyed in the country, only 16.3\% had primarily spent
their last week in paid employment \parencite{2008}. When including work
within family businesses -- the informal sector possesses a considerable
proportion of Kenya's laborers \parencite{Livingstone1991} -- the total amount
of Kenyans with disabilities in consistent employment remains extremely low at
49.6\%. 7.2\% of Kenyans surveyed reported never having been employed, with
most of these respondents being in the 15-24-year-old cohort.

Compounding this phenomenon is a lack of access to education for Kenyans with
disabilities. In the same national study, over 70\% of respondents had not
attended schooling past the primary education level, and 4\% report having
experienced having been refused entry into a school due to disability. 

\subsection{Segregated Communities}

There is a long history of segregated employment opportunities for persons
with disabilities. In the United States since 1840, segregated educational and
labor opportunities have been available for the disabled in order to shelter
them from competition on the labor market, where it was expected that they
would surely be overlooked \parencite{2011}. Even today, United States federal
law provides an exemption to the minimum wage for persons with disabilities in
section~14~(c) of the Fair Labor Standards Act of 1938, who are permitted to
be paid lower rates than abled workers. The vast majority of certificates
authorizing employment of disabled workers at below minimum wage are awarded
to rehabilitation centers, where disabled people perform paid labor in a
segregated setting \parencite{1991}. The National Disability Rights Network
has on this point objected strongly to the continuation of segregated
employment in the United States, calling the system a condescending and
oppressive ``dead end'' for workers with disabilities \parencite[32]{2011}.
However, \textcite{Cobley2012} argues that in the developing world, building
broad-based support mechanisms is a more pressing goal than true integration.

While many sheltered workshops may encourage long-term or permanent
employment, it is inevitable that workers will depart from their positions, as
in any formal employment situation. Persons with disabilities, however, exist
in a more precarious state than abled workers and can be expected to be
affected by loss of employment differently than others. Specifically, the
termination of an employee can have a strong influence on a person's life.
This study set out to survey the effects of termination on employees of a
facility run by the Association for the Physically Disabled of Kenya (APDK).

\section{Methodology}

\subsection{Study Site}

APDK provides rehabilitative and assistance services to persons with physical
disabilities throughout Kenya. In the coastal city of Mombasa and the
surrounding areas, APDK's Coast Branch operates three large facilities.
Bombolulu Workshops and Cultural Centre employs PWDs in the production of
jewelry, textiles, wheelchairs, and toys, as well as in presentations of
indigenous Kenyan cultural activities for visitors. Likoni Furniture Workshop
is another sheltered workshop invested in the manufacturing of high-end
furniture. Port Reitz Rehabilitation Clinic, an independent part of the larger
state-owned Port Reitz District Hospital, specializes in long-term inpatient
care for children with correctable physical disabilities, including a small
classroom to ensure continuous education even when treatment occurs during the
school year. At Bombolulu and Likoni Workshops, most employees are provided
housing in residential facilities owned by APDK and cordoned off from the rest
of the town, paid for out of the garnished wages of residents. Both workshops
have also been the subject of former study by \textcite{Cobley2012}. APDK
Coast Branch also operates a microfinance program referred to as
Income-Generating Programme (IGP) or Faida (Swahili-language word for
``profit''), issuing small loans to PWDs for the operation and maintenance of
their own small businesses in the informal sector \parencites[for further
discussion of the Faida program, see][]{Wameyo2015}[also][]{Cobley2012}.

At Bombolulu Workshops, the subject of this study, commodity production is
primarily for export. Global fair-trade jewelry retailers are the
organization's largest clients, with produced items affixed the brand of the
client before packaging. Despite frequent high-volume sales, the organization
is most reliant on donations from non-governmental organizations to fund its
operations, previously reported by \textcite{Cobley2012}. In 2011 and 2012,
the organization terminated over twenty of their permanent staff to cut costs.
These former employees were instructed to vacate their APDK-owned residences,
yet still reside on-site due to a judicial injunction temporarily preventing
their eviction. 

\subsection{Data Collection}

The study was performed through four interviews with APDK Coast Branch
administrative staff, one focus group discussion, and observation of
employees' and administrators' activities over the course of six weeks working
alongside production staff. Topics discussed with administrators covered
compensation of workers, costs of state of APDK finances, and reasons for
employees' departure. The group discussion consisted of twenty-one former
employees who had been dismissed in the retrenchments of 2011 and 2012.
Following the taxonomy of social-science group discussions outlined by
\textcite{Frey1991} the discussion conducted at Bombolulu was structured as a
formal field interview and phenomenological in nature. Questions focused on
quality of life before and during residence at Bombolulu, compensation for
work, and life prospects post-dismissal. A currently-employed member of the
bureaucratic staff served as an interpreter for Swahili and a volunteer from
within the group translated to and from Kenyan Sign Language.

\section{Findings}

\subsection{Project Sustainability}

Administrators throughout APDK Coast Branch lamented the organization's
reliance on donations, expressing a desire for their projects'
self-sufficiency. At Bombolulu and Likoni, frequent reference was made to a
decline in funding, blamed primarily on a change in donors' opinions on the
efficacy and morality of sheltered workshops. Reference was frequently made to
the U.N.\ Declaration on the Rights of Persons with Disabilities, whose
ratification by the Kenyan government was paradoxically followed by a decline
in donations to APDK\@. Staff explained this discrepancy as the result of
foreign donors' belief in the passage of disability-focused law as inherently
implying its effective enforcement.

Today, many of the residential blocks at Bombolulu are commercially rented to
abled persons in order to generate revenue, a break from APDK's initial goals
of providing shelter for the needy. 

\subsection{Quality of Life in Kenya}

Workers described their lives before APDK as a spiral of poverty. They had
felt ignored by society, with a lack of medical attention or accessible
education. Only a handful graduated secondary school, although a few did
graduate from post-secondary education. Even for college graduates,
it was difficult or impossible to find stable employment. Those with degrees
were no more likely to report having had independent employment than those
without a primary school education, and those who did still described heavy
dependence on family for care and shelter. Jobs attained were almost
exclusively in tailoring or as house help, with little variation. Compounding
these effects were the impacts of single motherhood, a frequently cited
problem for the women who made up a majority of the group discussion's
participants, detailing that their male partners had fled upon learning of
their pregnancy.

\subsection{Employment}

Usually by word-of-mouth, discussion participants had in young adulthood
learned of APDK's sheltered workshop programs and traveled to Mombasa to seek
employment. The promise of independence was the greatest draw for them, as
dependence on family for basic necessities well into adulthood felt
humiliating. At APDK, workers felt that they could be the arbiter of their own
destinies. As one worker described ``life became easy for the first time.''
For those with children to care for, a home of one's own and a reliable
paycheck was invaluable. Daycare services are available at Bombolulu even for
non-affiliated families outside APDK, and during the school break -- when this
study took place -- it was common to observe children playing throughout the
day on the large sports field and playground.

Not all aspects of tenure at Bombolulu were so described as a blessing. The
former workers described low salaries and a high cost of living while at the
facility. Most found it possible to sustain themselves at the most basic
level, yet most agreed with one worker who stated, ``I couldn't save even a
shilling.'' Administrators reported the average salary to be 17,600~KES
(approx.~\$176~USD) per month, after garnishment for residential utility
costs. However, workers detailed that payment was based on individual
production, not on labor-hours worked. Some reported having worked
dawn-to-dusk to meet their production quotas, with no overtime. Concurrently,
there was a shared resentment amongst workers regarding the administration --
primarily seen as able-bodied -- raising their own salaries while the aging
workforce struggled to make ends meet. Despite having full-time employment,
the workers reported having to survive on loans from an informal credit union
formed amongst themselves. Several workers described having gone stretches of
up to two days without food, waiting for the next paycheck.

\subsection{Effect of Termination}

When Bombolulu administration declared the cost-cutting retrenchments in 2011
and 2012 of those who comprised the focus group discussion, advance notice was
given one month beforehand. The workers, however, thought one month's warning
was inadequate. Having expected to work for APDK until retirement age,
dismissal left the laborers in a position of uncertainty whether they would be
able to find employment outside. Compounding this was the mandate to vacate
their employer-supplied housing, a labor relation uncommon in factory work.
Despite receiving a court injunction temporarily permitting continued
residence, many live in fear of the day they are permanently evicted. No
members of the discussion said they knew of a place to move to should they
have to leave Bombolulu.

APDK's mission statement is ``to enable persons with different abilities to
overcome their physical limitations by empowering them economically and
socially so that they become fully-integrated members of their communities,''
inscribed on the external wall of a housing block clearly visible upon entry
into Bombolulu. During the focus group discussion, a collective murmur discord
echoed in response to recital of this statement. Regarding economic
empowerment, the workers felt that their socioeconomic position had not
improved during the course of their tenures, going so far as to declare that
their lives are more difficult now than before entering the employ of APDK.
Many had worked since young adulthood, and now lacked the vitality they had
possessed at the time of their application to work at Bombolulu. Families who
had formerly provided support were cited as no longer being able to, having
aged in the decades their children had worked for APDK. Furthermore, many had
chosen to raise families at the workshop, and without income or savings it has
become difficult to provide for them. As one worker put it, ``I've gone from
the support of a family to supporting a family.''

These resentments have led to concern in the eyes of the terminated workers
over whether APDK administration truly cares for PWDs' quality of life. As one
worker stated and several concurred, ``this place was founded by persons with
disabilities, it belongs to us.''

\section{Discussion}

Despite grandiose goals, APDK's use of the sheltered workshop model has
presented issues both in its sustainability and in its ability to care for
PWDs. In concurrence with what \textcite{Cobley2012} and \textcite{2011} have
argued, the segregated system fails PWDs in many ways and serves to maintain a
two-tier employment system. Workers feel cut-off and are unable to reach the
promised self-sufficiency. It is advised that disability-rights organizations
continue to make inclusive approaches, as discussed by \textcite{Cobley2012}.

\section{Limitations}

Due to restrictions on independent undergraduate research, it was not possible
to conduct one-on-one interviews with non-administrative employees of APDK. No
focus group discussion was conducted amongst those currently in the employ of
the workshops, as was planned, nor amongst those who had retired and vacated
their residences of their own volition, most of whom no longer lived near
Mombasa. The single focus group discussion in this study took three weeks to
organize, and as it was held in the final week of the study no follow-up
inquiries could be investigated.

\printbibliography

\end{document}
