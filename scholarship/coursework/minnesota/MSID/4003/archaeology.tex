% find this class at https://archive.danleonard.us/scholarship/coursework/coursework.cls
\documentclass[american]{../../../coursework}

\title{Current Disputes in the Theory of African Archaeology}
\subtitle{}
\shorttitle{}
\author{Daniel}{Glenn}{Leonard}
\newdate{date}{14}{12}{2018} % all MSID papers submitted 2018-12-14T16:28:20+03:00
\date{\displaydate{date}}
\course{MSID}{4003}{Community Engagement in the Global South}{University of Minnesota Twin-Cities}
\instructor{Mr}{Mwanzia}{David}{}{Kyule}

\keywords{out-of-Africa, multiregional, colonialism, \emph{Homo}, art}
\addbibresource[location=remote]{https://archive.danleonard.us/scholarship/coursework/minnesota/MSID/4003/archaeology.bib}
\baseurl{https://archive.danleonard.us/scholarship/coursework/minnesota/MSID/4003/archaeology.xhtml}

\begin{document}

\maketitle

\begin{abstract}
    Africa has the oldest evidence of human habitation on the planet. As such,
    most of its history is open to interpretation by paleoanthropologists.
    Today, intense disputes occur over the path of human evolution as well as
    the theory of the field itself. These disputes are discussed.
\end{abstract}

\printkeywords

\section{Introduction}

Housing the geographic origin of humans and their primate relatives, Africa
has the oldest archaeological record of anywhere on Earth
\parencite{Holl2010}. Archaeological discoveries taken as markers of human
society, such as tool manufacturing, symbolism, and agriculture, find their
earliest occurrences in Africa. The Rift Valley of East Africa alone is home
to nearly all stone tools claimed to be the oldest \parencite{Torre2011}. The
draw of archaeologists and their students to Africa can be attributed to this
uniquely ancient, uninterrupted history of human evolution and society. As
such, the practice and theoretical underpinnings of African archaeology have
existed in a state of intense debate and change, evolving at a faster pace
than its subjects. Little is settled fact in this discipline, from the
earliest history to the present day.

\section{Western and Colonial Influence on the Development of African Archaeology}

African archaeology is inexorably marked by its relationship to the Western
colonial project and global capitalism. As a scientific discipline,
archaeology arose from the era of Enlightenment in Western Europe, and
developed from the era's understanding of history. As such, the underlying
paradigms often conflict with the needs and historical conception of African
societies \parencite{Schmidt1995,Shepherd2002}.

\subsection{The Colonial Era}

Colonialism in Africa served to provide labor and resources to Western
capital, and anthropologists on the continent were primarily Western
Enlightenment-era thinkers, employed by the colonial administration or
European universities. Archaeology served as a tool of the colonial
government, whose findings coincidentally served as justification for the
right of Europeans to rule over the continent. European archaeology denied the
extant African population the right to claim as their heritage any complex
social structures discovered in the archaeological record, and used this
denial to explain the ``backwardness'' of Africans and thus the necessity for
European conquest \parencite{Koff1997}.

An exemplary example of this is that of Great Zimbabwe, a series of stone
ruins in southern Africa. The South Africa Company in its initial commissions
for study of the site specified the necessity to find evidence for their
European origin, and denied from the outset any claims Black Africans' claims
to the heritage. Early archaeologists and colonial architects thus attributed
the site to ancient peoples of the Levant rather than the resident African
population, and would defend this conclusion virulently. The company's results
were used to justify their own conquest of southern Africa as a return to
their homeland. Professional archaeologists, many of whom took opposing
viewpoints on the origin of Great Zimbabwe and on other southern African
historical dilemmas, found themselves censored by the colonial administration
\parencite{Koff1997,Shepherd2002}.

\subsection{Archaeology in Independence Movements}

During and after independence, the practice of African archaeology reflected a
new social and political system. Some African nationalist movements sought to
reclaim the prehistory of the continent, taking their ancestral history back
from the colonial empires. Today's majority-ruled Republic of Zimbabwe, for
instance, gets its name from the archaeological site which had been the
subject of colonial falsification. Likewise, Nigeria, Senegal, Ghana, and
Zambia all introduced archaeology to the students of their new national
universities, allowing Africans to voice their own histories
\parencite{Koff1997}.

Other nationalist movements were slow to adopt archaeology as a discipline. In
Kenya, archaeology outside that of early human evolution has failed to see the
light of day, with most archaeologists being Europeans following in the
footsteps of the Leakeys \parencite{Koff1997}. \textcite{Schmidt1995} has
argued that this white-dominated focus on paleoanthropology is an ideological
choice by the government: the Kenyan ruling elite is threatened by the land
claims of the country's ethnic groups, many of whom had their traditional land
holdings replaced with prejudice against each other by the British ``divide \&
conquer'' strategy.

\begin{quote}
    Archaeology, if allowed to flourish at a regional level, can easily be
    identified with an attempt to valorize the history … of one ethnic group
    at the perceived expense of others. The state's deep investment in the
    white intellectuals' pursuit of ancient human ancestors has been an ideal
    way to neutralize regional histories … it focuses on a `population' devoid
    of ethnicity -- indeed, devoid of humanness.

    -- \textcite[128-129]{Schmidt1995}
\end{quote}

Just as archaeology served the ideological goals of the colonial empire,
post-independence scholarship continues to reflect the dilemmas within African
political spheres.

\section{Rise of \emph{Homo sapiens}}

Despite the African continent's undisputed status as the geographic origin of
early primates and hominins, debate still arises over the particulars of how
members of the genus \emph{Homo} interacted with one another and spread
throughout the world. Paleoanthropologists are certain that \emph{H. habilis}
lived exclusively in Africa, followed by populations of H. erectus,
\emph{H. heidelbergensis}, and presently \emph{H. sapiens}, all of whom appear
to have had significant presence on the Eurasian continent
\parencite{Wenke2007}. How these populations interacted and evolved is
uncertain.

\subsection{Early Hominin Expansion}

\emph{Homo erectus} was most likely the first hominin to venture off the
African continent, with a range of sites from southeast Asia to the Iberian
Peninsula. Compared to \emph{H. sapiens}, \emph{H. erectus} had more
protruding jaws, smaller cranial capacity, and larger brow ridges
\parencite[see][fig.\ 4.2]{Wenke2007}. \emph{H. erectus} likely lived in caves
and used fire, suggesting adaptability to the cold climates outside Africa.
Evidence strongly suggests \emph{H. erectus} manufactured tools, with
\emph{H. erectus} sites containing tools dating to .6~Ma in Europe and over
1~Ma in Africa. Various improvements in toolmaking may have allowed
\emph{H. erectus} to adapt to the wide variety of climates and ecosystems
encountered throughout the species' historical range. Specifically, the
Acheulian hand-axe shows a creativity in its craftsmanship, requiring
forethought and an implicit understanding of stone flaking behavior. Its
widespread range suggests a versatility in use unmatched in preceding hominin
technology \parencite{Wenke2007}.

\emph{H. neanderthalensis} (sometimes \emph{H. sapiens neanderthalensis})
emerged around .5~Ma in Europe and Central Asia, with a physical structure
vastly similar to that of anatomically-modern humans.
\emph{H. neanderthalensis} showed some adaptations compared to
\emph{H. erectus} to the cold weather of their native Europe, with more
massive and muscular bodies. The species also appears to have practiced
burials and manufactured specialized stone tools, a marked similarity to
\emph{H. sapiens} in comparison to \emph{H. erectus}.

The lack of a direct line of hominin remains showing flows of single
populations, as well as the phenotypical similarities and gradients within the
remains of early hominins, make it difficult to pinpoint the origin of
anatomically modern humans. Theories have thus arisen to explain the genetic
and phenotypical differences current \emph{H. sapiens} shows across geographic
regions.

\subsection{Out-of-Africa~\RN{2}}

The replacement model, better known as the ``out-of-Africa~\RN{2}''
hypothesis, proposes that \emph{H. sapiens} originated exclusively in Africa
and spread to Eurasia recently. This wave of emigration drove the
already-present Eurasian \emph{Homo} populations, who had migrated nearly a
million years beforehand in ``out-of-Africa~\RN{1},'' to extinction,
``replacing'' them. In the 1980s, paleoanthropologists accepted the idea of
\emph{H. sapiens} appearing around .2~Ma and rapidly spreading outside Africa,
completely replacing populations of hominins without any population mixture
\parencite{Kendrick2014}.

The oldest \emph{H. sapiens} fossils have all been found in Africa, with
Richard Leakey's Omo River Valley fossils dated to 195~ka. Further,
transitional fossils have only been found on the African continent, with
absences of any transitional remains in Eurasia or Australasia.
Morphologically, African fossils show the greatest internal diversity as well
as the greatest difference from all non-African populations. Even without
genetic testing, replacement theory was accepted as the story of human
evolution by the 1980s \parencite{Kendrick2014}.

Recent genetic evidence provides further support for this second migration of
humans. Because mitochondria have their own genetic sequence and are passed
down exclusively maternally, geneticists have been able to hunt for a maternal
ancestor of all living humans. \textcite{Cann1987} studied the mitochondrial
DNA sequences of 147 people and, extrapolating from the known rate of mutation
of DNA, concluded that all living populations today derive from one ancestor
who lived approximately .2~Ma, likely in Africa.

\subsection{Multiregional Hypothesis}

The multiregional or regional-continuity hypothesis proposes that modern
\emph{H. sapiens} are the result of contact and interbreeding between
disparate human populations across Eurasia. What is now one human species is
seen as the result of convergent evolution from different members of the genus
\emph{Homo} who interacted and -- at least partially -- united globally.

Multiregionalists point to the presence of unique genes in European
\emph{H. sapiens} populations, attributed to interbreeding with
\emph{H. neanderthalensis}. Thus, it is likely that what came to be
\emph{H. sapiens} is the result of mergers between formerly distinct hominin
populations. Conversely, they criticize the use of mitochondrial DNA by
replacement theorists, citing it as based on unfounded assumptions
\parencite{Kendrick2014}.

Morphologically, the extant peoples of Eurasia and Africa are quite distinct.
Followers of multiregional theory use this discrepancy as evidence for
distinct evolutionary paths, which gained distinct genetic traits from
separate species of \emph{Homo} \parencite{Kendrick2014}.

\section{Prehistoric African Art}

Some of the earliest evidence for anatomically modern humans is the art they
left behind. Cave paintings, pottery, and beadwork are all major classes of
artifacts in African archaeology. Especially with rock art, the existence of
these have long enthralled European archaeologists; however, the European
conception of art brings with it cultural baggage \parencite{Arnaut1991}.
Postmodern criticism of these research topics have questioned the veracity of
whether we can ``know'' anything about what the creators of such creations
were thinking at the time. It is tempting to consider these from the
standpoint of the Western conception of art, yet it is impossible to know what
a hominin to which we have no social connection could have ever thought during
its lifetime.

\printbibliography

\end{document}
