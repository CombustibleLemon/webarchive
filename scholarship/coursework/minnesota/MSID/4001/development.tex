% find this class at https://archive.danleonard.us/scholarship/coursework/coursework.cls
\documentclass[american]{../../../coursework}

\title{The Illusion of Progress:}
\subtitle{How International Development Discourse Comforts the Comfortable}
\shorttitle{The Illusion of Progress}
\author{Daniel}{Glenn}{Leonard}
\newdate{date}{14}{12}{2018} % all MSID papers submitted 2018-12-14T16:28:20+03:00
\date{\displaydate{date}}
\course{MSID}{4001}{International Development: Critical Perspectives on Theory and Practice}{University of Minnesota Twin-Cities}
\instructor{Dr}{Mohamud}{Abdi}{}{Jama}

\keywords{development, core, periphery, modernization, dependency, neoliberalism}
\addbibresource[location=remote]{https://archive.danleonard.us/scholarship/coursework/minnesota/MSID/4001/development.bib}
\baseurl{https://archive.danleonard.us/scholarship/coursework/minnesota/MSID/4001/development.xhtml}

\begin{document}

\maketitle

\begin{abstract}
    International development has been a growing field of study over the past
    half-century. Through multiple incompatible models, the discourse
    surrounding the relationship between the global rich and poor has evolved
    rapidly. However, despite the neo-Marxist schools of thought predominant
    in academic development settings, the practice of development is presently
    neoliberal in nature. This discrepancy is discussed, along with its
    effects on global poverty and core-periphery relations.
\end{abstract}

\printkeywords

\section{Introduction}

Since the decolonization of Western empires in the second half of the
twentieth century, the field of international development has been host to
intense debate over the efficacy of its various theoretical underpinnings as
well as what ``development'' means as a concept.

Colonialists' attitudes toward their subject populations were overwhelmingly
negative. Westerners looked upon captive populations as if their traditional
ways of life were ``primitive'' or ``backwards,'' and frequently justified
their colonial endeavors as exercises to use the land and labor of the
indigenous peoples for ``productive'' purposes, often involving a progressive
notion of education of the unwashed masses. The real impetus for colonization
in the era of classical liberalism was not humanitarian, but rather the
expansion of finance-capital in the West and the necessity of its export to
new markets \parencite{Lenin1917}. The conception of development at its
inception was thus as a forced transition of world cultures toward
subservience to Western capital. It is evident that in practice, if not in
theory, the whole of development discourse remains an ideological project in
the interests of a minority class whose interests run counter to its own
stated goals.

\section{Early Postcolonial Development: The Modernization School}

The world order after the Second World War continued Western hegemony over the
vast majority of the world, albeit in a new form. The age of vast empires
ruling over captive populations, subject to the whims of European monarchs,
had come to an end; however, colonialism had brought the imperial bourgeoisie
a standard of living requiring constant consumption from the colonial
subjects. Empire could not relinquish its monopolies on resource extraction
and free labor, and maintained these controls despite the independence of
former subjects. Development is now run through global capitalism, free trade,
and international treaties, but lurking under the surface is the military
might of the former empires which the colonial subjects know all too well.

The idea of certain civilizations living ``backwards'' lives in need of
improvement towards the Western ideal carried on through decolonization, and
in the middle of the twentieth century became the centerpiece of what would
come to be called modernization theory. Thinkers in this school placed the
impetus of development -- defined as the attaining of a Western industrial
state -- upon the peoples of the Third World. As \textcite{Tipps1973}
identifies, the language of the ``savage'' and the ``civilized man''
disappear, yet the ideological foundation of nineteenth-century ethnocentrism
remains. Modernization theorists ``continue to evaluate the progress of
nations… by their proximity to the institutions of Western, and particularly
Anglo-American societies'' \parencite[206]{Tipps1973}. The new dichotomy is
``modern'' versus ``tradition,'' where formerly colonized societies must take
it upon themselves to eschew established customs and pull themselves up to the
rest of the world.

The modernization school makes little attempt to mask its ideological bent,
likely due to its significant financial support from the United States
government. Modernization theory was during the Cold War used to direct Third
World countries away from communism and towards open relations with the West,
and thus found itself at home in the whitepapers of the U.S.\ State Department
\parencite{So1990}. The Western governments found it effective to have a
coherent framework with which to justify newly-independent nations' staying
within their orbits.

The development project, as defined by \textcite{McMichael2004}, was the
nation-state-centric model of development pursued through the Bretton Woods
system from the 1940s to the 1970s. Decolonization was in full swing, and
newly-formed countries were eager to improve their lives outside the bounds of
colonial government. The endpoint for a nation in this era was considered to
be the achievement of Western-style luxury and affluence. Newly-formed states
designed themselves around development, making large public expenditures to
build infrastructure and state-building projects. The states that took this
route adopted Western capitalism, and enormous loans were distributed to
countries across the global South, many in food and most in hard currency.
Industries spawned from Western investment and began exporting cheap products
en masse to the West.

While the changes observed during the development project appear as success,
they mask the parasitism of Western influence. For the gains that these
nations made internally, former imperial nations reaped even greater rewards.
Monocropping, introduced by colonial powers to produce one major product for
the imperial homeland, was insufficient to feed the new nations' growing urban
industrial proletariat. Thus, even lush nations were importing grains on a
large scale from the United States. These imports came at a price -- large
loans from Bretton Woods institutions and Western states would need to be paid
back, but many countries found themselves in debt traps, getting new loans to
pay off the interest on previous ones \parencite{McMichael2004}. Despite
radical growth, the industrial sector of these new economies was dominated by
the finance-capital of Western nations, where foreign investors enjoyed
majority stakes in domestic firms.

Within the bounds of global capitalism, it was impossible for
newly-independent states to escape these traps. The development project, and
the modernization theory that was its underpinning, failed to address the
impact of colonialism on former colonies. From the very beginning, the project
was used to ensure continuous profit rather than true global equity
\parencite{Frank1966}. Despite being ``independent,'' in many cases the former
empire was permitted to write the constitution of the new nation. The system
of land ownership -- and the owners of land -- thus stayed intact in nearly
every postcolonial country besides Cuba, permitting European extraction of
resources from native lands even in the age of independence. Eager to escape
dependence on single-resource exports, countries sought to boost industrial
production \parencite{McMichael2004}, and in doing so gave a major boon to
First World business. Newly-industrializing countries allowed foreign
investment in their manufacturing, and soon found the bourgeoisie of the First
World directly controlling the Third World proletariat. Labor-intensive
products could be made inexpensively using the underpaid workers previously
exploited by colonists, while capital-intensive products stayed at home.

\section{A Neo-Marxist Critique: The Dependency School}

The flaws of the modernization theorists did not go unnoticed. Many Marxists
in Latin America criticized these normative and Western-centric assumptions
implicit in the modernization model and developed a competing framework,
christened dependency theory \parencite{So1990}. Dependency theorists argued
that the world could be divided into a capitalist ``core'' which by the
necessity of the capitalist system extracts resources from the global
``periphery,'' restricting the peripheral regions' agency
\parencite{Frank1966}. The theory further improved on the Marxist
understanding of imperialism. For the orthodox Leninist, imperialism would
collapse as individual great powers consolidate control over shrinking
resources \parencite{Lenin1917}; dependency theorists argue that the core
should be seen as one unitary entity which imposes a dependency relationship
\parencite{Ferraro2008}. Most importantly, dependency theory recognizes that
the historical trajectory of Western countries cannot act as a model for the
development of the periphery, as the core has never experienced colonialism
\parencite{So1990} -- Rostow's ``stages of growth'' are completely irrelevant
to the periphery.

Despite the theoretical advancements of the dependency theorists, the practice
of development continues to remain within the interests of the global core.
Few countries have totally eschewed the Western capitalist model, and those
who operate outside the bounds of the global financial system often do so as a
result of sanctions rather than self-imposed isolation. The examples of those
that have, however, provide insight into why dependency theory has not
affected the policies of mainstream international development.

The failure of dependency theory to materialize the radical change it calls
for can best be attributed to an inability to fully antagonize capital.
Marxist and anti-imperial movements were able to gain ground in the final
quarter of the twentieth century, but were frequently crushed by Western
interests. For instance, see Burkina Faso after 1983 (coup 1987), Chile after
1970 (U.S.-backed coup 1973), or Libya after 1969 (U.S.-backed coup 2011).
The lack of class analysis within dependency theory outside of the
core-periphery dichotomy leaves dependency theory's policy proposals poorly
staged to prevent the core from imposing its will upon the periphery. Members
of the New Left placed too much emphasis on the transcendentalist nature of a
desired post-revolutionary culture, failing to truly antagonize or even
significantly impact the dominant capitalist class
\parencite[see][]{Schiff1973}. Thus, despite the neo-Marxist dependency
theorists' efforts, the neoliberal world order developed approaching the turn
of the century.

\section{A Continued Hegemony: Globalization}

Following the end of the Bretton Woods system, the global financial order
entered what \textcite{McMichael2004} terms the era of globalization.
Individual Third World states were led to collapse by the development project
and its neoclassical economic system, and out of their ashes the global core
shaped new state systems ripe for further extraction. Rather than an
increasing GNP, the globalization project set as the premier marker of
development a country's successful participation in the capitalist world
market. Thus, open free trade in all goods and services was revered as the
supreme policy goal for Third World nations.

The global core had previously found the postcolonial nation-state to be an
effective source of resource extraction and cheap labor; however, the idealist
self-reliance and equitable provision of public services would come to be seen
as a hindrance by Western bankers and financiers. The debt crisis of the 1980s
allowed the Western powers to impose structural adjustment programs on the
global South, requiring privatization and openness to capital as requirements
to escape the debt trap. Periphery countries were required to relinquish their
sovereignty to the Western financial institutions, and the little agency Third
World nations had enjoyed would soon vanish \parencite{McMichael2004}.

Internal policies of nations are no longer determined by internal factors, but
by the whims of the global managerial class, who exist outside the bounds of
any political system and are only beholden to their financial backers. The
World Trade Organization sets binding rules on nation-states, requiring that
periphery nations expose themselves to Western monied interests or face
sanctions. The global financiers require countries to adopt structural
adjustment programs in order to receive money, or even to refinance crippling
debt. These programs often include provisions for the privatization of public
services and reduction in social benefit programs. In many countries,
privatization of state resources and public services allowed Western financial
services corporations to reap profits by exploiting the impoverishment of
Third World residents \parencite{McMichael2004}.

The development project had allowed Western nations to maintain the colonial
division of labor in extracting resources and labor-power from the global
South. The globalization project furthered this premise, but in addition
allowed the extraction of profit directly from the states the development
project had built as if governments did not exist at all. Thus, the globalized
world acts as one unified, continuous economy. This is strikingly evident in
the size of the global unemployed, who number far greater than in previous
eras. \textcite{Frobel1976} presciently describe this phenomenon in terms of
the reserve army of labor \parencite[see][ch.\ 25.2]{Marx1887}, applied to a
global scale in the shadow of the colonial division of labor. The development
structure of the global south is designed -- by Western interests -- to
provide a seemingly-inexhaustible supply of labor, the individual persons of
which are overworked and spent quickly.

The non-governmental organization (NGO) serves important facets of
contemporary globalization. Since the mid-nineteenth century, this structure
has played important roles in international governance. In 1876, King
Leopold~II of Belgium sought funding for a scientific NGO, L'Association
Internationale Africaine, with claimed humanitarian aims in developing the
Congo river basin. In reality, the Association's sole aim was the personal
enrichment of the king, and quickly formed into a state controlling large
swaths of Central Africa. Promising free trade with the world powers,
Leopold~II's puppet organization was granted absolute control over the Congo
basin in the Berlin Conference of 1884-85, and proceeded to extract resources
through one of the most brutal colonial regimes imposed upon Africa
\parencites{Sayre1919}[see also][]{Charnovitz1997}. While no other such
organization has since received the blessing of de facto statehood, it is
notable as an early example of Western world powers awarding great control
over development to an unaccountable organization in pursuit of a liberal
ideal (in the case of the Congo Free State, free trade of natural rubber to
all nations). Today, small NGOs are growing at a rapid pace and serve a
different purpose -- that of cultural change. \textcite{Charnovitz1997} notes
that NGOs are often based in core countries, and amplify the views of the
core, regardless of how prevalent those views are in the periphery.
NGOs also serve to break down the nation-state system, creating cross-border
associations free to roam unbound by the rigid hierarchy of states.
While organizations may be non-profit, they frequently count as members or
directors large for-profit corporations. Thus, even the most ``humanitarian''
of NGOs implicitly serve the goals of Western capital, either knowingly or
unknowingly.

\section{Discussion}

Development as an academic discipline is inherently flawed, if not in theory
then certainly in practice. Since the nineteenth century, organizations have
been formed by Western colonial interests in order to extract riches from the
land and labor of the global South. While academics have provided string
critiques of this system, the practice of international relations remains a
one-way street for extraction and oppression. The global periphery has not
been permitted to achieve true independence from the core, and remains stuck
within the colonial legacy. Save for the Republic of Cuba and the Democratic
People's Republic of Korea, there are few extant examples of periphery social
structure outside the model implanted by the core.

\printbibliography

\end{document}
