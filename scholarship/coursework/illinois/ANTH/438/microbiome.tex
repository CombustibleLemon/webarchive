% find this class at https://archive.danleonard.us/scholarship/coursework/coursework.cls
\documentclass[american]{../../../coursework}

\title{Non-Bacterial Components of the Primate Microbiome:}
\subtitle{Against a Bacterial Bias}
\shorttitle{Non-Bacterial Components of the Primate Microbiome}
\author{Daniel}{Glenn}{Leonard}
\newdate{date}{18}{10}{2019}
\date{\displaydate{date}}
\course{ANTH}{438}{Primate Life History Evolution}{University of Illinois at Urbana-Champaign}
\instructor{Dr}{Kathryn}{B.}{H.}{Clancy}

\keywords{archaea, bacteria, eukarya, viruses, taxonomy}
\addbibresource[location=remote]{https://archive.danleonard.us/scholarship/coursework/illinois/ANTH/438/microbiome.bib}
\baseurl{https://archive.danleonard.us/scholarship/coursework/illinois/ANTH/438/microbiome.xhtml}

% Packages and commands for R code in Appendix A
\usepackage{color}
\usepackage{framed}
\usepackage{fancyvrb}
\newcommand{\VerbBar}{|}
\newcommand{\VERB}{\Verb[commandchars=\\\{\}]}
\DefineVerbatimEnvironment{Highlighting}{Verbatim}{commandchars=\\\{\}}
\definecolor{shadecolor}{RGB}{248,248,248}
\newenvironment{Shaded}{\begin{snugshade}}{\end{snugshade}}
\newcommand{\AlertTok}[1]{\textcolor[rgb]{0.94,0.16,0.16}{#1}}
\newcommand{\AnnotationTok}[1]{\textcolor[rgb]{0.56,0.35,0.01}{\textbf{\textit{#1}}}}
\newcommand{\AttributeTok}[1]{\textcolor[rgb]{0.77,0.63,0.00}{#1}}
\newcommand{\BaseNTok}[1]{\textcolor[rgb]{0.00,0.00,0.81}{#1}}
\newcommand{\BuiltInTok}[1]{#1}
\newcommand{\CharTok}[1]{\textcolor[rgb]{0.31,0.60,0.02}{#1}}
\newcommand{\CommentTok}[1]{\textcolor[rgb]{0.56,0.35,0.01}{\textit{#1}}}
\newcommand{\CommentVarTok}[1]{\textcolor[rgb]{0.56,0.35,0.01}{\textbf{\textit{#1}}}}
\newcommand{\ConstantTok}[1]{\textcolor[rgb]{0.00,0.00,0.00}{#1}}
\newcommand{\ControlFlowTok}[1]{\textcolor[rgb]{0.13,0.29,0.53}{\textbf{#1}}}
\newcommand{\DataTypeTok}[1]{\textcolor[rgb]{0.13,0.29,0.53}{#1}}
\newcommand{\DecValTok}[1]{\textcolor[rgb]{0.00,0.00,0.81}{#1}}
\newcommand{\DocumentationTok}[1]{\textcolor[rgb]{0.56,0.35,0.01}{\textbf{\textit{#1}}}}
\newcommand{\ErrorTok}[1]{\textcolor[rgb]{0.64,0.00,0.00}{\textbf{#1}}}
\newcommand{\ExtensionTok}[1]{#1}
\newcommand{\FloatTok}[1]{\textcolor[rgb]{0.00,0.00,0.81}{#1}}
\newcommand{\FunctionTok}[1]{\textcolor[rgb]{0.00,0.00,0.00}{#1}}
\newcommand{\ImportTok}[1]{#1}
\newcommand{\InformationTok}[1]{\textcolor[rgb]{0.56,0.35,0.01}{\textbf{\textit{#1}}}}
\newcommand{\KeywordTok}[1]{\textcolor[rgb]{0.13,0.29,0.53}{\textbf{#1}}}
\newcommand{\NormalTok}[1]{#1}
\newcommand{\OperatorTok}[1]{\textcolor[rgb]{0.81,0.36,0.00}{\textbf{#1}}}
\newcommand{\OtherTok}[1]{\textcolor[rgb]{0.56,0.35,0.01}{#1}}
\newcommand{\PreprocessorTok}[1]{\textcolor[rgb]{0.56,0.35,0.01}{\textit{#1}}}
\newcommand{\RegionMarkerTok}[1]{#1}
\newcommand{\SpecialCharTok}[1]{\textcolor[rgb]{0.00,0.00,0.00}{#1}}
\newcommand{\SpecialStringTok}[1]{\textcolor[rgb]{0.31,0.60,0.02}{#1}}
\newcommand{\StringTok}[1]{\textcolor[rgb]{0.31,0.60,0.02}{#1}}
\newcommand{\VariableTok}[1]{\textcolor[rgb]{0.00,0.00,0.00}{#1}}
\newcommand{\VerbatimStringTok}[1]{\textcolor[rgb]{0.31,0.60,0.02}{#1}}
\newcommand{\WarningTok}[1]{\textcolor[rgb]{0.56,0.35,0.01}{\textbf{\textit{#1}}}}

\begin{document}
\setcounter{page}{52}
\maketitle

\section{Introduction}

Since the discovery that the world in which we live is entirely filled with
organisms too small to see with the eye, biologists have drawn a line between
the microorganism and the multicellular organism. However, the search for the
evolutionary apex of cellular organisms has blurred this line, from the
theory of lateral gene flow \parencite{Woe98} to the discovery that archaea
and eukarya share more in common with each other than either shares with
bacteria \parencite{Woe77}. In researching the microbiome -- the physical
commensal relationship between animals and their microbial inhabitants -- it
is important to consider the full diversity of microbial taxa. This paper
seeks to investigate biases in microbiome research toward specific microbial
taxa and seek out how the less-studied varieties of microbial inhabitants may
have effects unaccounted for.

Despite a growth of research interest in archaea since its identification by
\textcite{Woe77}, little research has been conducted into their relationships
to other organisms, such as humans. As the domain was originally understood
as a clade of extremophiles due to their original isolation from geothermal
vents \parencite{DeL01}, it would be expected that archaea would be a minor
part of more familiar environments; however, it is now known that members of
archaea exist with great frequency in non-extreme environments, such as
plankton, lakes, and soils \parencite{DeL01}. Regardless, research into the
composition of human microbiota retains a strong bias toward bacteria
\parencite{Llo16, The12}. Comparative primatology has often failed to expand
microbiota research beyond bacteria, frequently excluding unknown or
infrequent phyla from results of DNA extraction
\parencite{Gar19,Mal18,Ork191,Ork19}. At its extreme, the term ``microbe''
has been used synonymously with the term ``bacteria'' despite the far more
precise definition of the latter \parencite[for an example, see][]{Mal18}.

\section{Analysis of previous research}

In order to assay the relative biases of microbiome research it would be
necessary to compare the references to microbial phyla in published
literature to the prevalence of such microbes in microbiomes. As much
research into the human microbiome is undertaken from a medical perspective,
NCBI PubMed contains a wealth of full-text and abstracts in its catalog,
which is easily indexed for specific terms and phrases. Thus, NCBI PubMed's
article archive was chosen as a sample for language analysis.

\subsection{Methods}

A search for the exact phrase ``human microbiome'' was conducted on NCBI
PubMed in October 2019, fetching 1,779 results, which were downloaded as an
XML file. Using R version 3.6.1 \parencite{RCo19} on Microsoft Windows 10
18362.418, these results were indexed with \texttt{pubmed.mineR} version
1.0.16 \parencite{Ran19} and queried for metadata from the NCBI PubTator
service \parencite{Wei19}. PubTator analyzes the text of PubMed abstracts and
full-text articles to determine the names of species and taxonomic tanks
mentioned in such articles. Excluding mentions to \textit{Homo sapiens}, the
taxonomic ranks were looked up in the NCBI Taxonomy Database
\parencite{Nat19} using \texttt{ncbitax2lin} \parencite{Xue19} to convert the
database to CSV.

References to taxonomic ranks more specific than phylum were simplified to
phyla, and less specific references were recorded as ``\textit{domain}
(nonspecific).'' In the case of viruses, which have separate ranking system,
the NCBI value ``no rank – 0'' was used in place of phylum. The number of
references to each phylum was totaled and categorized by domain. All R code
executed is presented in \hyperref[sec:appendix_a]{Appendix A}.

\subsection{Results}

\begin{figure}
    % find this class at https://archive.danleonard.us/scholarship/coursework/coursework.cls
\documentclass{../../../coursework}

\title{Non-Bacterial Components of the Primate Microbiome:}
\subtitle{Against a Bacterial Bias}
\shorttitle{Non-Bacterial Components of the Primate Microbiome}
\author{Daniel}{Glenn}{Leonard}
\newdate{date}{18}{10}{2019}
\date{\displaydate{date}}
\course{ANTH}{438}{Primate Life History Evolution}{University of Illinois at Urbana-Champaign}
\instructor{Dr}{Kathryn}{B.}{H.}{Clancy}

\keywords{archaea, bacteria, eukarya, viruses, taxonomy}
\addbibresource[location=remote]{https://archive.danleonard.us/scholarship/coursework/illinois/ANTH/438/microbiome.bib}
\baseurl{https://archive.danleonard.us/scholarship/coursework/illinois/ANTH/438/microbiome.xhtml}

% Packages and commands for R code in Appendix A
\usepackage{color}
\usepackage{framed}
\usepackage{fancyvrb}
\newcommand{\VerbBar}{|}
\newcommand{\VERB}{\Verb[commandchars=\\\{\}]}
\DefineVerbatimEnvironment{Highlighting}{Verbatim}{commandchars=\\\{\}}
\definecolor{shadecolor}{RGB}{248,248,248}
\newenvironment{Shaded}{\begin{snugshade}}{\end{snugshade}}
\newcommand{\AlertTok}[1]{\textcolor[rgb]{0.94,0.16,0.16}{#1}}
\newcommand{\AnnotationTok}[1]{\textcolor[rgb]{0.56,0.35,0.01}{\textbf{\textit{#1}}}}
\newcommand{\AttributeTok}[1]{\textcolor[rgb]{0.77,0.63,0.00}{#1}}
\newcommand{\BaseNTok}[1]{\textcolor[rgb]{0.00,0.00,0.81}{#1}}
\newcommand{\BuiltInTok}[1]{#1}
\newcommand{\CharTok}[1]{\textcolor[rgb]{0.31,0.60,0.02}{#1}}
\newcommand{\CommentTok}[1]{\textcolor[rgb]{0.56,0.35,0.01}{\textit{#1}}}
\newcommand{\CommentVarTok}[1]{\textcolor[rgb]{0.56,0.35,0.01}{\textbf{\textit{#1}}}}
\newcommand{\ConstantTok}[1]{\textcolor[rgb]{0.00,0.00,0.00}{#1}}
\newcommand{\ControlFlowTok}[1]{\textcolor[rgb]{0.13,0.29,0.53}{\textbf{#1}}}
\newcommand{\DataTypeTok}[1]{\textcolor[rgb]{0.13,0.29,0.53}{#1}}
\newcommand{\DecValTok}[1]{\textcolor[rgb]{0.00,0.00,0.81}{#1}}
\newcommand{\DocumentationTok}[1]{\textcolor[rgb]{0.56,0.35,0.01}{\textbf{\textit{#1}}}}
\newcommand{\ErrorTok}[1]{\textcolor[rgb]{0.64,0.00,0.00}{\textbf{#1}}}
\newcommand{\ExtensionTok}[1]{#1}
\newcommand{\FloatTok}[1]{\textcolor[rgb]{0.00,0.00,0.81}{#1}}
\newcommand{\FunctionTok}[1]{\textcolor[rgb]{0.00,0.00,0.00}{#1}}
\newcommand{\ImportTok}[1]{#1}
\newcommand{\InformationTok}[1]{\textcolor[rgb]{0.56,0.35,0.01}{\textbf{\textit{#1}}}}
\newcommand{\KeywordTok}[1]{\textcolor[rgb]{0.13,0.29,0.53}{\textbf{#1}}}
\newcommand{\NormalTok}[1]{#1}
\newcommand{\OperatorTok}[1]{\textcolor[rgb]{0.81,0.36,0.00}{\textbf{#1}}}
\newcommand{\OtherTok}[1]{\textcolor[rgb]{0.56,0.35,0.01}{#1}}
\newcommand{\PreprocessorTok}[1]{\textcolor[rgb]{0.56,0.35,0.01}{\textit{#1}}}
\newcommand{\RegionMarkerTok}[1]{#1}
\newcommand{\SpecialCharTok}[1]{\textcolor[rgb]{0.00,0.00,0.00}{#1}}
\newcommand{\SpecialStringTok}[1]{\textcolor[rgb]{0.31,0.60,0.02}{#1}}
\newcommand{\StringTok}[1]{\textcolor[rgb]{0.31,0.60,0.02}{#1}}
\newcommand{\VariableTok}[1]{\textcolor[rgb]{0.00,0.00,0.00}{#1}}
\newcommand{\VerbatimStringTok}[1]{\textcolor[rgb]{0.31,0.60,0.02}{#1}}
\newcommand{\WarningTok}[1]{\textcolor[rgb]{0.56,0.35,0.01}{\textbf{\textit{#1}}}}

\begin{document}

\maketitle

\section{Introduction}

Since the discovery that the world in which we live is entirely filled with
organisms too small to see with the eye, biologists have drawn a line between
the microorganism and the multicellular organism. However, the search for the
evolutionary apex of cellular organisms has blurred this line, from the
theory of lateral gene flow \parencite{Woe98} to the discovery that archaea
and eukarya share more in common with each other than either shares with
bacteria \parencite{Woe77}. In researching the microbiome -- the physical
commensal relationship between animals and their microbial inhabitants -- it
is important to consider the full diversity of microbial taxa. This paper
seeks to investigate biases in microbiome research toward specific microbial
taxa and seek out how the less-studied varieties of microbial inhabitants may
have effects unaccounted for.

Despite a growth of research interest in archaea since its identification by
\textcite{Woe77}, little research has been conducted into their relationships
to other organisms, such as humans. As the domain was originally understood
as a clade of extremophiles due to their original isolation from geothermal
vents \parencite{DeL01}, it would be expected that archaea would be a minor
part of more familiar environments; however, it is now known that members of
archaea exist with great frequency in non-extreme environments, such as
plankton, lakes, and soils \parencite{DeL01}. Regardless, research into the
composition of human microbiota retains a strong bias toward bacteria
\parencite{Llo16, The12}. Comparative primatology has often failed to expand
microbiota research beyond bacteria, frequently excluding unknown or
infrequent phyla from results of DNA extraction
\parencite{Gar19,Mal18,Ork191,Ork19}. At its extreme, the term ``microbe''
has been used synonymously with the term ``bacteria'' despite the far more
precise definition of the latter \parencite[for an example, see][]{Mal18}.

\section{Analysis of previous research}

In order to assay the relative biases of microbiome research it would be
necessary to compare the references to microbial phyla in published
literature to the prevalence of such microbes in microbiomes. As much
research into the human microbiome is undertaken from a medical perspective,
NCBI PubMed contains a wealth of full-text and abstracts in its catalog,
which is easily indexed for specific terms and phrases. Thus, NCBI PubMed's
article archive was chosen as a sample for language analysis.

\subsection{Methods}

A search for the exact phrase ``human microbiome'' was conducted on NCBI
PubMed in October 2019, fetching 1,779 results, which were downloaded as an
XML file. Using R version 3.6.1 \parencite{RCo19} on Microsoft Windows 10
18362.418, these results were indexed with \texttt{pubmed.mineR} version
1.0.16 \parencite{Ran19} and queried for metadata from the NCBI PubTator
service \parencite{Wei19}. PubTator analyzes the text of PubMed abstracts and
full-text articles to determine the names of species and taxonomic tanks
mentioned in such articles. Excluding mentions to \textit{Homo sapiens}, the
taxonomic ranks were looked up in the NCBI Taxonomy Database
\parencite{Nat19} using \texttt{ncbitax2lin} \parencite{Xue19} to convert the
database to CSV.

References to taxonomic ranks more specific than phylum were simplified to
phyla, and less specific references were recorded as ``\textit{domain}
(nonspecific).'' In the case of viruses, which have separate ranking system,
the NCBI value ``no rank – 0'' was used in place of phylum. The number of
references to each phylum was totaled and categorized by domain. All R code
executed is presented in \hyperref[sec:appendix_a]{Appendix A}.

\subsection{Results}

\begin{figure}
    % find this class at https://archive.danleonard.us/scholarship/coursework/coursework.cls
\documentclass{../../../coursework}

\title{Non-Bacterial Components of the Primate Microbiome:}
\subtitle{Against a Bacterial Bias}
\shorttitle{Non-Bacterial Components of the Primate Microbiome}
\author{Daniel}{Glenn}{Leonard}
\newdate{date}{18}{10}{2019}
\date{\displaydate{date}}
\course{ANTH}{438}{Primate Life History Evolution}{University of Illinois at Urbana-Champaign}
\instructor{Dr}{Kathryn}{B.}{H.}{Clancy}

\keywords{archaea, bacteria, eukarya, viruses, taxonomy}
\addbibresource[location=remote]{https://archive.danleonard.us/scholarship/coursework/illinois/ANTH/438/microbiome.bib}
\baseurl{https://archive.danleonard.us/scholarship/coursework/illinois/ANTH/438/microbiome.xhtml}

% Packages and commands for R code in Appendix A
\usepackage{color}
\usepackage{framed}
\usepackage{fancyvrb}
\newcommand{\VerbBar}{|}
\newcommand{\VERB}{\Verb[commandchars=\\\{\}]}
\DefineVerbatimEnvironment{Highlighting}{Verbatim}{commandchars=\\\{\}}
\definecolor{shadecolor}{RGB}{248,248,248}
\newenvironment{Shaded}{\begin{snugshade}}{\end{snugshade}}
\newcommand{\AlertTok}[1]{\textcolor[rgb]{0.94,0.16,0.16}{#1}}
\newcommand{\AnnotationTok}[1]{\textcolor[rgb]{0.56,0.35,0.01}{\textbf{\textit{#1}}}}
\newcommand{\AttributeTok}[1]{\textcolor[rgb]{0.77,0.63,0.00}{#1}}
\newcommand{\BaseNTok}[1]{\textcolor[rgb]{0.00,0.00,0.81}{#1}}
\newcommand{\BuiltInTok}[1]{#1}
\newcommand{\CharTok}[1]{\textcolor[rgb]{0.31,0.60,0.02}{#1}}
\newcommand{\CommentTok}[1]{\textcolor[rgb]{0.56,0.35,0.01}{\textit{#1}}}
\newcommand{\CommentVarTok}[1]{\textcolor[rgb]{0.56,0.35,0.01}{\textbf{\textit{#1}}}}
\newcommand{\ConstantTok}[1]{\textcolor[rgb]{0.00,0.00,0.00}{#1}}
\newcommand{\ControlFlowTok}[1]{\textcolor[rgb]{0.13,0.29,0.53}{\textbf{#1}}}
\newcommand{\DataTypeTok}[1]{\textcolor[rgb]{0.13,0.29,0.53}{#1}}
\newcommand{\DecValTok}[1]{\textcolor[rgb]{0.00,0.00,0.81}{#1}}
\newcommand{\DocumentationTok}[1]{\textcolor[rgb]{0.56,0.35,0.01}{\textbf{\textit{#1}}}}
\newcommand{\ErrorTok}[1]{\textcolor[rgb]{0.64,0.00,0.00}{\textbf{#1}}}
\newcommand{\ExtensionTok}[1]{#1}
\newcommand{\FloatTok}[1]{\textcolor[rgb]{0.00,0.00,0.81}{#1}}
\newcommand{\FunctionTok}[1]{\textcolor[rgb]{0.00,0.00,0.00}{#1}}
\newcommand{\ImportTok}[1]{#1}
\newcommand{\InformationTok}[1]{\textcolor[rgb]{0.56,0.35,0.01}{\textbf{\textit{#1}}}}
\newcommand{\KeywordTok}[1]{\textcolor[rgb]{0.13,0.29,0.53}{\textbf{#1}}}
\newcommand{\NormalTok}[1]{#1}
\newcommand{\OperatorTok}[1]{\textcolor[rgb]{0.81,0.36,0.00}{\textbf{#1}}}
\newcommand{\OtherTok}[1]{\textcolor[rgb]{0.56,0.35,0.01}{#1}}
\newcommand{\PreprocessorTok}[1]{\textcolor[rgb]{0.56,0.35,0.01}{\textit{#1}}}
\newcommand{\RegionMarkerTok}[1]{#1}
\newcommand{\SpecialCharTok}[1]{\textcolor[rgb]{0.00,0.00,0.00}{#1}}
\newcommand{\SpecialStringTok}[1]{\textcolor[rgb]{0.31,0.60,0.02}{#1}}
\newcommand{\StringTok}[1]{\textcolor[rgb]{0.31,0.60,0.02}{#1}}
\newcommand{\VariableTok}[1]{\textcolor[rgb]{0.00,0.00,0.00}{#1}}
\newcommand{\VerbatimStringTok}[1]{\textcolor[rgb]{0.31,0.60,0.02}{#1}}
\newcommand{\WarningTok}[1]{\textcolor[rgb]{0.56,0.35,0.01}{\textbf{\textit{#1}}}}

\begin{document}

\maketitle

\section{Introduction}

Since the discovery that the world in which we live is entirely filled with
organisms too small to see with the eye, biologists have drawn a line between
the microorganism and the multicellular organism. However, the search for the
evolutionary apex of cellular organisms has blurred this line, from the
theory of lateral gene flow \parencite{Woe98} to the discovery that archaea
and eukarya share more in common with each other than either shares with
bacteria \parencite{Woe77}. In researching the microbiome -- the physical
commensal relationship between animals and their microbial inhabitants -- it
is important to consider the full diversity of microbial taxa. This paper
seeks to investigate biases in microbiome research toward specific microbial
taxa and seek out how the less-studied varieties of microbial inhabitants may
have effects unaccounted for.

Despite a growth of research interest in archaea since its identification by
\textcite{Woe77}, little research has been conducted into their relationships
to other organisms, such as humans. As the domain was originally understood
as a clade of extremophiles due to their original isolation from geothermal
vents \parencite{DeL01}, it would be expected that archaea would be a minor
part of more familiar environments; however, it is now known that members of
archaea exist with great frequency in non-extreme environments, such as
plankton, lakes, and soils \parencite{DeL01}. Regardless, research into the
composition of human microbiota retains a strong bias toward bacteria
\parencite{Llo16, The12}. Comparative primatology has often failed to expand
microbiota research beyond bacteria, frequently excluding unknown or
infrequent phyla from results of DNA extraction
\parencite{Gar19,Mal18,Ork191,Ork19}. At its extreme, the term ``microbe''
has been used synonymously with the term ``bacteria'' despite the far more
precise definition of the latter \parencite[for an example, see][]{Mal18}.

\section{Analysis of previous research}

In order to assay the relative biases of microbiome research it would be
necessary to compare the references to microbial phyla in published
literature to the prevalence of such microbes in microbiomes. As much
research into the human microbiome is undertaken from a medical perspective,
NCBI PubMed contains a wealth of full-text and abstracts in its catalog,
which is easily indexed for specific terms and phrases. Thus, NCBI PubMed's
article archive was chosen as a sample for language analysis.

\subsection{Methods}

A search for the exact phrase ``human microbiome'' was conducted on NCBI
PubMed in October 2019, fetching 1,779 results, which were downloaded as an
XML file. Using R version 3.6.1 \parencite{RCo19} on Microsoft Windows 10
18362.418, these results were indexed with \texttt{pubmed.mineR} version
1.0.16 \parencite{Ran19} and queried for metadata from the NCBI PubTator
service \parencite{Wei19}. PubTator analyzes the text of PubMed abstracts and
full-text articles to determine the names of species and taxonomic tanks
mentioned in such articles. Excluding mentions to \textit{Homo sapiens}, the
taxonomic ranks were looked up in the NCBI Taxonomy Database
\parencite{Nat19} using \texttt{ncbitax2lin} \parencite{Xue19} to convert the
database to CSV.

References to taxonomic ranks more specific than phylum were simplified to
phyla, and less specific references were recorded as ``\textit{domain}
(nonspecific).'' In the case of viruses, which have separate ranking system,
the NCBI value ``no rank – 0'' was used in place of phylum. The number of
references to each phylum was totaled and categorized by domain. All R code
executed is presented in \hyperref[sec:appendix_a]{Appendix A}.

\subsection{Results}

\begin{figure}
    % find this class at https://archive.danleonard.us/scholarship/coursework/coursework.cls
\documentclass{../../../coursework}

\title{Non-Bacterial Components of the Primate Microbiome:}
\subtitle{Against a Bacterial Bias}
\shorttitle{Non-Bacterial Components of the Primate Microbiome}
\author{Daniel}{Glenn}{Leonard}
\newdate{date}{18}{10}{2019}
\date{\displaydate{date}}
\course{ANTH}{438}{Primate Life History Evolution}{University of Illinois at Urbana-Champaign}
\instructor{Dr}{Kathryn}{B.}{H.}{Clancy}

\keywords{archaea, bacteria, eukarya, viruses, taxonomy}
\addbibresource[location=remote]{https://archive.danleonard.us/scholarship/coursework/illinois/ANTH/438/microbiome.bib}
\baseurl{https://archive.danleonard.us/scholarship/coursework/illinois/ANTH/438/microbiome.xhtml}

% Packages and commands for R code in Appendix A
\usepackage{color}
\usepackage{framed}
\usepackage{fancyvrb}
\newcommand{\VerbBar}{|}
\newcommand{\VERB}{\Verb[commandchars=\\\{\}]}
\DefineVerbatimEnvironment{Highlighting}{Verbatim}{commandchars=\\\{\}}
\definecolor{shadecolor}{RGB}{248,248,248}
\newenvironment{Shaded}{\begin{snugshade}}{\end{snugshade}}
\newcommand{\AlertTok}[1]{\textcolor[rgb]{0.94,0.16,0.16}{#1}}
\newcommand{\AnnotationTok}[1]{\textcolor[rgb]{0.56,0.35,0.01}{\textbf{\textit{#1}}}}
\newcommand{\AttributeTok}[1]{\textcolor[rgb]{0.77,0.63,0.00}{#1}}
\newcommand{\BaseNTok}[1]{\textcolor[rgb]{0.00,0.00,0.81}{#1}}
\newcommand{\BuiltInTok}[1]{#1}
\newcommand{\CharTok}[1]{\textcolor[rgb]{0.31,0.60,0.02}{#1}}
\newcommand{\CommentTok}[1]{\textcolor[rgb]{0.56,0.35,0.01}{\textit{#1}}}
\newcommand{\CommentVarTok}[1]{\textcolor[rgb]{0.56,0.35,0.01}{\textbf{\textit{#1}}}}
\newcommand{\ConstantTok}[1]{\textcolor[rgb]{0.00,0.00,0.00}{#1}}
\newcommand{\ControlFlowTok}[1]{\textcolor[rgb]{0.13,0.29,0.53}{\textbf{#1}}}
\newcommand{\DataTypeTok}[1]{\textcolor[rgb]{0.13,0.29,0.53}{#1}}
\newcommand{\DecValTok}[1]{\textcolor[rgb]{0.00,0.00,0.81}{#1}}
\newcommand{\DocumentationTok}[1]{\textcolor[rgb]{0.56,0.35,0.01}{\textbf{\textit{#1}}}}
\newcommand{\ErrorTok}[1]{\textcolor[rgb]{0.64,0.00,0.00}{\textbf{#1}}}
\newcommand{\ExtensionTok}[1]{#1}
\newcommand{\FloatTok}[1]{\textcolor[rgb]{0.00,0.00,0.81}{#1}}
\newcommand{\FunctionTok}[1]{\textcolor[rgb]{0.00,0.00,0.00}{#1}}
\newcommand{\ImportTok}[1]{#1}
\newcommand{\InformationTok}[1]{\textcolor[rgb]{0.56,0.35,0.01}{\textbf{\textit{#1}}}}
\newcommand{\KeywordTok}[1]{\textcolor[rgb]{0.13,0.29,0.53}{\textbf{#1}}}
\newcommand{\NormalTok}[1]{#1}
\newcommand{\OperatorTok}[1]{\textcolor[rgb]{0.81,0.36,0.00}{\textbf{#1}}}
\newcommand{\OtherTok}[1]{\textcolor[rgb]{0.56,0.35,0.01}{#1}}
\newcommand{\PreprocessorTok}[1]{\textcolor[rgb]{0.56,0.35,0.01}{\textit{#1}}}
\newcommand{\RegionMarkerTok}[1]{#1}
\newcommand{\SpecialCharTok}[1]{\textcolor[rgb]{0.00,0.00,0.00}{#1}}
\newcommand{\SpecialStringTok}[1]{\textcolor[rgb]{0.31,0.60,0.02}{#1}}
\newcommand{\StringTok}[1]{\textcolor[rgb]{0.31,0.60,0.02}{#1}}
\newcommand{\VariableTok}[1]{\textcolor[rgb]{0.00,0.00,0.00}{#1}}
\newcommand{\VerbatimStringTok}[1]{\textcolor[rgb]{0.31,0.60,0.02}{#1}}
\newcommand{\WarningTok}[1]{\textcolor[rgb]{0.56,0.35,0.01}{\textbf{\textit{#1}}}}

\begin{document}

\maketitle

\section{Introduction}

Since the discovery that the world in which we live is entirely filled with
organisms too small to see with the eye, biologists have drawn a line between
the microorganism and the multicellular organism. However, the search for the
evolutionary apex of cellular organisms has blurred this line, from the
theory of lateral gene flow \parencite{Woe98} to the discovery that archaea
and eukarya share more in common with each other than either shares with
bacteria \parencite{Woe77}. In researching the microbiome -- the physical
commensal relationship between animals and their microbial inhabitants -- it
is important to consider the full diversity of microbial taxa. This paper
seeks to investigate biases in microbiome research toward specific microbial
taxa and seek out how the less-studied varieties of microbial inhabitants may
have effects unaccounted for.

Despite a growth of research interest in archaea since its identification by
\textcite{Woe77}, little research has been conducted into their relationships
to other organisms, such as humans. As the domain was originally understood
as a clade of extremophiles due to their original isolation from geothermal
vents \parencite{DeL01}, it would be expected that archaea would be a minor
part of more familiar environments; however, it is now known that members of
archaea exist with great frequency in non-extreme environments, such as
plankton, lakes, and soils \parencite{DeL01}. Regardless, research into the
composition of human microbiota retains a strong bias toward bacteria
\parencite{Llo16, The12}. Comparative primatology has often failed to expand
microbiota research beyond bacteria, frequently excluding unknown or
infrequent phyla from results of DNA extraction
\parencite{Gar19,Mal18,Ork191,Ork19}. At its extreme, the term ``microbe''
has been used synonymously with the term ``bacteria'' despite the far more
precise definition of the latter \parencite[for an example, see][]{Mal18}.

\section{Analysis of previous research}

In order to assay the relative biases of microbiome research it would be
necessary to compare the references to microbial phyla in published
literature to the prevalence of such microbes in microbiomes. As much
research into the human microbiome is undertaken from a medical perspective,
NCBI PubMed contains a wealth of full-text and abstracts in its catalog,
which is easily indexed for specific terms and phrases. Thus, NCBI PubMed's
article archive was chosen as a sample for language analysis.

\subsection{Methods}

A search for the exact phrase ``human microbiome'' was conducted on NCBI
PubMed in October 2019, fetching 1,779 results, which were downloaded as an
XML file. Using R version 3.6.1 \parencite{RCo19} on Microsoft Windows 10
18362.418, these results were indexed with \texttt{pubmed.mineR} version
1.0.16 \parencite{Ran19} and queried for metadata from the NCBI PubTator
service \parencite{Wei19}. PubTator analyzes the text of PubMed abstracts and
full-text articles to determine the names of species and taxonomic tanks
mentioned in such articles. Excluding mentions to \textit{Homo sapiens}, the
taxonomic ranks were looked up in the NCBI Taxonomy Database
\parencite{Nat19} using \texttt{ncbitax2lin} \parencite{Xue19} to convert the
database to CSV.

References to taxonomic ranks more specific than phylum were simplified to
phyla, and less specific references were recorded as ``\textit{domain}
(nonspecific).'' In the case of viruses, which have separate ranking system,
the NCBI value ``no rank – 0'' was used in place of phylum. The number of
references to each phylum was totaled and categorized by domain. All R code
executed is presented in \hyperref[sec:appendix_a]{Appendix A}.

\subsection{Results}

\begin{figure}
    \input{microbiome.eps_tex}
    \caption{References to phyla in a PubMed search for the phrase
             ``human microbiome.'' References less specific than phylum are
             recorded as ``\textit{domain} (nonspecific),'' domains are
             presented as colors. All references to humans were excluded, and
             would have been sorted into Chordata.}
    \label{fig:1}
\end{figure}

Of the 1,779 articles indexed, 1,730 mentioned a taxonomic rank or species.
This totaled 5,354 individual mentions, of which 2,419 were to species or
ranks other than humans. As visible in \hyperref[fig:1]{Figure 1}, bacteria
made up the majority of references, with the phyla Proteobacteria and
Firmicutes second only to nonspecific references to bacteria as a domain.
Eukarya were second to bacteria in total number of references; however, the
most frequently discussed phylum within the domain was Chordata, which likely
stems from discussion of vertebrate microbiomes such as Mus musculus rather
than indicating the presence of chordates in the human microbiome. Viruses
were referenced moderately frequently, with nonspecific virus references
making up the greatest proportion of the domain's coverage, followed by
bacteriophages. Finally, archaea were referenced least frequently of all,
with nine total references amounting to less than any viral phylum.

\section{The Role of Non-Bacterial Organisms in the Microbiome}

\subsection{Eukarya}

Yeasts have been found to colonize infant guts gradually, with an isolation
rate of 13\% at 28-46 days old \parencite{Ben84} and 50\% as
soon as four months of age \parencite{Ell75}.
Notably, this rate of colonization was consistent between breast-fed and
formula-fed infants, in contrast to the colonization rates of bacterial phyla
which differ significantly between the two feeding mechanisms. From a
life-history perspective, this presents a novel question of how yeast
colonize the infant gut and whether the mechanism by which breastmilk
contains microbiota associated with the mother's gut microbiome
\parencite{Mar13} is also capable of providing yeast, whose eukaryotic cells
differ in size by orders of magnitude from prokaryotic cells.

However, research conducted on yeast inhabitants of animals focuses primarily
on their pathological role. The genus \textit{Candida} is well known for its
role in infectious disease in humans, particularly in the immunocompromised
\parencite{Kou11}. However, species of yeast, including many
\textit{Candida}~spp., have frequently been identified on the human body,
including the vagina, gut, and skin, without any associated mycosis
\parencite{Man10}. Even for species known to be infectious agents, between
12\% \parencite{Cho86} and 80\% \parencite{Soe07} of asymptomatic healthy
women are vaginal carriers. In place of viewing yeast through the lens of
infection, it should be considered as a part of the ecosystem that makes up
the human body. How it reaches and colonizes infants, and why its prevalence
makes up far less of the population than do bacterial commensals, merits
further study.

\subsection{Archaea}

In 1966 it was discovered that humans both respire and flatulate methane,
which is not known to be biologically produced by any bacteria or eukaryote.
Following this observation, methanogenic prokaryotes, from what would later
be called the archaea, were identified in human feces \parencite{Not68}. The
archaeal domain remains the only taxon with members known to be biological
producers of methane \parencite{Tha06}, and as such is unique in its
residence within the human gut. Despite widespread colonization by a diverse
array of archaeal species, the ecological niche which they occupy within the
human body and whether they are transferred vertically or environmentally
remains unknown \parencite{Dri11}. It has been theorized that many more
lineages of archaea remain undiscovered, as the DNA analysis methods used to
identify bacterial genomes are incompatible with archaeal cell walls
\parencite{Hor15}.

\subsection{Viruses}

Despite not fitting many definitions of biological life, viruses are the most
diverse form of biological material on the planet. They are known to
outnumber bacteria by a factor of nearly ten \parencite{Ore97} and are a
major driver of evolution in prokaryotes. Although humans are susceptible to
a wide variety of viruses, the viral community in the microbiome, so fast it
has itself been called a ``virome'' \parencite{Wyl12}, is dominated primarily
by bacteriophages whose sequences are novel \parencite{Dut14}. The phages
identifiable by genetic sequence are primarily those known to attack
bacterial phyla associated with the microbiome, yet appear to be vastly more
diverse than the virome of the ocean \parencite{Wal14}. Unlike the theories
of bacterial colonization, how such a vast and dynamic virome is able to
establish itself by adulthood remains unknown. The effect of age, and whether
milk carries a virome of similar diversity would be intriguing avenues for
further study.

\section{Limitations}

Utilizing both life-history and ecological frameworks for approaching
non-bacterial components of the microbiome, it is clear that the roles such
taxa play within our bodies is underexplored. While this paper sought to
touch on these issues, it is limited by both methodology and its brevity. The
use of NCBI PubMed for language analysis suffers from a bias toward medical
research, and so microbiome research will often consider the roles of known
pathological agents rather than in-depth discussions of all possible
inhabitants. Eukarya is also a domain of organisms highly diverse in physical
makeup and metabolic strategy, while only yeasts were discussed here. The
roles of other unicellular parasites (for instance,
\textit{Toxoplasma gondii}) and of helminthic worms, which each form diverse
monophyletic groups, can also be explored via the lens of the microbiome's
life-history. More research is needed in this area to investigate how the
diverse community of microbes within humans evolved and what role it plays
even in the absence of pathology.

\printbibliography

\setcounter{secnumdepth}{0}

\newpage
\section{Appendix A}
\label{sec:appendix_a}

\begin{Shaded}
    \begin{Highlighting}[]
    \CommentTok{\# R script to analyze PubMed abstracts}
    \CommentTok{\# Author: Dan Leonard}
    
    \CommentTok{\# Import pubmed analysis library}
    \CommentTok{\# \parencite{Ran19}}
    \KeywordTok{library}\NormalTok{(pubmed.mineR)}
    \CommentTok{\# Import file{-}reading library}
    \CommentTok{\# \parencite{Wic18}}
    \KeywordTok{library}\NormalTok{(readr)}
    
    \CommentTok{\# Load XML search result file from PubMed into list}
    \NormalTok{abstracts <{-}}\StringTok{ }\KeywordTok{xmlreadabs}\NormalTok{(}\StringTok{"pubmed\_result.xml"}\NormalTok{)}
    
    \CommentTok{\# Load NCBI lineage data into list}
    \CommentTok{\# \parencite{Xue19}}
    \NormalTok{lineages <{-}}\StringTok{ }\KeywordTok{read\_csv}\NormalTok{(}\StringTok{"lineages{-}2017{-}03{-}17.csv"}\NormalTok{,}
        \DataTypeTok{col\_types =} \KeywordTok{cols}\NormalTok{(}
            \DataTypeTok{.default =} \KeywordTok{col\_character}\NormalTok{(),}
            \DataTypeTok{tax\_id =} \KeywordTok{col\_integer}\NormalTok{()}
    \NormalTok{    )}
    \NormalTok{)}
    
    \CommentTok{\# Run PMIDs through PubTator}
    \NormalTok{pubtators <{-}}\StringTok{ }\KeywordTok{lapply}\NormalTok{(abstracts}\OperatorTok{@}\NormalTok{PMID, pubtator\_function)}
    \CommentTok{\# Remove [" No Data "] from PubTator results}
    \NormalTok{pubtators <{-}}\StringTok{ }\NormalTok{pubtators [}\OperatorTok{!}\StringTok{ }\NormalTok{pubtators }\OperatorTok{\%in\%}\StringTok{ }\KeywordTok{list}\NormalTok{(}\StringTok{" No Data "}\NormalTok{)]}
    
    \CommentTok{\# Get list of vectors of species names}
    \CommentTok{\# Species names are located in column 5}
    \NormalTok{species <{-}}\StringTok{ }\KeywordTok{sapply}\NormalTok{(pubtators, }\StringTok{"[["}\NormalTok{, }\DecValTok{5}\NormalTok{)}
    \CommentTok{\# Remove nulls}
    \NormalTok{species [}\KeywordTok{sapply}\NormalTok{(species, is.null)] <{-}}\StringTok{ }\OtherTok{NULL}
    \CommentTok{\# Flatten}
    \NormalTok{species <{-}}\StringTok{ }\KeywordTok{unlist}\NormalTok{(species)}
    
    \CommentTok{\# Remove species names, leave NCBI numerical ID}
    \NormalTok{species <{-}}\StringTok{ }\KeywordTok{sapply}\NormalTok{(species, }\ControlFlowTok{function}\NormalTok{(x) }\KeywordTok{sapply}\NormalTok{(}\KeywordTok{strsplit}\NormalTok{(x, }\StringTok{">"}\NormalTok{), }\StringTok{"[["}\NormalTok{, }\DecValTok{2}\NormalTok{))}
    \CommentTok{\# Convert to numeric form}
    \NormalTok{species <{-}}\StringTok{ }\KeywordTok{as.numeric}\NormalTok{(species)}
    \CommentTok{\# Remove references to Homo sapiens (NCBI ID 9606)}
    \NormalTok{species.nohuman <{-}}\StringTok{ }\NormalTok{species [}\OperatorTok{!}\StringTok{ }\NormalTok{species }\OperatorTok{\%in\%}\StringTok{ }\DecValTok{9606}\NormalTok{]}
    
    \CommentTok{\# Get domain names}
    \NormalTok{species.nohuman.domains <{-}}
    \StringTok{    }\NormalTok{lineages}\OperatorTok{$}\NormalTok{superkingdom[}\KeywordTok{match}\NormalTok{(species.nohuman, lineages}\OperatorTok{$}\NormalTok{tax\_id)]}
    \CommentTok{\# Get phyla names}
    \NormalTok{species.nohuman.phyla <{-}}
    \StringTok{    }\NormalTok{lineages}\OperatorTok{$}\NormalTok{phylum[}\KeywordTok{match}\NormalTok{(species.nohuman, lineages}\OperatorTok{$}\NormalTok{tax\_id)]}
    
    \CommentTok{\# Create data frame}
    \NormalTok{phylogeny <{-}}\StringTok{ }\KeywordTok{data.frame}\NormalTok{(}
        \DataTypeTok{Domain=}
            \KeywordTok{c}\NormalTok{(}
    \NormalTok{            lineages}\OperatorTok{$}\NormalTok{superkingdom[}
                    \KeywordTok{match}\NormalTok{(species.nohuman, lineages}\OperatorTok{$}\NormalTok{tax\_id)}
    \NormalTok{            ]}
    \NormalTok{        ),}
        \DataTypeTok{Phylum=}
            \KeywordTok{c}\NormalTok{(}
    \NormalTok{            lineages}\OperatorTok{$}\NormalTok{phylum[}
                    \KeywordTok{match}\NormalTok{(species.nohuman, lineages}\OperatorTok{$}\NormalTok{tax\_id)}
    \NormalTok{            ]}
    \NormalTok{        ),}
        \DataTypeTok{Norank=}
            \KeywordTok{c}\NormalTok{(}
    \NormalTok{            lineages}\OperatorTok{$}\StringTok{`}\DataTypeTok{no rank}\StringTok{`}\NormalTok{[}
                    \KeywordTok{match}\NormalTok{(species.nohuman, lineages}\OperatorTok{$}\NormalTok{tax\_id)}
    \NormalTok{            ]}
    \NormalTok{        ),}
        \DataTypeTok{stringsAsFactors=}\OtherTok{FALSE}
    \NormalTok{)}
    \CommentTok{\# Add additional column "Name"}
    \NormalTok{phylogeny [ , }\KeywordTok{c}\NormalTok{(}\StringTok{"Name"}\NormalTok{)] <{-}}\StringTok{ }\OtherTok{NA}
    \CommentTok{\# Remove unhelpful term "cellular organisms"}
    \NormalTok{phylogeny}\OperatorTok{$}\NormalTok{Norank[phylogeny}\OperatorTok{$}\NormalTok{Norank }\OperatorTok{==}\StringTok{ "cellular organisms"}\NormalTok{] <{-}}\StringTok{ }\OtherTok{NA}
    \CommentTok{\# Use virus types as name if present}
    \NormalTok{phylogeny}\OperatorTok{$}\NormalTok{Name <{-}}\StringTok{ }\NormalTok{phylogeny}\OperatorTok{$}\NormalTok{Norank}
    \CommentTok{\# Use phylum as name if present}
    \NormalTok{phylogeny}\OperatorTok{$}\NormalTok{Name[}\KeywordTok{is.na}\NormalTok{(phylogeny}\OperatorTok{$}\NormalTok{Name)] <{-}}
    \StringTok{    }\KeywordTok{as.character}\NormalTok{(phylogeny}\OperatorTok{$}\NormalTok{Phylum[}\KeywordTok{is.na}\NormalTok{(phylogeny}\OperatorTok{$}\NormalTok{Name)])}
    \CommentTok{\# Use "<domain> (nonspecific)" as name if previous two not present}
    \NormalTok{phylogeny}\OperatorTok{$}\NormalTok{Name[}\KeywordTok{is.na}\NormalTok{(phylogeny}\OperatorTok{$}\NormalTok{Name)] <{-}}
    \StringTok{    }\KeywordTok{paste}\NormalTok{(}
            \KeywordTok{as.character}\NormalTok{(phylogeny}\OperatorTok{$}\NormalTok{Domain[}\KeywordTok{is.na}\NormalTok{(phylogeny}\OperatorTok{$}\NormalTok{Name)]),}
            \StringTok{"(nonspecific)"}\NormalTok{,}
            \DataTypeTok{sep=}\StringTok{" "}
    \NormalTok{    )}
    \CommentTok{\# Remove phylum and virus type columns}
    \NormalTok{phylogeny <{-}}
    \StringTok{    }\KeywordTok{data.frame}\NormalTok{(}
            \DataTypeTok{Domain=}\NormalTok{phylogeny}\OperatorTok{$}\NormalTok{Domain,}
            \DataTypeTok{Name=}\NormalTok{phylogeny}\OperatorTok{$}\NormalTok{Name,}
            \DataTypeTok{stringsAsFactors=}\OtherTok{FALSE}
    \NormalTok{    )}
    \CommentTok{\# Use table() to count occurrences}
    \NormalTok{phylogeny <{-}}\StringTok{ }\KeywordTok{as.data.frame}\NormalTok{(}\KeywordTok{table}\NormalTok{(phylogeny))}
    \CommentTok{\# Remove extraneous values}
    \NormalTok{phylogeny <{-}}\StringTok{ }\KeywordTok{subset}\NormalTok{(phylogeny, Freq }\OperatorTok{!=}\StringTok{ }\DecValTok{0}\NormalTok{)}
    \CommentTok{\# Sort}
    \NormalTok{phylogeny <{-}}\StringTok{ }\NormalTok{phylogeny[}\KeywordTok{order}\NormalTok{(phylogeny}\OperatorTok{$}\NormalTok{Domain, }\OperatorTok{{-}}\NormalTok{phylogeny}\OperatorTok{$}\NormalTok{Freq),]}
    
    \CommentTok{\# Create dictionary for looking up colors}
    \NormalTok{colors <{-}}
    \StringTok{    }\KeywordTok{data.frame}\NormalTok{(}
            \DataTypeTok{Colors=}\KeywordTok{c}\NormalTok{(}\StringTok{"Red"}\NormalTok{, }\StringTok{"Green"}\NormalTok{, }\StringTok{"Yellow"}\NormalTok{, }\StringTok{"Blue"}\NormalTok{),}
            \DataTypeTok{Domains=}\KeywordTok{c}\NormalTok{(}\StringTok{"Archaea"}\NormalTok{, }\StringTok{"Bacteria"}\NormalTok{, }\StringTok{"Eukaryota"}\NormalTok{, }\StringTok{"Viruses"}\NormalTok{),}
            \DataTypeTok{stringsAsFactors=}\OtherTok{FALSE}
    \NormalTok{    )}
    
    \CommentTok{\# Create list of colors matching domains}
    \NormalTok{cols <{-}}
    \StringTok{    }\NormalTok{colors}\OperatorTok{$}\NormalTok{Colors[}
            \KeywordTok{match}\NormalTok{(phylogeny}\OperatorTok{$}\NormalTok{Domain, colors}\OperatorTok{$}\NormalTok{Domains)}
    \NormalTok{    ]}
    
    \CommentTok{\# Set margins}
    \KeywordTok{par}\NormalTok{(}\DataTypeTok{mar=}\KeywordTok{c}\NormalTok{(}\DecValTok{7}\NormalTok{,}\DecValTok{4}\NormalTok{,}\DecValTok{4}\NormalTok{,}\DecValTok{1}\NormalTok{)}\OperatorTok{+}\FloatTok{0.1}\NormalTok{)}
    \CommentTok{\# Print phylum plot}
    \NormalTok{plot <{-}}\StringTok{ }\KeywordTok{barplot}\NormalTok{(}
        \DataTypeTok{height =}\NormalTok{ phylogeny}\OperatorTok{$}\NormalTok{Freq,}
        \DataTypeTok{ylab =} \StringTok{"log References"}\NormalTok{,}
        \DataTypeTok{main =} \StringTok{"References to specific phyla in PubMed search for }\CharTok{\textbackslash{}"}\StringTok{Human Microbiome}\CharTok{\textbackslash{}"}\StringTok{"}\NormalTok{,}
        \DataTypeTok{col =}\NormalTok{ cols,}
        \DataTypeTok{names.arg =}\NormalTok{ phylogeny}\OperatorTok{$}\NormalTok{Name,}
        \DataTypeTok{log =} \StringTok{"y"}\NormalTok{,}
        \DataTypeTok{xaxt =} \StringTok{"n"}
    \NormalTok{)}
    \CommentTok{\# Add X{-}axis labels}
    \KeywordTok{text}\NormalTok{(}
    \NormalTok{    plot,}
        \FloatTok{0.95}\NormalTok{,}
        \DataTypeTok{labels =}\NormalTok{ phylogeny}\OperatorTok{$}\NormalTok{Name,}
        \DataTypeTok{srt =} \DecValTok{45}\NormalTok{,}
        \DataTypeTok{adj =} \KeywordTok{c}\NormalTok{(}\FloatTok{1.1}\NormalTok{,}\FloatTok{1.1}\NormalTok{),}
        \DataTypeTok{xpd =} \OtherTok{TRUE}\NormalTok{,}
        \DataTypeTok{cex =} \FloatTok{0.8}
    \NormalTok{)}
    \CommentTok{\# Add legend for domain colors}
    \KeywordTok{legend}\NormalTok{(}
        \StringTok{"topright"}\NormalTok{,}
    \NormalTok{    plot,}
        \DataTypeTok{legend =}\NormalTok{ colors}\OperatorTok{$}\NormalTok{Domains,}
        \DataTypeTok{fill =}\NormalTok{ colors}\OperatorTok{$}\NormalTok{Colors}
    \NormalTok{)}
    \end{Highlighting}
    \end{Shaded}

\end{document}
    \caption{References to phyla in a PubMed search for the phrase
             ``human microbiome.'' References less specific than phylum are
             recorded as ``\textit{domain} (nonspecific),'' domains are
             presented as colors. All references to humans were excluded, and
             would have been sorted into Chordata.}
    \label{fig:1}
\end{figure}

Of the 1,779 articles indexed, 1,730 mentioned a taxonomic rank or species.
This totaled 5,354 individual mentions, of which 2,419 were to species or
ranks other than humans. As visible in \hyperref[fig:1]{Figure 1}, bacteria
made up the majority of references, with the phyla Proteobacteria and
Firmicutes second only to nonspecific references to bacteria as a domain.
Eukarya were second to bacteria in total number of references; however, the
most frequently discussed phylum within the domain was Chordata, which likely
stems from discussion of vertebrate microbiomes such as Mus musculus rather
than indicating the presence of chordates in the human microbiome. Viruses
were referenced moderately frequently, with nonspecific virus references
making up the greatest proportion of the domain's coverage, followed by
bacteriophages. Finally, archaea were referenced least frequently of all,
with nine total references amounting to less than any viral phylum.

\section{The Role of Non-Bacterial Organisms in the Microbiome}

\subsection{Eukarya}

Yeasts have been found to colonize infant guts gradually, with an isolation
rate of 13\% at 28-46 days old \parencite{Ben84} and 50\% as
soon as four months of age \parencite{Ell75}.
Notably, this rate of colonization was consistent between breast-fed and
formula-fed infants, in contrast to the colonization rates of bacterial phyla
which differ significantly between the two feeding mechanisms. From a
life-history perspective, this presents a novel question of how yeast
colonize the infant gut and whether the mechanism by which breastmilk
contains microbiota associated with the mother's gut microbiome
\parencite{Mar13} is also capable of providing yeast, whose eukaryotic cells
differ in size by orders of magnitude from prokaryotic cells.

However, research conducted on yeast inhabitants of animals focuses primarily
on their pathological role. The genus \textit{Candida} is well known for its
role in infectious disease in humans, particularly in the immunocompromised
\parencite{Kou11}. However, species of yeast, including many
\textit{Candida}~spp., have frequently been identified on the human body,
including the vagina, gut, and skin, without any associated mycosis
\parencite{Man10}. Even for species known to be infectious agents, between
12\% \parencite{Cho86} and 80\% \parencite{Soe07} of asymptomatic healthy
women are vaginal carriers. In place of viewing yeast through the lens of
infection, it should be considered as a part of the ecosystem that makes up
the human body. How it reaches and colonizes infants, and why its prevalence
makes up far less of the population than do bacterial commensals, merits
further study.

\subsection{Archaea}

In 1966 it was discovered that humans both respire and flatulate methane,
which is not known to be biologically produced by any bacteria or eukaryote.
Following this observation, methanogenic prokaryotes, from what would later
be called the archaea, were identified in human feces \parencite{Not68}. The
archaeal domain remains the only taxon with members known to be biological
producers of methane \parencite{Tha06}, and as such is unique in its
residence within the human gut. Despite widespread colonization by a diverse
array of archaeal species, the ecological niche which they occupy within the
human body and whether they are transferred vertically or environmentally
remains unknown \parencite{Dri11}. It has been theorized that many more
lineages of archaea remain undiscovered, as the DNA analysis methods used to
identify bacterial genomes are incompatible with archaeal cell walls
\parencite{Hor15}.

\subsection{Viruses}

Despite not fitting many definitions of biological life, viruses are the most
diverse form of biological material on the planet. They are known to
outnumber bacteria by a factor of nearly ten \parencite{Ore97} and are a
major driver of evolution in prokaryotes. Although humans are susceptible to
a wide variety of viruses, the viral community in the microbiome, so fast it
has itself been called a ``virome'' \parencite{Wyl12}, is dominated primarily
by bacteriophages whose sequences are novel \parencite{Dut14}. The phages
identifiable by genetic sequence are primarily those known to attack
bacterial phyla associated with the microbiome, yet appear to be vastly more
diverse than the virome of the ocean \parencite{Wal14}. Unlike the theories
of bacterial colonization, how such a vast and dynamic virome is able to
establish itself by adulthood remains unknown. The effect of age, and whether
milk carries a virome of similar diversity would be intriguing avenues for
further study.

\section{Limitations}

Utilizing both life-history and ecological frameworks for approaching
non-bacterial components of the microbiome, it is clear that the roles such
taxa play within our bodies is underexplored. While this paper sought to
touch on these issues, it is limited by both methodology and its brevity. The
use of NCBI PubMed for language analysis suffers from a bias toward medical
research, and so microbiome research will often consider the roles of known
pathological agents rather than in-depth discussions of all possible
inhabitants. Eukarya is also a domain of organisms highly diverse in physical
makeup and metabolic strategy, while only yeasts were discussed here. The
roles of other unicellular parasites (for instance,
\textit{Toxoplasma gondii}) and of helminthic worms, which each form diverse
monophyletic groups, can also be explored via the lens of the microbiome's
life-history. More research is needed in this area to investigate how the
diverse community of microbes within humans evolved and what role it plays
even in the absence of pathology.

\printbibliography

\setcounter{secnumdepth}{0}

\newpage
\section{Appendix A}
\label{sec:appendix_a}

\begin{Shaded}
    \begin{Highlighting}[]
    \CommentTok{\# R script to analyze PubMed abstracts}
    \CommentTok{\# Author: Dan Leonard}
    
    \CommentTok{\# Import pubmed analysis library}
    \CommentTok{\# \parencite{Ran19}}
    \KeywordTok{library}\NormalTok{(pubmed.mineR)}
    \CommentTok{\# Import file{-}reading library}
    \CommentTok{\# \parencite{Wic18}}
    \KeywordTok{library}\NormalTok{(readr)}
    
    \CommentTok{\# Load XML search result file from PubMed into list}
    \NormalTok{abstracts <{-}}\StringTok{ }\KeywordTok{xmlreadabs}\NormalTok{(}\StringTok{"pubmed\_result.xml"}\NormalTok{)}
    
    \CommentTok{\# Load NCBI lineage data into list}
    \CommentTok{\# \parencite{Xue19}}
    \NormalTok{lineages <{-}}\StringTok{ }\KeywordTok{read\_csv}\NormalTok{(}\StringTok{"lineages{-}2017{-}03{-}17.csv"}\NormalTok{,}
        \DataTypeTok{col\_types =} \KeywordTok{cols}\NormalTok{(}
            \DataTypeTok{.default =} \KeywordTok{col\_character}\NormalTok{(),}
            \DataTypeTok{tax\_id =} \KeywordTok{col\_integer}\NormalTok{()}
    \NormalTok{    )}
    \NormalTok{)}
    
    \CommentTok{\# Run PMIDs through PubTator}
    \NormalTok{pubtators <{-}}\StringTok{ }\KeywordTok{lapply}\NormalTok{(abstracts}\OperatorTok{@}\NormalTok{PMID, pubtator\_function)}
    \CommentTok{\# Remove [" No Data "] from PubTator results}
    \NormalTok{pubtators <{-}}\StringTok{ }\NormalTok{pubtators [}\OperatorTok{!}\StringTok{ }\NormalTok{pubtators }\OperatorTok{\%in\%}\StringTok{ }\KeywordTok{list}\NormalTok{(}\StringTok{" No Data "}\NormalTok{)]}
    
    \CommentTok{\# Get list of vectors of species names}
    \CommentTok{\# Species names are located in column 5}
    \NormalTok{species <{-}}\StringTok{ }\KeywordTok{sapply}\NormalTok{(pubtators, }\StringTok{"[["}\NormalTok{, }\DecValTok{5}\NormalTok{)}
    \CommentTok{\# Remove nulls}
    \NormalTok{species [}\KeywordTok{sapply}\NormalTok{(species, is.null)] <{-}}\StringTok{ }\OtherTok{NULL}
    \CommentTok{\# Flatten}
    \NormalTok{species <{-}}\StringTok{ }\KeywordTok{unlist}\NormalTok{(species)}
    
    \CommentTok{\# Remove species names, leave NCBI numerical ID}
    \NormalTok{species <{-}}\StringTok{ }\KeywordTok{sapply}\NormalTok{(species, }\ControlFlowTok{function}\NormalTok{(x) }\KeywordTok{sapply}\NormalTok{(}\KeywordTok{strsplit}\NormalTok{(x, }\StringTok{">"}\NormalTok{), }\StringTok{"[["}\NormalTok{, }\DecValTok{2}\NormalTok{))}
    \CommentTok{\# Convert to numeric form}
    \NormalTok{species <{-}}\StringTok{ }\KeywordTok{as.numeric}\NormalTok{(species)}
    \CommentTok{\# Remove references to Homo sapiens (NCBI ID 9606)}
    \NormalTok{species.nohuman <{-}}\StringTok{ }\NormalTok{species [}\OperatorTok{!}\StringTok{ }\NormalTok{species }\OperatorTok{\%in\%}\StringTok{ }\DecValTok{9606}\NormalTok{]}
    
    \CommentTok{\# Get domain names}
    \NormalTok{species.nohuman.domains <{-}}
    \StringTok{    }\NormalTok{lineages}\OperatorTok{$}\NormalTok{superkingdom[}\KeywordTok{match}\NormalTok{(species.nohuman, lineages}\OperatorTok{$}\NormalTok{tax\_id)]}
    \CommentTok{\# Get phyla names}
    \NormalTok{species.nohuman.phyla <{-}}
    \StringTok{    }\NormalTok{lineages}\OperatorTok{$}\NormalTok{phylum[}\KeywordTok{match}\NormalTok{(species.nohuman, lineages}\OperatorTok{$}\NormalTok{tax\_id)]}
    
    \CommentTok{\# Create data frame}
    \NormalTok{phylogeny <{-}}\StringTok{ }\KeywordTok{data.frame}\NormalTok{(}
        \DataTypeTok{Domain=}
            \KeywordTok{c}\NormalTok{(}
    \NormalTok{            lineages}\OperatorTok{$}\NormalTok{superkingdom[}
                    \KeywordTok{match}\NormalTok{(species.nohuman, lineages}\OperatorTok{$}\NormalTok{tax\_id)}
    \NormalTok{            ]}
    \NormalTok{        ),}
        \DataTypeTok{Phylum=}
            \KeywordTok{c}\NormalTok{(}
    \NormalTok{            lineages}\OperatorTok{$}\NormalTok{phylum[}
                    \KeywordTok{match}\NormalTok{(species.nohuman, lineages}\OperatorTok{$}\NormalTok{tax\_id)}
    \NormalTok{            ]}
    \NormalTok{        ),}
        \DataTypeTok{Norank=}
            \KeywordTok{c}\NormalTok{(}
    \NormalTok{            lineages}\OperatorTok{$}\StringTok{`}\DataTypeTok{no rank}\StringTok{`}\NormalTok{[}
                    \KeywordTok{match}\NormalTok{(species.nohuman, lineages}\OperatorTok{$}\NormalTok{tax\_id)}
    \NormalTok{            ]}
    \NormalTok{        ),}
        \DataTypeTok{stringsAsFactors=}\OtherTok{FALSE}
    \NormalTok{)}
    \CommentTok{\# Add additional column "Name"}
    \NormalTok{phylogeny [ , }\KeywordTok{c}\NormalTok{(}\StringTok{"Name"}\NormalTok{)] <{-}}\StringTok{ }\OtherTok{NA}
    \CommentTok{\# Remove unhelpful term "cellular organisms"}
    \NormalTok{phylogeny}\OperatorTok{$}\NormalTok{Norank[phylogeny}\OperatorTok{$}\NormalTok{Norank }\OperatorTok{==}\StringTok{ "cellular organisms"}\NormalTok{] <{-}}\StringTok{ }\OtherTok{NA}
    \CommentTok{\# Use virus types as name if present}
    \NormalTok{phylogeny}\OperatorTok{$}\NormalTok{Name <{-}}\StringTok{ }\NormalTok{phylogeny}\OperatorTok{$}\NormalTok{Norank}
    \CommentTok{\# Use phylum as name if present}
    \NormalTok{phylogeny}\OperatorTok{$}\NormalTok{Name[}\KeywordTok{is.na}\NormalTok{(phylogeny}\OperatorTok{$}\NormalTok{Name)] <{-}}
    \StringTok{    }\KeywordTok{as.character}\NormalTok{(phylogeny}\OperatorTok{$}\NormalTok{Phylum[}\KeywordTok{is.na}\NormalTok{(phylogeny}\OperatorTok{$}\NormalTok{Name)])}
    \CommentTok{\# Use "<domain> (nonspecific)" as name if previous two not present}
    \NormalTok{phylogeny}\OperatorTok{$}\NormalTok{Name[}\KeywordTok{is.na}\NormalTok{(phylogeny}\OperatorTok{$}\NormalTok{Name)] <{-}}
    \StringTok{    }\KeywordTok{paste}\NormalTok{(}
            \KeywordTok{as.character}\NormalTok{(phylogeny}\OperatorTok{$}\NormalTok{Domain[}\KeywordTok{is.na}\NormalTok{(phylogeny}\OperatorTok{$}\NormalTok{Name)]),}
            \StringTok{"(nonspecific)"}\NormalTok{,}
            \DataTypeTok{sep=}\StringTok{" "}
    \NormalTok{    )}
    \CommentTok{\# Remove phylum and virus type columns}
    \NormalTok{phylogeny <{-}}
    \StringTok{    }\KeywordTok{data.frame}\NormalTok{(}
            \DataTypeTok{Domain=}\NormalTok{phylogeny}\OperatorTok{$}\NormalTok{Domain,}
            \DataTypeTok{Name=}\NormalTok{phylogeny}\OperatorTok{$}\NormalTok{Name,}
            \DataTypeTok{stringsAsFactors=}\OtherTok{FALSE}
    \NormalTok{    )}
    \CommentTok{\# Use table() to count occurrences}
    \NormalTok{phylogeny <{-}}\StringTok{ }\KeywordTok{as.data.frame}\NormalTok{(}\KeywordTok{table}\NormalTok{(phylogeny))}
    \CommentTok{\# Remove extraneous values}
    \NormalTok{phylogeny <{-}}\StringTok{ }\KeywordTok{subset}\NormalTok{(phylogeny, Freq }\OperatorTok{!=}\StringTok{ }\DecValTok{0}\NormalTok{)}
    \CommentTok{\# Sort}
    \NormalTok{phylogeny <{-}}\StringTok{ }\NormalTok{phylogeny[}\KeywordTok{order}\NormalTok{(phylogeny}\OperatorTok{$}\NormalTok{Domain, }\OperatorTok{{-}}\NormalTok{phylogeny}\OperatorTok{$}\NormalTok{Freq),]}
    
    \CommentTok{\# Create dictionary for looking up colors}
    \NormalTok{colors <{-}}
    \StringTok{    }\KeywordTok{data.frame}\NormalTok{(}
            \DataTypeTok{Colors=}\KeywordTok{c}\NormalTok{(}\StringTok{"Red"}\NormalTok{, }\StringTok{"Green"}\NormalTok{, }\StringTok{"Yellow"}\NormalTok{, }\StringTok{"Blue"}\NormalTok{),}
            \DataTypeTok{Domains=}\KeywordTok{c}\NormalTok{(}\StringTok{"Archaea"}\NormalTok{, }\StringTok{"Bacteria"}\NormalTok{, }\StringTok{"Eukaryota"}\NormalTok{, }\StringTok{"Viruses"}\NormalTok{),}
            \DataTypeTok{stringsAsFactors=}\OtherTok{FALSE}
    \NormalTok{    )}
    
    \CommentTok{\# Create list of colors matching domains}
    \NormalTok{cols <{-}}
    \StringTok{    }\NormalTok{colors}\OperatorTok{$}\NormalTok{Colors[}
            \KeywordTok{match}\NormalTok{(phylogeny}\OperatorTok{$}\NormalTok{Domain, colors}\OperatorTok{$}\NormalTok{Domains)}
    \NormalTok{    ]}
    
    \CommentTok{\# Set margins}
    \KeywordTok{par}\NormalTok{(}\DataTypeTok{mar=}\KeywordTok{c}\NormalTok{(}\DecValTok{7}\NormalTok{,}\DecValTok{4}\NormalTok{,}\DecValTok{4}\NormalTok{,}\DecValTok{1}\NormalTok{)}\OperatorTok{+}\FloatTok{0.1}\NormalTok{)}
    \CommentTok{\# Print phylum plot}
    \NormalTok{plot <{-}}\StringTok{ }\KeywordTok{barplot}\NormalTok{(}
        \DataTypeTok{height =}\NormalTok{ phylogeny}\OperatorTok{$}\NormalTok{Freq,}
        \DataTypeTok{ylab =} \StringTok{"log References"}\NormalTok{,}
        \DataTypeTok{main =} \StringTok{"References to specific phyla in PubMed search for }\CharTok{\textbackslash{}"}\StringTok{Human Microbiome}\CharTok{\textbackslash{}"}\StringTok{"}\NormalTok{,}
        \DataTypeTok{col =}\NormalTok{ cols,}
        \DataTypeTok{names.arg =}\NormalTok{ phylogeny}\OperatorTok{$}\NormalTok{Name,}
        \DataTypeTok{log =} \StringTok{"y"}\NormalTok{,}
        \DataTypeTok{xaxt =} \StringTok{"n"}
    \NormalTok{)}
    \CommentTok{\# Add X{-}axis labels}
    \KeywordTok{text}\NormalTok{(}
    \NormalTok{    plot,}
        \FloatTok{0.95}\NormalTok{,}
        \DataTypeTok{labels =}\NormalTok{ phylogeny}\OperatorTok{$}\NormalTok{Name,}
        \DataTypeTok{srt =} \DecValTok{45}\NormalTok{,}
        \DataTypeTok{adj =} \KeywordTok{c}\NormalTok{(}\FloatTok{1.1}\NormalTok{,}\FloatTok{1.1}\NormalTok{),}
        \DataTypeTok{xpd =} \OtherTok{TRUE}\NormalTok{,}
        \DataTypeTok{cex =} \FloatTok{0.8}
    \NormalTok{)}
    \CommentTok{\# Add legend for domain colors}
    \KeywordTok{legend}\NormalTok{(}
        \StringTok{"topright"}\NormalTok{,}
    \NormalTok{    plot,}
        \DataTypeTok{legend =}\NormalTok{ colors}\OperatorTok{$}\NormalTok{Domains,}
        \DataTypeTok{fill =}\NormalTok{ colors}\OperatorTok{$}\NormalTok{Colors}
    \NormalTok{)}
    \end{Highlighting}
    \end{Shaded}

\end{document}
    \caption{References to phyla in a PubMed search for the phrase
             ``human microbiome.'' References less specific than phylum are
             recorded as ``\textit{domain} (nonspecific),'' domains are
             presented as colors. All references to humans were excluded, and
             would have been sorted into Chordata.}
    \label{fig:1}
\end{figure}

Of the 1,779 articles indexed, 1,730 mentioned a taxonomic rank or species.
This totaled 5,354 individual mentions, of which 2,419 were to species or
ranks other than humans. As visible in \hyperref[fig:1]{Figure 1}, bacteria
made up the majority of references, with the phyla Proteobacteria and
Firmicutes second only to nonspecific references to bacteria as a domain.
Eukarya were second to bacteria in total number of references; however, the
most frequently discussed phylum within the domain was Chordata, which likely
stems from discussion of vertebrate microbiomes such as Mus musculus rather
than indicating the presence of chordates in the human microbiome. Viruses
were referenced moderately frequently, with nonspecific virus references
making up the greatest proportion of the domain's coverage, followed by
bacteriophages. Finally, archaea were referenced least frequently of all,
with nine total references amounting to less than any viral phylum.

\section{The Role of Non-Bacterial Organisms in the Microbiome}

\subsection{Eukarya}

Yeasts have been found to colonize infant guts gradually, with an isolation
rate of 13\% at 28-46 days old \parencite{Ben84} and 50\% as
soon as four months of age \parencite{Ell75}.
Notably, this rate of colonization was consistent between breast-fed and
formula-fed infants, in contrast to the colonization rates of bacterial phyla
which differ significantly between the two feeding mechanisms. From a
life-history perspective, this presents a novel question of how yeast
colonize the infant gut and whether the mechanism by which breastmilk
contains microbiota associated with the mother's gut microbiome
\parencite{Mar13} is also capable of providing yeast, whose eukaryotic cells
differ in size by orders of magnitude from prokaryotic cells.

However, research conducted on yeast inhabitants of animals focuses primarily
on their pathological role. The genus \textit{Candida} is well known for its
role in infectious disease in humans, particularly in the immunocompromised
\parencite{Kou11}. However, species of yeast, including many
\textit{Candida}~spp., have frequently been identified on the human body,
including the vagina, gut, and skin, without any associated mycosis
\parencite{Man10}. Even for species known to be infectious agents, between
12\% \parencite{Cho86} and 80\% \parencite{Soe07} of asymptomatic healthy
women are vaginal carriers. In place of viewing yeast through the lens of
infection, it should be considered as a part of the ecosystem that makes up
the human body. How it reaches and colonizes infants, and why its prevalence
makes up far less of the population than do bacterial commensals, merits
further study.

\subsection{Archaea}

In 1966 it was discovered that humans both respire and flatulate methane,
which is not known to be biologically produced by any bacteria or eukaryote.
Following this observation, methanogenic prokaryotes, from what would later
be called the archaea, were identified in human feces \parencite{Not68}. The
archaeal domain remains the only taxon with members known to be biological
producers of methane \parencite{Tha06}, and as such is unique in its
residence within the human gut. Despite widespread colonization by a diverse
array of archaeal species, the ecological niche which they occupy within the
human body and whether they are transferred vertically or environmentally
remains unknown \parencite{Dri11}. It has been theorized that many more
lineages of archaea remain undiscovered, as the DNA analysis methods used to
identify bacterial genomes are incompatible with archaeal cell walls
\parencite{Hor15}.

\subsection{Viruses}

Despite not fitting many definitions of biological life, viruses are the most
diverse form of biological material on the planet. They are known to
outnumber bacteria by a factor of nearly ten \parencite{Ore97} and are a
major driver of evolution in prokaryotes. Although humans are susceptible to
a wide variety of viruses, the viral community in the microbiome, so fast it
has itself been called a ``virome'' \parencite{Wyl12}, is dominated primarily
by bacteriophages whose sequences are novel \parencite{Dut14}. The phages
identifiable by genetic sequence are primarily those known to attack
bacterial phyla associated with the microbiome, yet appear to be vastly more
diverse than the virome of the ocean \parencite{Wal14}. Unlike the theories
of bacterial colonization, how such a vast and dynamic virome is able to
establish itself by adulthood remains unknown. The effect of age, and whether
milk carries a virome of similar diversity would be intriguing avenues for
further study.

\section{Limitations}

Utilizing both life-history and ecological frameworks for approaching
non-bacterial components of the microbiome, it is clear that the roles such
taxa play within our bodies is underexplored. While this paper sought to
touch on these issues, it is limited by both methodology and its brevity. The
use of NCBI PubMed for language analysis suffers from a bias toward medical
research, and so microbiome research will often consider the roles of known
pathological agents rather than in-depth discussions of all possible
inhabitants. Eukarya is also a domain of organisms highly diverse in physical
makeup and metabolic strategy, while only yeasts were discussed here. The
roles of other unicellular parasites (for instance,
\textit{Toxoplasma gondii}) and of helminthic worms, which each form diverse
monophyletic groups, can also be explored via the lens of the microbiome's
life-history. More research is needed in this area to investigate how the
diverse community of microbes within humans evolved and what role it plays
even in the absence of pathology.

\printbibliography

\setcounter{secnumdepth}{0}

\newpage
\section{Appendix A}
\label{sec:appendix_a}

\begin{Shaded}
    \begin{Highlighting}[]
    \CommentTok{\# R script to analyze PubMed abstracts}
    \CommentTok{\# Author: Dan Leonard}
    
    \CommentTok{\# Import pubmed analysis library}
    \CommentTok{\# \parencite{Ran19}}
    \KeywordTok{library}\NormalTok{(pubmed.mineR)}
    \CommentTok{\# Import file{-}reading library}
    \CommentTok{\# \parencite{Wic18}}
    \KeywordTok{library}\NormalTok{(readr)}
    
    \CommentTok{\# Load XML search result file from PubMed into list}
    \NormalTok{abstracts <{-}}\StringTok{ }\KeywordTok{xmlreadabs}\NormalTok{(}\StringTok{"pubmed\_result.xml"}\NormalTok{)}
    
    \CommentTok{\# Load NCBI lineage data into list}
    \CommentTok{\# \parencite{Xue19}}
    \NormalTok{lineages <{-}}\StringTok{ }\KeywordTok{read\_csv}\NormalTok{(}\StringTok{"lineages{-}2017{-}03{-}17.csv"}\NormalTok{,}
        \DataTypeTok{col\_types =} \KeywordTok{cols}\NormalTok{(}
            \DataTypeTok{.default =} \KeywordTok{col\_character}\NormalTok{(),}
            \DataTypeTok{tax\_id =} \KeywordTok{col\_integer}\NormalTok{()}
    \NormalTok{    )}
    \NormalTok{)}
    
    \CommentTok{\# Run PMIDs through PubTator}
    \NormalTok{pubtators <{-}}\StringTok{ }\KeywordTok{lapply}\NormalTok{(abstracts}\OperatorTok{@}\NormalTok{PMID, pubtator\_function)}
    \CommentTok{\# Remove [" No Data "] from PubTator results}
    \NormalTok{pubtators <{-}}\StringTok{ }\NormalTok{pubtators [}\OperatorTok{!}\StringTok{ }\NormalTok{pubtators }\OperatorTok{\%in\%}\StringTok{ }\KeywordTok{list}\NormalTok{(}\StringTok{" No Data "}\NormalTok{)]}
    
    \CommentTok{\# Get list of vectors of species names}
    \CommentTok{\# Species names are located in column 5}
    \NormalTok{species <{-}}\StringTok{ }\KeywordTok{sapply}\NormalTok{(pubtators, }\StringTok{"[["}\NormalTok{, }\DecValTok{5}\NormalTok{)}
    \CommentTok{\# Remove nulls}
    \NormalTok{species [}\KeywordTok{sapply}\NormalTok{(species, is.null)] <{-}}\StringTok{ }\OtherTok{NULL}
    \CommentTok{\# Flatten}
    \NormalTok{species <{-}}\StringTok{ }\KeywordTok{unlist}\NormalTok{(species)}
    
    \CommentTok{\# Remove species names, leave NCBI numerical ID}
    \NormalTok{species <{-}}\StringTok{ }\KeywordTok{sapply}\NormalTok{(species, }\ControlFlowTok{function}\NormalTok{(x) }\KeywordTok{sapply}\NormalTok{(}\KeywordTok{strsplit}\NormalTok{(x, }\StringTok{">"}\NormalTok{), }\StringTok{"[["}\NormalTok{, }\DecValTok{2}\NormalTok{))}
    \CommentTok{\# Convert to numeric form}
    \NormalTok{species <{-}}\StringTok{ }\KeywordTok{as.numeric}\NormalTok{(species)}
    \CommentTok{\# Remove references to Homo sapiens (NCBI ID 9606)}
    \NormalTok{species.nohuman <{-}}\StringTok{ }\NormalTok{species [}\OperatorTok{!}\StringTok{ }\NormalTok{species }\OperatorTok{\%in\%}\StringTok{ }\DecValTok{9606}\NormalTok{]}
    
    \CommentTok{\# Get domain names}
    \NormalTok{species.nohuman.domains <{-}}
    \StringTok{    }\NormalTok{lineages}\OperatorTok{$}\NormalTok{superkingdom[}\KeywordTok{match}\NormalTok{(species.nohuman, lineages}\OperatorTok{$}\NormalTok{tax\_id)]}
    \CommentTok{\# Get phyla names}
    \NormalTok{species.nohuman.phyla <{-}}
    \StringTok{    }\NormalTok{lineages}\OperatorTok{$}\NormalTok{phylum[}\KeywordTok{match}\NormalTok{(species.nohuman, lineages}\OperatorTok{$}\NormalTok{tax\_id)]}
    
    \CommentTok{\# Create data frame}
    \NormalTok{phylogeny <{-}}\StringTok{ }\KeywordTok{data.frame}\NormalTok{(}
        \DataTypeTok{Domain=}
            \KeywordTok{c}\NormalTok{(}
    \NormalTok{            lineages}\OperatorTok{$}\NormalTok{superkingdom[}
                    \KeywordTok{match}\NormalTok{(species.nohuman, lineages}\OperatorTok{$}\NormalTok{tax\_id)}
    \NormalTok{            ]}
    \NormalTok{        ),}
        \DataTypeTok{Phylum=}
            \KeywordTok{c}\NormalTok{(}
    \NormalTok{            lineages}\OperatorTok{$}\NormalTok{phylum[}
                    \KeywordTok{match}\NormalTok{(species.nohuman, lineages}\OperatorTok{$}\NormalTok{tax\_id)}
    \NormalTok{            ]}
    \NormalTok{        ),}
        \DataTypeTok{Norank=}
            \KeywordTok{c}\NormalTok{(}
    \NormalTok{            lineages}\OperatorTok{$}\StringTok{`}\DataTypeTok{no rank}\StringTok{`}\NormalTok{[}
                    \KeywordTok{match}\NormalTok{(species.nohuman, lineages}\OperatorTok{$}\NormalTok{tax\_id)}
    \NormalTok{            ]}
    \NormalTok{        ),}
        \DataTypeTok{stringsAsFactors=}\OtherTok{FALSE}
    \NormalTok{)}
    \CommentTok{\# Add additional column "Name"}
    \NormalTok{phylogeny [ , }\KeywordTok{c}\NormalTok{(}\StringTok{"Name"}\NormalTok{)] <{-}}\StringTok{ }\OtherTok{NA}
    \CommentTok{\# Remove unhelpful term "cellular organisms"}
    \NormalTok{phylogeny}\OperatorTok{$}\NormalTok{Norank[phylogeny}\OperatorTok{$}\NormalTok{Norank }\OperatorTok{==}\StringTok{ "cellular organisms"}\NormalTok{] <{-}}\StringTok{ }\OtherTok{NA}
    \CommentTok{\# Use virus types as name if present}
    \NormalTok{phylogeny}\OperatorTok{$}\NormalTok{Name <{-}}\StringTok{ }\NormalTok{phylogeny}\OperatorTok{$}\NormalTok{Norank}
    \CommentTok{\# Use phylum as name if present}
    \NormalTok{phylogeny}\OperatorTok{$}\NormalTok{Name[}\KeywordTok{is.na}\NormalTok{(phylogeny}\OperatorTok{$}\NormalTok{Name)] <{-}}
    \StringTok{    }\KeywordTok{as.character}\NormalTok{(phylogeny}\OperatorTok{$}\NormalTok{Phylum[}\KeywordTok{is.na}\NormalTok{(phylogeny}\OperatorTok{$}\NormalTok{Name)])}
    \CommentTok{\# Use "<domain> (nonspecific)" as name if previous two not present}
    \NormalTok{phylogeny}\OperatorTok{$}\NormalTok{Name[}\KeywordTok{is.na}\NormalTok{(phylogeny}\OperatorTok{$}\NormalTok{Name)] <{-}}
    \StringTok{    }\KeywordTok{paste}\NormalTok{(}
            \KeywordTok{as.character}\NormalTok{(phylogeny}\OperatorTok{$}\NormalTok{Domain[}\KeywordTok{is.na}\NormalTok{(phylogeny}\OperatorTok{$}\NormalTok{Name)]),}
            \StringTok{"(nonspecific)"}\NormalTok{,}
            \DataTypeTok{sep=}\StringTok{" "}
    \NormalTok{    )}
    \CommentTok{\# Remove phylum and virus type columns}
    \NormalTok{phylogeny <{-}}
    \StringTok{    }\KeywordTok{data.frame}\NormalTok{(}
            \DataTypeTok{Domain=}\NormalTok{phylogeny}\OperatorTok{$}\NormalTok{Domain,}
            \DataTypeTok{Name=}\NormalTok{phylogeny}\OperatorTok{$}\NormalTok{Name,}
            \DataTypeTok{stringsAsFactors=}\OtherTok{FALSE}
    \NormalTok{    )}
    \CommentTok{\# Use table() to count occurrences}
    \NormalTok{phylogeny <{-}}\StringTok{ }\KeywordTok{as.data.frame}\NormalTok{(}\KeywordTok{table}\NormalTok{(phylogeny))}
    \CommentTok{\# Remove extraneous values}
    \NormalTok{phylogeny <{-}}\StringTok{ }\KeywordTok{subset}\NormalTok{(phylogeny, Freq }\OperatorTok{!=}\StringTok{ }\DecValTok{0}\NormalTok{)}
    \CommentTok{\# Sort}
    \NormalTok{phylogeny <{-}}\StringTok{ }\NormalTok{phylogeny[}\KeywordTok{order}\NormalTok{(phylogeny}\OperatorTok{$}\NormalTok{Domain, }\OperatorTok{{-}}\NormalTok{phylogeny}\OperatorTok{$}\NormalTok{Freq),]}
    
    \CommentTok{\# Create dictionary for looking up colors}
    \NormalTok{colors <{-}}
    \StringTok{    }\KeywordTok{data.frame}\NormalTok{(}
            \DataTypeTok{Colors=}\KeywordTok{c}\NormalTok{(}\StringTok{"Red"}\NormalTok{, }\StringTok{"Green"}\NormalTok{, }\StringTok{"Yellow"}\NormalTok{, }\StringTok{"Blue"}\NormalTok{),}
            \DataTypeTok{Domains=}\KeywordTok{c}\NormalTok{(}\StringTok{"Archaea"}\NormalTok{, }\StringTok{"Bacteria"}\NormalTok{, }\StringTok{"Eukaryota"}\NormalTok{, }\StringTok{"Viruses"}\NormalTok{),}
            \DataTypeTok{stringsAsFactors=}\OtherTok{FALSE}
    \NormalTok{    )}
    
    \CommentTok{\# Create list of colors matching domains}
    \NormalTok{cols <{-}}
    \StringTok{    }\NormalTok{colors}\OperatorTok{$}\NormalTok{Colors[}
            \KeywordTok{match}\NormalTok{(phylogeny}\OperatorTok{$}\NormalTok{Domain, colors}\OperatorTok{$}\NormalTok{Domains)}
    \NormalTok{    ]}
    
    \CommentTok{\# Set margins}
    \KeywordTok{par}\NormalTok{(}\DataTypeTok{mar=}\KeywordTok{c}\NormalTok{(}\DecValTok{7}\NormalTok{,}\DecValTok{4}\NormalTok{,}\DecValTok{4}\NormalTok{,}\DecValTok{1}\NormalTok{)}\OperatorTok{+}\FloatTok{0.1}\NormalTok{)}
    \CommentTok{\# Print phylum plot}
    \NormalTok{plot <{-}}\StringTok{ }\KeywordTok{barplot}\NormalTok{(}
        \DataTypeTok{height =}\NormalTok{ phylogeny}\OperatorTok{$}\NormalTok{Freq,}
        \DataTypeTok{ylab =} \StringTok{"log References"}\NormalTok{,}
        \DataTypeTok{main =} \StringTok{"References to specific phyla in PubMed search for }\CharTok{\textbackslash{}"}\StringTok{Human Microbiome}\CharTok{\textbackslash{}"}\StringTok{"}\NormalTok{,}
        \DataTypeTok{col =}\NormalTok{ cols,}
        \DataTypeTok{names.arg =}\NormalTok{ phylogeny}\OperatorTok{$}\NormalTok{Name,}
        \DataTypeTok{log =} \StringTok{"y"}\NormalTok{,}
        \DataTypeTok{xaxt =} \StringTok{"n"}
    \NormalTok{)}
    \CommentTok{\# Add X{-}axis labels}
    \KeywordTok{text}\NormalTok{(}
    \NormalTok{    plot,}
        \FloatTok{0.95}\NormalTok{,}
        \DataTypeTok{labels =}\NormalTok{ phylogeny}\OperatorTok{$}\NormalTok{Name,}
        \DataTypeTok{srt =} \DecValTok{45}\NormalTok{,}
        \DataTypeTok{adj =} \KeywordTok{c}\NormalTok{(}\FloatTok{1.1}\NormalTok{,}\FloatTok{1.1}\NormalTok{),}
        \DataTypeTok{xpd =} \OtherTok{TRUE}\NormalTok{,}
        \DataTypeTok{cex =} \FloatTok{0.8}
    \NormalTok{)}
    \CommentTok{\# Add legend for domain colors}
    \KeywordTok{legend}\NormalTok{(}
        \StringTok{"topright"}\NormalTok{,}
    \NormalTok{    plot,}
        \DataTypeTok{legend =}\NormalTok{ colors}\OperatorTok{$}\NormalTok{Domains,}
        \DataTypeTok{fill =}\NormalTok{ colors}\OperatorTok{$}\NormalTok{Colors}
    \NormalTok{)}
    \end{Highlighting}
    \end{Shaded}

\end{document}
    \caption{References to phyla in a PubMed search for the phrase
             ``human microbiome.'' References less specific than phylum are
             recorded as ``\textit{domain} (nonspecific),'' domains are
             presented as colors. All references to humans were excluded, and
             would have been sorted into Chordata.}
    \label{fig:1}
\end{figure}

Of the 1,779 articles indexed, 1,730 mentioned a taxonomic rank or species.
This totaled 5,354 individual mentions, of which 2,419 were to species or
ranks other than humans. As visible in \hyperref[fig:1]{Figure 1}, bacteria
made up the majority of references, with the phyla Proteobacteria and
Firmicutes second only to nonspecific references to bacteria as a domain.
Eukarya were second to bacteria in total number of references; however, the
most frequently discussed phylum within the domain was Chordata, which likely
stems from discussion of vertebrate microbiomes such as Mus musculus rather
than indicating the presence of chordates in the human microbiome. Viruses
were referenced moderately frequently, with nonspecific virus references
making up the greatest proportion of the domain's coverage, followed by
bacteriophages. Finally, archaea were referenced least frequently of all,
with nine total references amounting to less than any viral phylum.

\section{The Role of Non-Bacterial Organisms in the Microbiome}

\subsection{Eukarya}

Yeasts have been found to colonize infant guts gradually, with an isolation
rate of 13\% at 28-46 days old \parencite{Ben84} and 50\% as
soon as four months of age \parencite{Ell75}.
Notably, this rate of colonization was consistent between breast-fed and
formula-fed infants, in contrast to the colonization rates of bacterial phyla
which differ significantly between the two feeding mechanisms. From a
life-history perspective, this presents a novel question of how yeast
colonize the infant gut and whether the mechanism by which breastmilk
contains microbiota associated with the mother's gut microbiome
\parencite{Mar13} is also capable of providing yeast, whose eukaryotic cells
differ in size by orders of magnitude from prokaryotic cells.

However, research conducted on yeast inhabitants of animals focuses primarily
on their pathological role. The genus \textit{Candida} is well known for its
role in infectious disease in humans, particularly in the immunocompromised
\parencite{Kou11}. However, species of yeast, including many
\textit{Candida}~spp., have frequently been identified on the human body,
including the vagina, gut, and skin, without any associated mycosis
\parencite{Man10}. Even for species known to be infectious agents, between
12\% \parencite{Cho86} and 80\% \parencite{Soe07} of asymptomatic healthy
women are vaginal carriers. In place of viewing yeast through the lens of
infection, it should be considered as a part of the ecosystem that makes up
the human body. How it reaches and colonizes infants, and why its prevalence
makes up far less of the population than do bacterial commensals, merits
further study.

\subsection{Archaea}

In 1966 it was discovered that humans both respire and flatulate methane,
which is not known to be biologically produced by any bacteria or eukaryote.
Following this observation, methanogenic prokaryotes, from what would later
be called the archaea, were identified in human feces \parencite{Not68}. The
archaeal domain remains the only taxon with members known to be biological
producers of methane \parencite{Tha06}, and as such is unique in its
residence within the human gut. Despite widespread colonization by a diverse
array of archaeal species, the ecological niche which they occupy within the
human body and whether they are transferred vertically or environmentally
remains unknown \parencite{Dri11}. It has been theorized that many more
lineages of archaea remain undiscovered, as the DNA analysis methods used to
identify bacterial genomes are incompatible with archaeal cell walls
\parencite{Hor15}.

\subsection{Viruses}

Despite not fitting many definitions of biological life, viruses are the most
diverse form of biological material on the planet. They are known to
outnumber bacteria by a factor of nearly ten \parencite{Ore97} and are a
major driver of evolution in prokaryotes. Although humans are susceptible to
a wide variety of viruses, the viral community in the microbiome, so fast it
has itself been called a ``virome'' \parencite{Wyl12}, is dominated primarily
by bacteriophages whose sequences are novel \parencite{Dut14}. The phages
identifiable by genetic sequence are primarily those known to attack
bacterial phyla associated with the microbiome, yet appear to be vastly more
diverse than the virome of the ocean \parencite{Wal14}. Unlike the theories
of bacterial colonization, how such a vast and dynamic virome is able to
establish itself by adulthood remains unknown. The effect of age, and whether
milk carries a virome of similar diversity would be intriguing avenues for
further study.

\section{Limitations}

Utilizing both life-history and ecological frameworks for approaching
non-bacterial components of the microbiome, it is clear that the roles such
taxa play within our bodies is underexplored. While this paper sought to
touch on these issues, it is limited by both methodology and its brevity. The
use of NCBI PubMed for language analysis suffers from a bias toward medical
research, and so microbiome research will often consider the roles of known
pathological agents rather than in-depth discussions of all possible
inhabitants. Eukarya is also a domain of organisms highly diverse in physical
makeup and metabolic strategy, while only yeasts were discussed here. The
roles of other unicellular parasites (for instance,
\textit{Toxoplasma gondii}) and of helminthic worms, which each form diverse
monophyletic groups, can also be explored via the lens of the microbiome's
life-history. More research is needed in this area to investigate how the
diverse community of microbes within humans evolved and what role it plays
even in the absence of pathology.

\printbibliography

\setcounter{secnumdepth}{0}

\newpage
\section{Appendix A}
\label{sec:appendix_a}

\begin{Shaded}
    \begin{Highlighting}[]
    \CommentTok{\# R script to analyze PubMed abstracts}
    \CommentTok{\# Author: Dan Leonard}
    
    \CommentTok{\# Import pubmed analysis library}
    \CommentTok{\# \parencite{Ran19}}
    \KeywordTok{library}\NormalTok{(pubmed.mineR)}
    \CommentTok{\# Import file{-}reading library}
    \CommentTok{\# \parencite{Wic18}}
    \KeywordTok{library}\NormalTok{(readr)}
    
    \CommentTok{\# Load XML search result file from PubMed into list}
    \NormalTok{abstracts <{-}}\StringTok{ }\KeywordTok{xmlreadabs}\NormalTok{(}\StringTok{"pubmed\_result.xml"}\NormalTok{)}
    
    \CommentTok{\# Load NCBI lineage data into list}
    \CommentTok{\# \parencite{Xue19}}
    \NormalTok{lineages <{-}}\StringTok{ }\KeywordTok{read\_csv}\NormalTok{(}\StringTok{"lineages{-}2017{-}03{-}17.csv"}\NormalTok{,}
        \DataTypeTok{col\_types =} \KeywordTok{cols}\NormalTok{(}
            \DataTypeTok{.default =} \KeywordTok{col\_character}\NormalTok{(),}
            \DataTypeTok{tax\_id =} \KeywordTok{col\_integer}\NormalTok{()}
    \NormalTok{    )}
    \NormalTok{)}
    
    \CommentTok{\# Run PMIDs through PubTator}
    \NormalTok{pubtators <{-}}\StringTok{ }\KeywordTok{lapply}\NormalTok{(abstracts}\OperatorTok{@}\NormalTok{PMID, pubtator\_function)}
    \CommentTok{\# Remove [" No Data "] from PubTator results}
    \NormalTok{pubtators <{-}}\StringTok{ }\NormalTok{pubtators [}\OperatorTok{!}\StringTok{ }\NormalTok{pubtators }\OperatorTok{\%in\%}\StringTok{ }\KeywordTok{list}\NormalTok{(}\StringTok{" No Data "}\NormalTok{)]}
    
    \CommentTok{\# Get list of vectors of species names}
    \CommentTok{\# Species names are located in column 5}
    \NormalTok{species <{-}}\StringTok{ }\KeywordTok{sapply}\NormalTok{(pubtators, }\StringTok{"[["}\NormalTok{, }\DecValTok{5}\NormalTok{)}
    \CommentTok{\# Remove nulls}
    \NormalTok{species [}\KeywordTok{sapply}\NormalTok{(species, is.null)] <{-}}\StringTok{ }\OtherTok{NULL}
    \CommentTok{\# Flatten}
    \NormalTok{species <{-}}\StringTok{ }\KeywordTok{unlist}\NormalTok{(species)}
    
    \CommentTok{\# Remove species names, leave NCBI numerical ID}
    \NormalTok{species <{-}}\StringTok{ }\KeywordTok{sapply}\NormalTok{(species, }\ControlFlowTok{function}\NormalTok{(x) }\KeywordTok{sapply}\NormalTok{(}\KeywordTok{strsplit}\NormalTok{(x, }\StringTok{">"}\NormalTok{), }\StringTok{"[["}\NormalTok{, }\DecValTok{2}\NormalTok{))}
    \CommentTok{\# Convert to numeric form}
    \NormalTok{species <{-}}\StringTok{ }\KeywordTok{as.numeric}\NormalTok{(species)}
    \CommentTok{\# Remove references to Homo sapiens (NCBI ID 9606)}
    \NormalTok{species.nohuman <{-}}\StringTok{ }\NormalTok{species [}\OperatorTok{!}\StringTok{ }\NormalTok{species }\OperatorTok{\%in\%}\StringTok{ }\DecValTok{9606}\NormalTok{]}
    
    \CommentTok{\# Get domain names}
    \NormalTok{species.nohuman.domains <{-}}
    \StringTok{    }\NormalTok{lineages}\OperatorTok{$}\NormalTok{superkingdom[}\KeywordTok{match}\NormalTok{(species.nohuman, lineages}\OperatorTok{$}\NormalTok{tax\_id)]}
    \CommentTok{\# Get phyla names}
    \NormalTok{species.nohuman.phyla <{-}}
    \StringTok{    }\NormalTok{lineages}\OperatorTok{$}\NormalTok{phylum[}\KeywordTok{match}\NormalTok{(species.nohuman, lineages}\OperatorTok{$}\NormalTok{tax\_id)]}
    
    \CommentTok{\# Create data frame}
    \NormalTok{phylogeny <{-}}\StringTok{ }\KeywordTok{data.frame}\NormalTok{(}
        \DataTypeTok{Domain=}
            \KeywordTok{c}\NormalTok{(}
    \NormalTok{            lineages}\OperatorTok{$}\NormalTok{superkingdom[}
                    \KeywordTok{match}\NormalTok{(species.nohuman, lineages}\OperatorTok{$}\NormalTok{tax\_id)}
    \NormalTok{            ]}
    \NormalTok{        ),}
        \DataTypeTok{Phylum=}
            \KeywordTok{c}\NormalTok{(}
    \NormalTok{            lineages}\OperatorTok{$}\NormalTok{phylum[}
                    \KeywordTok{match}\NormalTok{(species.nohuman, lineages}\OperatorTok{$}\NormalTok{tax\_id)}
    \NormalTok{            ]}
    \NormalTok{        ),}
        \DataTypeTok{Norank=}
            \KeywordTok{c}\NormalTok{(}
    \NormalTok{            lineages}\OperatorTok{$}\StringTok{`}\DataTypeTok{no rank}\StringTok{`}\NormalTok{[}
                    \KeywordTok{match}\NormalTok{(species.nohuman, lineages}\OperatorTok{$}\NormalTok{tax\_id)}
    \NormalTok{            ]}
    \NormalTok{        ),}
        \DataTypeTok{stringsAsFactors=}\OtherTok{FALSE}
    \NormalTok{)}
    \CommentTok{\# Add additional column "Name"}
    \NormalTok{phylogeny [ , }\KeywordTok{c}\NormalTok{(}\StringTok{"Name"}\NormalTok{)] <{-}}\StringTok{ }\OtherTok{NA}
    \CommentTok{\# Remove unhelpful term "cellular organisms"}
    \NormalTok{phylogeny}\OperatorTok{$}\NormalTok{Norank[phylogeny}\OperatorTok{$}\NormalTok{Norank }\OperatorTok{==}\StringTok{ "cellular organisms"}\NormalTok{] <{-}}\StringTok{ }\OtherTok{NA}
    \CommentTok{\# Use virus types as name if present}
    \NormalTok{phylogeny}\OperatorTok{$}\NormalTok{Name <{-}}\StringTok{ }\NormalTok{phylogeny}\OperatorTok{$}\NormalTok{Norank}
    \CommentTok{\# Use phylum as name if present}
    \NormalTok{phylogeny}\OperatorTok{$}\NormalTok{Name[}\KeywordTok{is.na}\NormalTok{(phylogeny}\OperatorTok{$}\NormalTok{Name)] <{-}}
    \StringTok{    }\KeywordTok{as.character}\NormalTok{(phylogeny}\OperatorTok{$}\NormalTok{Phylum[}\KeywordTok{is.na}\NormalTok{(phylogeny}\OperatorTok{$}\NormalTok{Name)])}
    \CommentTok{\# Use "<domain> (nonspecific)" as name if previous two not present}
    \NormalTok{phylogeny}\OperatorTok{$}\NormalTok{Name[}\KeywordTok{is.na}\NormalTok{(phylogeny}\OperatorTok{$}\NormalTok{Name)] <{-}}
    \StringTok{    }\KeywordTok{paste}\NormalTok{(}
            \KeywordTok{as.character}\NormalTok{(phylogeny}\OperatorTok{$}\NormalTok{Domain[}\KeywordTok{is.na}\NormalTok{(phylogeny}\OperatorTok{$}\NormalTok{Name)]),}
            \StringTok{"(nonspecific)"}\NormalTok{,}
            \DataTypeTok{sep=}\StringTok{" "}
    \NormalTok{    )}
    \CommentTok{\# Remove phylum and virus type columns}
    \NormalTok{phylogeny <{-}}
    \StringTok{    }\KeywordTok{data.frame}\NormalTok{(}
            \DataTypeTok{Domain=}\NormalTok{phylogeny}\OperatorTok{$}\NormalTok{Domain,}
            \DataTypeTok{Name=}\NormalTok{phylogeny}\OperatorTok{$}\NormalTok{Name,}
            \DataTypeTok{stringsAsFactors=}\OtherTok{FALSE}
    \NormalTok{    )}
    \CommentTok{\# Use table() to count occurrences}
    \NormalTok{phylogeny <{-}}\StringTok{ }\KeywordTok{as.data.frame}\NormalTok{(}\KeywordTok{table}\NormalTok{(phylogeny))}
    \CommentTok{\# Remove extraneous values}
    \NormalTok{phylogeny <{-}}\StringTok{ }\KeywordTok{subset}\NormalTok{(phylogeny, Freq }\OperatorTok{!=}\StringTok{ }\DecValTok{0}\NormalTok{)}
    \CommentTok{\# Sort}
    \NormalTok{phylogeny <{-}}\StringTok{ }\NormalTok{phylogeny[}\KeywordTok{order}\NormalTok{(phylogeny}\OperatorTok{$}\NormalTok{Domain, }\OperatorTok{{-}}\NormalTok{phylogeny}\OperatorTok{$}\NormalTok{Freq),]}
    
    \CommentTok{\# Create dictionary for looking up colors}
    \NormalTok{colors <{-}}
    \StringTok{    }\KeywordTok{data.frame}\NormalTok{(}
            \DataTypeTok{Colors=}\KeywordTok{c}\NormalTok{(}\StringTok{"Red"}\NormalTok{, }\StringTok{"Green"}\NormalTok{, }\StringTok{"Yellow"}\NormalTok{, }\StringTok{"Blue"}\NormalTok{),}
            \DataTypeTok{Domains=}\KeywordTok{c}\NormalTok{(}\StringTok{"Archaea"}\NormalTok{, }\StringTok{"Bacteria"}\NormalTok{, }\StringTok{"Eukaryota"}\NormalTok{, }\StringTok{"Viruses"}\NormalTok{),}
            \DataTypeTok{stringsAsFactors=}\OtherTok{FALSE}
    \NormalTok{    )}
    
    \CommentTok{\# Create list of colors matching domains}
    \NormalTok{cols <{-}}
    \StringTok{    }\NormalTok{colors}\OperatorTok{$}\NormalTok{Colors[}
            \KeywordTok{match}\NormalTok{(phylogeny}\OperatorTok{$}\NormalTok{Domain, colors}\OperatorTok{$}\NormalTok{Domains)}
    \NormalTok{    ]}
    
    \CommentTok{\# Set margins}
    \KeywordTok{par}\NormalTok{(}\DataTypeTok{mar=}\KeywordTok{c}\NormalTok{(}\DecValTok{7}\NormalTok{,}\DecValTok{4}\NormalTok{,}\DecValTok{4}\NormalTok{,}\DecValTok{1}\NormalTok{)}\OperatorTok{+}\FloatTok{0.1}\NormalTok{)}
    \CommentTok{\# Print phylum plot}
    \NormalTok{plot <{-}}\StringTok{ }\KeywordTok{barplot}\NormalTok{(}
        \DataTypeTok{height =}\NormalTok{ phylogeny}\OperatorTok{$}\NormalTok{Freq,}
        \DataTypeTok{ylab =} \StringTok{"log References"}\NormalTok{,}
        \DataTypeTok{main =} \StringTok{"References to specific phyla in PubMed search for }\CharTok{\textbackslash{}"}\StringTok{Human Microbiome}\CharTok{\textbackslash{}"}\StringTok{"}\NormalTok{,}
        \DataTypeTok{col =}\NormalTok{ cols,}
        \DataTypeTok{names.arg =}\NormalTok{ phylogeny}\OperatorTok{$}\NormalTok{Name,}
        \DataTypeTok{log =} \StringTok{"y"}\NormalTok{,}
        \DataTypeTok{xaxt =} \StringTok{"n"}
    \NormalTok{)}
    \CommentTok{\# Add X{-}axis labels}
    \KeywordTok{text}\NormalTok{(}
    \NormalTok{    plot,}
        \FloatTok{0.95}\NormalTok{,}
        \DataTypeTok{labels =}\NormalTok{ phylogeny}\OperatorTok{$}\NormalTok{Name,}
        \DataTypeTok{srt =} \DecValTok{45}\NormalTok{,}
        \DataTypeTok{adj =} \KeywordTok{c}\NormalTok{(}\FloatTok{1.1}\NormalTok{,}\FloatTok{1.1}\NormalTok{),}
        \DataTypeTok{xpd =} \OtherTok{TRUE}\NormalTok{,}
        \DataTypeTok{cex =} \FloatTok{0.8}
    \NormalTok{)}
    \CommentTok{\# Add legend for domain colors}
    \KeywordTok{legend}\NormalTok{(}
        \StringTok{"topright"}\NormalTok{,}
    \NormalTok{    plot,}
        \DataTypeTok{legend =}\NormalTok{ colors}\OperatorTok{$}\NormalTok{Domains,}
        \DataTypeTok{fill =}\NormalTok{ colors}\OperatorTok{$}\NormalTok{Colors}
    \NormalTok{)}
    \end{Highlighting}
    \end{Shaded}

\end{document}