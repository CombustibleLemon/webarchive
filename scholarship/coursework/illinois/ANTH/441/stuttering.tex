% find this class at https://archive.danleonard.us/scholarship/coursework/coursework.cls
\documentclass{../../../coursework}

\title{The Monster Study}
\subtitle{}
\author{Daniel}{Glenn}{Leonard}
\newdate{date}{17}{12}{2019}
\date{\displaydate{date}}
\course{ANTH}{441}{Human Genetics}{University of Illinois at Urbana-Champaign}
\instructor{}{Karthik}{}{Yarlagadda}

\keywords{stuttering, ethics, legal, science, scandal}
\addbibresource{stuttering.bib}
%\addbibresource[location=remote]{https://archive.danleonard.us/scholarship/coursework/illinois/ANTH/441/stuttering.bib}
\baseurl{https://archive.danleonard.us/scholarship/coursework/illinois/ANTH/441/stuttering.xhtml}

\begin{document}

\maketitle

\section{Introduction}

In 1939, University of Iowa graduate student Mary Tudor submitted a
thesis attempting to tease out the psychological risk factors for the
development of stuttering. Studying under Dr. Wendell Johnson at the
genesis of speech pathology as a scientific discipline, her research was
guided by one of Dr. Johnson's early hypotheses on the origin of the
disability. Sixty-two years later, this study was brought to the
forefront of the popular press as the ``Monster Study,'' the title of
Jim Dyer's \parencite*{Dye01} nationally-syndicated column which evoked shock,
outrage, and debate, even inspiring a musical composition \parencite{Gar191}.
In the intervening two decades, scholars have debated the
outcomes and implications of the work by Tudor and Johnson.

\section{The Study}

Johnson suffered from stuttering since the age of five \parencite{GoldfarbEthics01},
which personally affected him greatly and drove much of his career. In
the early twentieth century, stuttering was considered innate, an
incurable symptom of a broader intellectual disability \parencite{GoldfarbEthics08}.
More specifically, the leading theory by the 1930s was of
incomplete brain hemisphere dominance, in which the right hemisphere is
unable to maintain control of the body, leading to stuttering and the
(at the time, dreaded) left-handedness \parencite{Sil88}. As a
successful academic, Johnson rejected this approach, setting his career
ambitions to find a cure for stuttering and learn how to prevent it in
others \parencite{GoldfarbEthics03}. Looking inward, he developed his
diagnosogenic hypothesis.

\subsection{Diagnosogenic Hypothesis}

While the theory does not develop until 1959, the first precursors to
Johnson's diagnosogenic hypothesis can be traced to earlier works of
his. In ``Psychological considerations of stuttering'' \parencite*{Joh36}, Johnson
explores the sufferer's attitude towards their own condition. He makes
two propositions: first, that people who stutter desire to not stutter,
and second, that people who stutter expect to stutter just before
speaking. Synthesizing these two, he concludes that there exists in
stuttering individuals an inhibition of speech that results directly
from the fear of negative connotations associated with stuttering. More
succinctly, people who stutter do so in part because of a psychological
fear of their own expectations to stutter. This self-inhibitory
hypothesis also explained stutterers' low socioeconomic position in a
way better than that of intellectual disability: ``the stutterer is shy
and seclusive, and he tends to seek vocations which demand little or no
normal speech'' \parencite[23]{Joh36}.

To flesh out a more concrete theoretical foundation for the basis of
stuttering, Johnson examined the psychiatric diagnosis of childhood
stuttering through this self-inhibitory lens. Three main points underlie
the methodology and pursuit of Tudor's thesis: first, that the speech of
stuttering children at the time of diagnosis is indistinguishable from
normal childhood disfluencies \parencite{GoldfarbEthics02}; second, the initial
diagnosis comes from laypeople, usually parents or teachers \parencite{Joh42};
third, the child diagnosed as a stutterer begins to experience
the aforementioned self-inhibitory phenomenon \parencite{Joh36}. To
confirm or reject this hypothesis, a controlled experiment would be
necessary.

\subsection{Methodology}

Mary Tudor was assigned a project to assess the impact of diagnosis on
stuttering children. Johnson sent her to the Soldiers and Sailors
Orphans' Home in Davenport, Iowa where there was an experimental
population of children. Of 256 orphans, 22 were selected for the study
and individually divided into four groups, reproduced in full:

\begin{quote}
\textbf{Group IA} consisted of five children who had been labelled
``stutterers'' by members of the institution. An attempt was made to
remove the label ``stuttering'' from the children in this group; that
is, they were told that they were not stutterers, but normal speakers
who had been erroneously called stutterers.\\
\textbf{Group IB} consisted of the other five children who had been
labelled ``stutterers'' by members of the institution. In the case of
these children the judges endorsed the label.\\
\textbf{Group IIA} consisted of six normal speakers with varying degrees
of fluency. To this group the judges attached the label ``stuttering'';
that is, they were told that the type of speech interruptions they were
having indicated that they were stutterers.\\
\textbf{Group IIB} consisted of six normal speakers matched in age, sex,
intelligence, and fluency with the corresponding six normal speakers in
group IIA. No negative evaluative label was attached to this group.

\parencite[4]{Tud39}
\end{quote}

Over the course of the semester-long study, orphanage instructors would
treat the children according to their experimental group rather than to
their previous diagnosis. Tudor engaged in regular speech recording
sessions with each child, reproducing their speech transcripts in full
in her thesis as the data for analysis.

\subsection{Results}

As reported in the thesis, there was no significant result for
individuals in groups IA, IB, or IIB. The sole consequential result
Tudor found was in group IIA, the normal speakers told that they
stuttered. As Tudor writes in her discussion, ``a decrease in verbal
output was characteristic of all six subjects; that is, they were
reluctant to speak and spoke only when they were urged to'' \parencite[147]{Tud39}.

\section{Response}

Group IIA was also to be the most controversial aspect of the study upon
its public exhumation. While the study had been documented previously in
the academic press \parencite[see][]{Sil88}, it was an article in the
\emph{San Jose Mercury News} that brought controversy and 2006 edited
book on the subject.

\subsection{From the News Media}

Dyer's \parencite*{Dye01} news article was syndicated widely and brought great
attention to the University of Iowa. In ``Ethics and Orphans: The
`Monster Study,''' he describes an unethical experiment which brought
permanent hardship to a group of Iowa orphans at the same time as Nazi
scientists' similar work across the ocean. The ethical issues with
Tudor's research, as described in the article, were many: it was
intended from the outset to induce stuttering in children and did so
successfully, it used children as subjects, and it did not involve
informed consent from all people involved, among others.

Following the article's publication, the six former members of group IIA
sued the State of Iowa in 2003, seeking \$13.5 million in compensation
for a lifetime of psychological harm. In 2007, the case was settled with
the state awarding \$900,000 in compensation to the six participants
\parencite{BBC07}.

\subsection{From Academia}

Despite this fanfare, academics have been far more defensive of the
study. \textcite{Amb02} reanalyzed Tudor's data to conclude that
Tudor's own conclusion was wrong: ``assessments of the two types of
quantifiable data, perceptual and speech, clearly indicate that all four
experimental questions were answered in the negative'' (p. 199). The
labeling of group IIA as having decreased in verbal output was unfounded
by Tudor's own data, let alone Dyer's deeming them as permanent
stutterers. Furthermore, \textcite{Amb02} find no aspect of the
study that would violate ethical norms of 1939, despite their initial
reservations that ``it is unquestionable that the study was ethically
wrong'' \parencite[17]{Yai01}.

\textcite{GoldfarbEthics04} later went on to criticize Dyer directly, outlining many
false and misleading statements in the original \emph{San Jose Mercury
News} article. Specifically, the posthumous diagnosis of group IIA
children as having become ``stutterers'' when even Tudor described only
mild disfluency, depicting Johnson's diagnosogenic theory as having
already been defined rather than in its earliest stages, and painting
the article as an exposé when the thesis had been checked out multiple
times from its depository in public archives for over half a century.

Regarding the specific ethical questions of Tudor's study,
\textcite{GoldfarbEthics05} argue that the most pernicious may be
Johnson's imposition of a research project on Tudor, viewing it as a
type of professor-student relation that is no longer acceptable in
academic culture. Asking whether such a study would pass an IRB today,
\textcite{GoldfarbEthics06} contends that the Tudor study, as conducted, would not,
for the reason that negative-feedback experiments are rarely acceptable
in today's research environment.

In contrast, \textcite{GoldfarbEthics07} refutes the criticism of Johnson having
performed the study as the test of an unexplored hypothesis, listing
several seminal works in the field which were only possible because they
were exploratory of new ideas. Most radically, \textcite{GoldfarbEthics09}
defends the study outright, explaining several postulates: harm was
neither intended nor done, as there is no evidence for permanent
repercussions; children were the only acceptable subject population for
the hypothesis, being based on developmental speech pathology; informed
consent in this case was provided, as although deception of the children
was required the administrator of the orphanage provided consent; the
experiment was limited in scope and time, being limited to only a few
children and lasting only one semester; finally, there was adequate
post-study care for the subjects. In a departure from his peers, he
argues the Tudor study would not only pass IRB approval in today's
academic environment, but that it is radically safer than much
contemporary research.

\section{Discussion}

Personally, I am convinced by Nicholas Johnson's defense of Tudor's
study. Although its methodological flaws outlined by \textcite{Amb02}
preclude the study's use as empirical science, I find no ethical
flaws in the methodology or practice of the study, especially in the
context of 1939. In my own reading of Tudor's thesis, I see a graduate
student who performed an experiment with all the care and nuance
human-subjects research requires. As an anthropology student, the study
is notable to me in that it took great care to describe the thoughts and
feelings of children with and without the ``stutterer'' label. A great
departure from the usual early-20\textsuperscript{th}-century
scholarship of disability as an innate flaw, Tudor's work reads almost
as an ethnography of identity and disability. The only flaw I find in
the thesis is the glaring lack of a fully fleshed-out theoretical
foundation, as the introduction section is startlingly brief. Whether
this was an artifact of its time I am not sure, but a more grounded
introduction would surely be necessary in contemporary research.

Save for the settlement from the State of Iowa, the study had no impact
on ethical guidelines of today. Being already more than a half-century
old at the time of the media frenzy, IRB guidelines already existed in
very well-defined terms, and there was no change to how they operated as
a direct result.

\printbibliography

\end{document}
