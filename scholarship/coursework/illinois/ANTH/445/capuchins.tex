% find this class at https://archive.danleonard.us/scholarship/coursework/coursework.cls
\documentclass[american]{../../../coursework}

\title{Social Spacing and Subgrouping in One Community of White-Faced Capuchins (\emph{Cebus~capucinus})}
\subtitle{}
\shorttitle{Social Spacing in White-Faced Capuchins}
\author{Daniel}{Glenn}{Leonard}
\newdate{date}{12}{2}{2018}
\date{\displaydate{date}}
\course{ANTH}{445}{Primate Behavior, Ecology, Ecosystem Sustainability, and Conservation}{University of Illinois at Urbana-Champaign}
\instructor{Dr}{Paul}{Alan}{}{Garber}

\keywords{New World monkeys, \emph{Cebus}, age, activity, diet, social spacing, subgroup}
\addbibresource[location=remote]{https://archive.danleonard.us/scholarship/coursework/illinois/ANTH/445/capuchins.bib}
\baseurl{https://archive.danleonard.us/scholarship/coursework/illinois/ANTH/445/capuchins.xhtml}

\begin{document}
\setcounter{page}{10}
\maketitle

\begin{abstract}
    I studied social spacing in one group of eight white-faced capuchin
    monkeys (\emph{Cebus~capucinus}) over seven mornings at La~Suerte
    Biological Field Station, Costa Rica. I recorded 528 instantaneous focal
    animal samples, consisting of individuals' activity, diet, distance from
    their nearest neighbor, diameter of the occupied tree crown, and the age
    composition of individuals sharing the same tree crown.
    \emph{C.~capucinus} spent a majority of their time feeding and foraging,
    and overwhelmingly consumed fruit during the study period. Individuals
    were alone in trees during 53.6\% of samples, and subgroups averaged 2.1
    (\(\sigma = 1.8\)) individuals. Average subgroup size differed by focal
    animal activity, from 1.4 (\(\sigma = 0.8\)) individuals while traveling
    to 2.8 (\(\sigma = 1.1\)) while socializing. Focal animals were most often
    (50.6\%) at least ten meters from their nearest neighbor. Excluding
    socialization, distance from neighbors was greatest while feeding and
    least while traveling.
\end{abstract}

\printkeywords

\section{Introduction}

Capuchin monkeys are classified as omnivores, consuming most of their calories
from fruit and their protein from invertebrate prey. In order to consume
enough insect protein to satiate their 4 kg~body size, \emph{Cebus~capucinus}
must spend a great deal of time foraging, and are thus considered ``energy
maximizers'' \parencite{Fragaszy2004}. The species shows a high degree of
problem-solving skill and shows marked ability to consume hard-to-access
foods, with the thickest tooth enamel of any nonhuman primate relative to
tooth size \parencite{Jack2011}. The mean \emph{C.~capucinus} group size is
16.36 individuals, with a male-to-female ratio of 0.71
\parencite{Fragaszy2004}.

An animal's placement within its group has a profound impact on its feeding
success as well as the time it spends in vigilance (as a proxy for risk of
predation). In a 26-month study of wedge-capped capuchins
(\emph{Cebus~olivaceus}) in Venezuela, \textcite{Robinson1981} documented
individuals' foraging success correlated directly with the distance from their
nearest neighbor, but increased distance also led to more time spent in
vigilance against predators. In a similar study of intragroup spacing in
\emph{C.~capucinus}, \textcite{Hall1997} recorded dominant adults occupying a
front-center position within the group, balancing predation risk with access
to food. It has been argued, however, that because \emph{C.~capucinus} tend to
forage independently while \emph{C.~capucinus} forage successively, intragroup
feeding competition may not play a major role in the former's spatial behavior
\parencite{Fragaszy2004}.

The vast majority of mammals are not social, yet 80\% of interactions between
New World monkeys are affiliative \parencite{Sussman2005}. Thus, it would be
expected that there are some benefits to group cohabitation. Many benefits
have been hypothesized or identified, including infant care, reduced predation
risk, and better reproductive success \parencite{Fragaszy2004}. The most
prevalent theory is that of feeding competition, describing primate groups as
large enough to allow for predator detection yet small enough to minimize
intragroup feeding competition. \textcite{Sussman2011} argue that there is
little empirical basis for theories to accurately depict predation risk; they
propose that primate social groups are more flexible than normally assumed,
explaining why directly agonistic behavior accounts for a minimal amount of
primate activity.

Relative to body size, primates have evolved a juvenile period much longer
than all other mammals \parencite{Harvey1985}. This juvenile period is
associated with significant development of social skills \parencite{Joffe1997}
and foraging behavior \parencite{Rapaport2008}. \textcite{Joffe1997} also
found that length of a species' juvenile period correlated positively with
group size, suggesting that greater social development allows for larger
social structures. It has been theorized that juveniles may learn certain
behaviors by staying in close proximity to adult role models
\parencite{Sherrow2011} Alternatively, \textcite{Strier2007} hypothesizes that
they learn social skills by preferring proximity to other juveniles to
facilitate play. In \emph{C.~capucinus}, juveniles have been observed actively
begging for food from adults \parencite{Perry1994} and show markedly more
interest in adult food-processing behavior than do adults
\parencite{OMalley2005}. To facilitate such interactions, it was predicted
that juveniles would physically associate with other individuals at a higher
rate and in closer proximity than would adults.

\section{Methods}

\subsection{Study Site}

La Suerte Biological Field Station is located in northeastern Costa Rica
(10.442°~N, 83.771°~W), approximately \SI{50}{\metre} above sea level. The
site covers 300~ha of tropical forest as well as \SI{300}{\hectare} of
formerly-logged secondary forest and regenerating pasture
\parencite{Mallott2017}. Bisecting the land is a river, Río La Suerte, which
flows to the Caribbean. Annual precipitation in the region averages
\SI{3962}{\milli\metre}, peaking in June-July and November-December
\parencite{Sanford1994}. Neighboring the field station are several plantations
of coconut and oil palms, of which the coconut plantation was observed to be a
feeding site for the capuchins. In addition to white-faced capuchins, the site
is also inhabited by mantled howler monkeys (\emph{Alouatta~paliatta}) and
black-handed spider monkeys (\emph{Ateles~geoffroyi}), which were both
observed at least once during the study period occupying the same tree crown
as \emph{C.~capucinus} individuals.

\subsection{Study Group}

One habituated group of white-faced capuchins was identified. The group was
not marked and thus it was not possible to record data on or sex particular
individuals. It was nevertheless estimated that the group consisted of eight
individuals (5~adults, 3~juveniles) during the study period. No individuals
were identified as infants. Juveniles were defined as individuals not
dependent on their mother yet still immature, with age determined based on
size relative to adults.

\subsection{Observational Data Collection}

Information on subgrouping and individuals' activity was collected at
two-minute intervals from dawn until midday using an instantaneous focal
animal sampling technique \parencite{Altmann1974}. 528 samples were collected
from January~2 to 8, 2018, totaling \SI{17.6}{\hour} of behavioral data
(see~\hyperref[sec:appendix_a]{Appendix A} for a template of the data
recording format). An effort was made to follow focal animals for
\SI{2}{\hour} before changing to a second target, but the capuchins' rapid
locomotion made losing sight of focal animals a regular occurrence.

\subsubsection{Activity budget}

Information on focal animals' activity and diet was recorded. Activities were
defined as: feeding -- manipulation and ingestion of a potential food item;
foraging -- locomotion within the same tree crown for the immediate purpose of
obtaining food; traveling -- locomotion within the same tree crown or adjacent
tree crowns not for the immediate purpose of obtaining food; social --
reciprocal affiliative or agonistic interaction with at least one other
individual; resting -- a period of inactivity; other -- any activity not
otherwise listed; unknown -- not able to discern. Diet was recorded as fruit,
leaves, flowers, or invertebrate prey, as well as unknown. When possible,
fruit consumed was identified to species. Plant identification was performed
with the assistance of local staff supplemented with botanical reference texts.

\subsubsection{Social spacing}

The identity of and distance to a focal animal's nearest
neighbor was recorded each sampling interval. Nearest neighbor was considered
to be the \emph{C.~capucinus} individual closest to the focal animal in
three-dimensional space without regard to canopy terrain. Distances were
estimated visually as ``in contact'' (\SI{0}{\metre}), less than~1,
\numrange{1}{3}, \numrange{3}{5}, \numrange{5}{10}, and greater than
\SI{10}{\metre}. Similarly, the diameter of the tree crown in which a focal
animal resided was visually estimated to less than 5, \numrange{5}{10},
\numrange{10}{15}, and greater than \SI{15}{\metre}. The number of individuals
in the same tree crown was considered to be a subgroup, assuming that a focal
animal has some affinity with those it shares a tree with
\parencite{Bezanson2008}, and the age composition of such subgroups was also
recorded.

\subsection{Interpretation}

Data was converted from human-\ to machine-readable format in C\# and compiled
in Microsoft Excel. Bivariate analyses were generated using PivotTables.

In five samples (1\% of total), the focal animal was recorded as on the forest
floor. Because subgroups were defined based on the animal's present tree crown,
these data points were not used in the analysis. Values of ``other'' and
``unknown'' consisted

\section{Results}

\begin{table}
    \caption{Activity Budget and Age}\label{tbl:activity_age}
    \begin{tabu} to \linewidth{XXXXXXXX}
        \toprule
        && \multicolumn{6}{c}{Activity (\%)} \\
        \cmidrule{3-8}
        & N & Feeding & Foraging & Traveling & Socializing & Resting &
        Unknown \\
        \midrule
        Adult & 318 & 37.4 & 24.5 & 28.9 & 3.1 & 5.7 & 0.3 \\
        Juvenile & 210 & 50.0 & 20.5 & 27.1 & 0.0 & 1.9 & 0.5 \\
        \textit{Total} & 528 & 42.4 & 22.9 & 28.2 & 1.9 & 4.2 & 0.4 \\
        \bottomrule
    \end{tabu}
    \par\vspace{.5em}{\footnotesize\emph{Note}: N, sample size; \%, percent of
    age class data (rows add to 100\%); Total, all activities regardless of
    age}
\end{table}

\begin{table}
    \caption{Diet and Age}
    \label{tbl:diet_age}
    \begin{tabu} to \linewidth{XXXXXXX}
        \toprule
        && \multicolumn{5}{c}{Food Type (\%)} \\
        \cmidrule{3-7}
        & N & Fruit & Leaves & Flowers & Invertebrate & Unknown \\
        \midrule
        Adult & 119 & 83.2 & 3.4 & 0.0 & 10.1 & 3.4 \\
        Juvenile & 105 & 81.9 & 1.9 & 1.0 & 15.2 & 0.0 \\
        \textit{Total} & 224 & 82.6 & 2.7 & 0.4 & 12.5 & 1.8 \\
        \bottomrule
    \end{tabu}
    \par\vspace{.5em}{\footnotesize\emph{Note}: N, sample size; \%, percent of
    age class data (rows add to 100\%); Total, all activities regardless of
    age}
\end{table}

\begin{table}
    \caption{Age Composition of Subgroups}
    \label{tbl:subgroup_age}
    \begin{tabu} to \linewidth{X[.5l]X[.1]X[.7]X[.7]X}
        \toprule
        && \multicolumn{2}{c}{Average representation of age group (\%)} \\
        \cmidrule{3-4}
        Subgroup size & N & Adults & Juveniles & Adult difference from whole
        (\%) \\
        \midrule
        1 & 269 & -- & -- & -- \\
        2 & 103 & 65.2 & 34.8 & 4.3 \\
        3 & 69 & 64.4 & 35.6 & 3.1 \\
        4 & 16 & 70.8 & 29.2 & 13.3 \\
        5 & 9 & 60.0 & 40.0 & -4.0 \\
        6 & 7 & 66.7 & 33.3 & 6.7 \\
        8 & 29 & 62.5 & 37.5 & -- \\
        \bottomrule
    \end{tabu}
    \par\vspace{.5em}{\footnotesize\emph{Note}: No recorded subgroups of size
    7; Subgroups of size 1 excluded from age distribution; Subgroups of size 8
    equivalent to age composition of entire \emph{C.~capucinus} group; ``adult
    difference from whole'', difference between percentage of adults for
    subgroup size and percentage of adults in group as a whole; N, sample
    size; \%, percent of individuals making up subgroup on average}
\end{table}

\subsection{Activity and Diet}

Members of both age groups spent a plurality of their time engaged in feeding;
however, juveniles fed much more (50\%) than did adults (37\%). This greater
time spent feeding came at a cost to time spent in all other behaviors
compared to adults (Table~\ref{tbl:activity_age}). \emph{C.~capucinus}
individuals spent the vast majority of their feeding time consuming fruit
(83\%). While fruit consumption did not differ greatly by age, juveniles
consumed more (15\%) invertebrates than did adults (10\%) and conversely for
leaf consumption (Table~\ref{tbl:diet_age}). Juveniles were never recorded
socializing, and adults were only recorded socializing in 3.1\% of samples
(Table~\ref{tbl:activity_age}). Of these ten recorded social interactions,
three were agonistic, five affiliative, and two unknown.

\subsection{Subgrouping}

Subgroups averaged 2.12 (\(\sigma = 1.80\)) members in the same tree crown. In
a majority of samples (53.6\%), \emph{C.~capucinus} individuals were observed
alone in a tree; thus, the modal subgroup size is 1. Age composition of
subgroups differed minimally by age. Excluding subgroups of size 1 and 8, the
fraction of adults making up a subgroup averaged a 4.7\% difference from the
fraction of adults making up the group as a whole. The greatest difference was
observed in subgroups of size 4, which on average contained 13.3\% more adults
than would be expected (Table~\ref{tbl:subgroup_age}).

Also examined was the difference in subgroup size based on the size of the
tree crown in which the subgroup resided. A strong direct correlation was
observed, with tree crowns with diameter less than \SI{5}{\metre} averaged 1.2
(\(\sigma = 0.5\)) individuals, while trees greater than \SI{15}{\metre} in
diameter averaged 3.4 (\(\sigma = 2.6\)) individuals
(Table~\ref{tbl:subgroup_crown}). As with subgroup size in general, there was
no correlation between tree crown size and the age composition of subgroups;
age composition of subgroups reflected the composition of the
\emph{C.~capucinus} group as a whole.

Subgroup size was observed to be related to a focal animal's activity.
Subgroups ranged from an average of 1.4 (\(\sigma = 0.8\)) individuals when
traveling to 2.8 (\(\sigma = 1.1\)) individuals when the focal animal was
engaged in social interaction. Feeding activity represented the median
subgroup size, 2.4 (\(\sigma = 2.0\)) individuals
(Table~\ref{tbl:subgroup_activity}). Again, age composition differed minimally
between activities, after taking into account different activity budgets of
the age groups.

\subsection{Distance to Nearest Neighbor}

In addition to the size of subgroups, observations were made on the distance
to a focal animal's nearest neighbor. The majority (50.6\%) of observations
found the focal animal greater than ten meters from its nearest neighbor, and
only 5.6\% of samples observed animals within one meter of one another,
including in contact. The preference for distances greater than
\SI{10}{\metre} extended to all activities except socialization, which was the
only activity not to have the plurality of observations see animals
\SI{10}{\metre} from each other. Excluding the low-sample-size activities of
socialization (N=10) and resting (N=21), the most likely activity to be
greater than \SI{10}{\metre} apart was feeding (53.7\%) and least likely was
traveling (46.8\%). This correlation extended across both age classes; however,
regardless of activity, juvenile focal animals were more likely to be greater
than \SI{10}{\metre} apart from their nearest neighbor than were adults
(Juveniles: 55.1\%; Adults: 47.7\%). Juveniles were also never observed in
physical contact to their nearest neighbor, while adults were observed in
contact in 3.9\% of samples (Table~\ref{tbl:neighbor_activity}).

\section{Discussion}

\begin{table}
    \caption{Subgroup Size and Tree Crown Diameter}
    \label{tbl:subgroup_crown}
    \begin{tabu} to \linewidth{XXXX}
        \toprule
        Tree diameter (\si{\metre}) & N & Average subgroup size & \(\sigma\) \\
        \midrule
        \num{< 5} & 86 & 1.2 & 0.5 \\
        \numrange{5}{10} & 149 & 1.5 & 1.0 \\
        \numrange{10}{15} & 121 & 1.9 & 1.0 \\
        \num{> 15} & 167 & 3.4 & 2.6 \\
        \bottomrule
    \end{tabu}
    \par\vspace{.5em}{\footnotesize\emph{Note}: N, sample size; \(\sigma\),
    one standard deviation of subgroup size}
\end{table}

\begin{table}
    \caption{Subgroup Size and Activity}
    \label{tbl:subgroup_activity}
    \begin{tabu} to \linewidth{XXXX}
        \toprule
        Activity & N & Average subgroup size & \(\sigma\) \\
        \midrule
        Feeding & 224 & 2.4 & 2.0 \\
        Foraging & 121 & 2.6 & 2.3 \\
        Traveling & 149 & 1.4 & 0.8 \\
        Socializing & 10 & 2.8 & 1.1 \\
        Resting & 22 & 1.6 & 1.1 \\
        \bottomrule
    \end{tabu}
    \par\vspace{.5em}{\footnotesize\emph{Note}: N, sample size; \(\sigma\),
    one standard deviation of subgroup size}
\end{table}

\begin{table}
    \caption{Nearest Neighbor Distance and Activity}
    \label{tbl:neighbor_activity}
    \begin{tabu} to \linewidth{XX[.25]X[1.4]XXXXX}
        \toprule
        && \multicolumn{6}{c}{Nearest neighbor distance (\%)} \\
        \cmidrule{3-8}
        Activity & N & \SI{0}{\metre} (in contact) & \SI{< 1}{\metre} &
        \SIrange{1}{3}{\metre} & \SIrange{3}{5}{\metre} &
        \SIrange{5}{10}{\metre} & \SI{> 10}{\metre} \\
        \midrule\rowfont{\bfseries}
        Adult & 310 & 3.9 & 2.9 & 6.1 & 16.1 & 23.2 & 47.7 \\
        ~Feeding & 119 & 1.7 & 1.7 & 10.9 & 17.6 & 16.8 & 51.3 \\
        ~Foraging & 76 & 0.0 & 3.9 & 3.9 & 18.4 & 30.3 & 43.4 \\
        ~Traveling & 88 & 1.1 & 2.3 & 3.4 & 15.9 & 33.0 & 44.3 \\
        ~Socializing & 10 & 90.0 & 10.0 & 0.0 & 0.0 & 0.0 & 0.0 \\
        ~Resting & 17 & 0.0 & 5.9 & 0.0 & 5.9 & 0.0 & 88.2 \\
        \midrule\rowfont{\bfseries}
        Juvenile & 198 & 0.0 & 3.5 & 2.5 & 20.2 & 18.7 & 55.1 \\
        ~Feeding & 99 & 0.0 & 1.0 & 1.0 & 22.2 & 19.2 & 56.6 \\
        ~Foraging & 42 & 0.0 & 2.4 & 4.8 & 16.7 & 16.7 & 59.5 \\
        ~Traveling & 53 & 0.0 & 9.4 & 1.9 & 20.8 & 17.0 & 50.9 \\
        ~Socializing & 0 & 0.0 & 0.0 & 0.0 & 0.0 & 0.0 & 0.0 \\
        ~Resting & 4 & 0.0 & 0.0 & 25.0 & 0.0 & 50.0 & 25.0 \\    
        \midrule\rowfont{\bfseries}
        \textit{Total} & 508 & 2.4 & 3.2 & 4.8 & 17.7 & 21.5 & 50.6 \\
        ~Feeding & 218 & 0.9 & 1.4 & 6.4 & 19.7 & 17.9 & 53.7 \\
        ~Foraging & 118 & 0.0 & 3.4 & 4.2 & 17.8 & 25.4 & 49.2 \\
        ~Traveling & 141 & 0.7 & 5.0 & 2.8 & 17.7 & 27.0 & 46.8 \\
        ~Socializing & 10 & 90.0 & 10.0 & 0.0 & 0.0 & 0.0 & 0.0 \\
        ~Resting & 21 & 0.0 & 4.8 & 4.8 & 4.8 & 9.52 & 76.19 \\  
        \bottomrule
    \end{tabu}
    \par\vspace{.5em}{\footnotesize\emph{Note}: N, sample size; \%, percent of
    samples (rows add to 100\%); Total, all activities regardless of age}
\end{table}

\subsection{Activity and Diet}

Juveniles' time spent feeding and foraging was greater than that of adults,
consistent with previous studies of \emph{C.~capucinus} at La Suerte
Biological Field Station \parencite{Bezanson2009}. Capuchins' primary food
source being fruit was also consistent with most previous literature on
\emph{Cebus} primates \parencite{Jack2011}.

\subsection{Sociality}

Capuchins preferred overwhelmingly to feed and forage alone or in small
subgroups, consistent with \textcite{Phillips1995}'s finding in
\emph{C.~capucinus} on Barro Colorado Island, Panama. Individuals were most
likely to be over ten meters away from their nearest neighbor when engaged in
feeding, and were very rarely observed feeding within three meters of another.
Even in trees of large diameter, subgroup size did not increase a commensurate
amount. This suggests minimal intragroup feeding competition, consistent with
\textcite{Sussman2011}'s hypothesis that feeding competition contributes
little to primate social structure.

The time spent in direct social interaction was minimal, significantly less
than that recorded in past studies of \emph{C.~capucinus}
\parencite{Sussman2005}. This is possibly due to the study being limited to
morning hours, and did not observe capuchins during the hottest period of the
day, when the study group reportedly engages in more rest (P.~A.~Garber, personal communication, Jan.~12, 2018). More 

\subsection{Ontogeny}

Ontogeny appears to have played little role in the observed social structure
of the study group. Adults and juveniles were just as likely to occupy
subgroups of all sizes, trees of all sizes, and spent similar amounts of time
in each activity. This is similar to earlier findings of \emph{C.~capucinus}
positional behavior at La Suerte, in that juveniles resemble adults at six
months of age regardless of their sexual immaturity \parencite{Bezanson2009}.
In fact, juveniles were significantly more likely than adults to be over
\SI{10}{\metre} from their closest neighbor. This observation contrasts with
the prediction that they would prefer nearby neighbors for either social
development \parencite{Strier2007} or mimicry of adult behavior \parencite{Sherrow2011}.

\printbibliography

\setcounter{secnumdepth}{0}

\section{Acknowledgements}

I would like to thank Prof.\ Emeritus Paul A. Garber for instructing me in the
methods of field primatology and introducing me to a new culture as part of
his final ANTH 445 class at La Suerte Biological Field Station, as well as his
wife Chrissy McKenney. Thanks as well to Agustín Flores, who continually
ensured -- across a language barrier -- I didn't endanger myself and without
whose assistance I would have collected a tenth of the data that I now have.
The staff at La Suerte made my stay wonderful, from cooking dozens of meals to
restocking toilet paper in my cabin without me noticing. Finally, to my
parents, Glenn and Mary Leonard, who generously funded this expedition as well
as a significant portion of my college career.

\newpage
\begin{landscape}
    \section{Appendix A}
    \label{sec:appendix_a}
    \setcounter{table}{0}
    \begin{longtabu} to \linewidth{X[0.5]*5{X[l]}*4{X[1.5l]}}
        \noalign{\phantomsection}\caption{Sample Data}\label{tbl:sample_data} \\
        \toprule
        Time & Focal Animal &
        Location\footnote{The nearest or most recently seen trail marker} &
        Activity & Diet & Social Interaction & Nearest Neighbor
        Distance\footnote{Distances estimated based on sight} &
        Subgroup Size\footnote{Number of individuals in the focal animal's
        tree crown, including the focal animal} & Subgroup
        Composition\footnote{Age composition of individuals in the focal
        animal's subgroup} & Crown Diameter \\
        \midrule
        8:02 & Juvenile & O40 & Social &  & Affiliative & In contact & 4 & AAJJ & \SI{< 5}{\metre} \\
        8:04 & Juvenile & O40 & Foraging & Fruit & & \SIrange{1}{3}{\metre} & 5 & Unknown & \SIrange{5}{10}{\metre} \\
        8:06 & Juvenile & O40 & Feeding & Invertebrates && \SIrange{3}{5}{\metre} & 5 & AAAJJ & \SIrange{10}{15}{\metre} \\
        8:08 & Juvenile & O40 & Resting &&& \SIrange{5}{10}{\metre} & 1 & J & \SI{> 15}{\metre} \\
        8:10 & Adult & O41 & Traveling &&& \SI{> 10}{\metre} & 2 & AJ & Unknown \\
        8:12 & Adult & O42 & Unknown &&& Unknown & Unknown & Unknown & Unknown \\
        8:14 & Adult & Unknown & Other &&& In contact & On ground & On ground & On ground \\
        \bottomrule
    \end{longtabu}
\end{landscape}

\end{document}
