% find this class at https://archive.danleonard.us/scholarship/coursework/coursework.cls
\documentclass{../../../coursework}

\title{Do Hibernating Mammals Lose Bone Mass?}
\subtitle{A Review of Disuse Osteoporosis and Mammalian Torpor}
\author{Daniel}{Glenn}{Leonard}
\newdate{date}{16}{12}{2019}
\date{\displaydate{date}}
\course{ANTH}{408}{Human Evolutionary Anatomy}{University of Illinois at Urbana-Champaign}
\instructor{Dr}{John}{D.}{Polk}

\keywords{osteoporosis, astronaut, sedentary, evolution}
\addbibresource[location=remote]{https://archive.danleonard.us/scholarship/coursework/illinois/ANTH/408/hibernation.bib}
\baseurl{https://archive.danleonard.us/scholarship/coursework/illinois/ANTH/408/hibernation.xhtml}

\begin{document}

\maketitle

\section{Introduction}

Age-related osteoporosis affects tens of millions of individuals worldwide.
Estimations of disease prevalence are scarce for the developing world
\parencite{Han08}, but range in developed countries from 1\% to 8\% in men
and 9\% to 38\% in women above 50 years of age \parencite{Wad14}.
Understanding the biological mechanisms underlying the condition is an
important long-term pursuit of public health research.

Osteoporosis can also result from reduced mechanical loading on bones, classified
separately from age-related bone loss as disuse osteoporosis. The response of bone
to mechanical stimuli, altering its mass and morphology to meet the load-bearing
stressors demanded of it, has been long recognized as Wolff's Law \parencite{Wol86},
usually modeled as a dual feedback loop between bone stressors and bone deposition
or loss \parencite{Ruf06}. Heightened risk of bone fracture is
well-documented amongst patients following long-term hospitalization for stroke
\parencite{Kan01,Wei18} or spinal cord injury \parencite{Wan01}. Astronauts'
exposure to weightlessness results in drastic and rapid bone density loss
\parencite{LeB00,Hol98}, which has been replicated in mice \parencite{Agu09}
and rats \parencite{Gar00}.

Despite humans' and rats' experience of disuse osteoporosis, sedentary
lifestyles which would in humans contribute to significant bone loss -- for
instance, hibernation -- are found in multiple mammalian taxa including
Rodentia \parencite{Daa91,Mzi12}, Carnivora \parencite{Toi11}, and Primates
\parencite{Dau04}. The length of time many of these mammals spend in hibernation
is enough to cause extreme bone degradation in many non-hibernating animals.
How these species' regular torpor affects their bone growth and resorption can
give novel insights into medicine both terrestrial and in space.

\section{Central Questions in Forming Connections between Torpor and Osteoporosis}

Understanding the relationship between torpor and osteoporosis requires
investigation of three central questions:

\begin{enumerate}
    \item \emph{What physiological mechanisms underlie disuse osteoporosis?}
    Bone density is a complex physiological system mediated by multiple
    processes in different parts of the body. Teasing out which mechanisms
    are responsible for disuse osteoporosis in particular is necessary to
    understand how hibernation plays a role in its mediation.
    
    \item \emph{Do all hibernating animals share an evolutionary history?}
    Hibernation occurs in many families of mammals. If hibernation is an
    ancestral trait, then its absence in separate taxa may be linked to
    development of disuse osteoporosis in multiple groups independently;
    conversely, if hibernation is a derived trait in its separate taxa,
    then the prevention of disuse osteoporosis may have developed
    independently.
    
    \item \emph{Do hibernating animals experience disuse osteoporosis?}
    If hibernating animals do not experience a loss of bone density
    despite long-term sedentary practice, it can provide insights into
    how bone is formed and density regulated in all mammals.
\end{enumerate}

\subsection{Physiology of Disuse Osteoporosis}

Disuse osteoporosis, despite having structural similarity to that
associated with aging, is notable for its rapid onset. Cessation of
bone stress induces bone loss at ten times the rate of that occurring
in post-menopausal osteopenia \parencite{Blo10}, indicating a
different mechanism or combination of mechanisms inducing the
condition.

In humans, disuse osteoporosis occurs most commonly in bed rest
patients immobilized by a medical condition. However, the population
most well-suited as models for the phenomenon is astronauts, whose
dual status as government-employed scientists and test subjects
provides a wealth of information on their experience. Along with
the various other medical conditions associated with microgravity,
this spaceflight osteopenia presents a significant health risk to
astronauts. Recent estimations of bone density loss in low-Earth
orbit find astronauts lose on average 1\% of bone mass per month,
especially concentrated in the long bones \parencite{Lan04}.

Disuse osteoporosis traditionally presents as excess resorption of
bone, which has been disputed over whether via extended resorption
time \parencite{Bai90} or increase in abundance of osteoclasts
relative to osteoblasts \parencite{Wei89}.
Following this resorption, bones lose structural integrity and
mechanical properties \parencite{LiC05}, resulting in significantly
weaker bone \parencite{McG08CareyDon}. Not all bone
is equally susceptible to degradation -- trabeculae have been shown
to decrease in mass at a rate approximately fivefold that of
cortical bone \parencite{Lan04}, which has been attributed to
the greater surface area of trabecular bone for osteoclast function
\parencite{McG08CareyDon}.

While the usual recommendation to prevent bone loss is to increase
weight-bearing activity \parencite{Koh04},
there is some evidence to suggest there is more at play in disuse
osteopenia than a lack of bodyweight resistance. Even astronauts
keeping to the current proscribed ISS exercise regimen -- 2-3 hours
of resistance training per day and wearing clothing inducing
passive resistance -- is insufficient at reducing spaceflight
osteopenia \parencite{LeB00}. Attempts to address this
dichotomy have been highly varied and the subject of much academic
debate. Scholars disagree over which mechanical stimulus most
adequately explains Wolff's Law, whether fluid flow, strain rate,
magnitude, or distribution, or loading frequency \parencite{Pat20}.

In analyzing the impact of mechanical forces on bone cellular
structure, researchers have used whole-body vibration as a proxy
for the gravitational and muscle forces experienced in
non-sedentary life \parencite{Tot09}.
Whole-body vibration has been found to be strongly associated with
incidence of industrial injury, especially lower-back pain
\parencite{Car03}; however, \textcite{Rub01}
argue that at low magnitudes, vibration can be osteogenic to the
point of doubling the rate of bone growth. With such a wide gulf
between the magnitude of its harms and benefits, the medical
community remains apprehensive about vibration as an osteoporosis
therapy \parencite{Wys11} especially
considering the high variability in commercially-available
devices \parencite{Mui13}, but it remains a novel
way to understand bone loading.

Other researchers have looked towards the fluid pooling
associated with immobility as a causative agent, regardless
of weight-bearing \parencite{Blo10}. Within ten minutes of
unloading, rats' hindlimbs rapidly lose blood flow and cephalic
fluid volume, which becomes only further pronounced after 28
days \parencite{Col001}. In turkeys, immobilization
leading to significant bone loss has been mitigated by
artificially increasing fluid volume and pressure \parencite{Qin03}.
Given this, \textcite{Blo06,Blo10} has proposed interstitial
fluid pressure as the main predictive and diagnostic tool for
understanding disuse osteoporosis.

\subsection{Evolution and Physiology of Hibernation}

Mammals are not the only class that practice hibernation -- many
species of birds have been observed in a torpid state in both the
wild and induced conditions, but the physiology of their hibernation
is not well understood \parencite{Lym82}. Regardless, it provides
evidence for regular torpor as a widespread and evolutionarily old
trait. \textcite{Gei08} has argued for three separate events in which
chordates evolved torpor: a plesiomorphic evolution early in the
development of both marsupials and birds, and once as a derived
trait in placental mammals. \textcite{Kik10} used the probable
environmental conditions following the K-Pg asteroid impact to argue
for the mass extinction as a selection event in favor of hibernating
mammals. Rapid change in atmospheric O\textsubscript{3} concentration at the K-Pg
boundary is proposed as the leading cause of mammalian extinction,
where only torpid mammals were able to reduce their exposure to safe
levels. \textcite{Lov12} used a maximum likelihood phylogeny to trace
the history of mammalian hibernation beyond the K-Pg boundary to the
split between marsupials and placental mammals 144 mya, finding a ML
of 0.58 for torpor at the root of the tree.

Although the physiological function of torpor may be evolutionarily old,
its practice in individual species shows marked diversity and
adaptability. \textcite{Dau12} argue that
the differences in hibernation between three heterothermic strepsirrhine
species derive not from physiological changes but from evolutionary and
reproductive strategy. Torpor, while most known in the context of
overwintering, also has lesser-explored physiological and fitness
functions. Torpor can provide benefits in times of resource disadvantage,
postponing nutrient and water needs until more favorable periods.
Furthermore, individual species utilize hibernation for lesser-understood
functions such as female sperm storage and migratory fat buildup \parencite{Gei12}.
Another hibernation mechanism more analogous to that of bone deposition
may be the management of the immune system. Like osteons, immune cells are
dispersed throughout the body, created in marrow, and are heavily dependent
on serum composition. During hibernation, leukocyte count drops by 90\%,
adaptive immunity is drastically depressed, and antibody production
slows \parencite{Bou12}.

\subsection{Torpor and Osteoporosis}

Research into mammalian hibernation has found some evidence for bone
loss in small mammals. Decline in bone mass has been identified in bats
\parencite{Kro77}, hamsters \parencite{Ste81}, and ground squirrels \parencite{McG11}.
\textcite{Kro77} explain this phenomenon as
``obviously, the maintenance of calcium levels compatible with survival'' (p. 271);
however, the majority of these studies have been primarily observational and are
thus limited in explanatory potential \parencite{McG08CareyDon}.
Complicating this is the observation that while bats' bones decrease in mass
during hibernation, they remain until late in the hibernation period denser than
in the summer \parencite{Bru70}, potentially implying that bone is built up
in preparation for overwintering as is done with fat stores. More recent
quantitative work on hibernating ground squirrels has identified trabecular
microstructural bone loss, but without any corresponding loss in cortical bone
strength or macrostructure \parencite{McG11}. It is hypothesized that
these animals have an as-yet-unidentified biological mechanism to maintain bone
strength for future use while simultaneously using trabecular bone as mineral
storage for metabolism.

Bears' bones, in contrast, have long been understood to maintain mass throughout
hibernation. While past theories involved the idea of bone degradation followed
by rapid bone formation after leaving hibernation \parencite{McG08CareyDon},
\emph{in vivo} studies have found balanced levels of resorption and formation in
hibernating bears \parencite{McG08}, suggesting a physiological
adaptation to hibernation. Blood serum measurements in polar bears indicate that
bone formation may be accelerated during hibernation, although methodological
limitations preclude detailed data \parencite{McG08CareyDon}.
Furthermore, serum measurements are unable to assess localized differences in
bone degradation, which is important in researching disuse osteoporosis as long
bones are more susceptible than are other bones \parencite{Lan04}. Trabecular
bone, which in humans is the most affected by disuse, showed no difference between
bears killed just before or after hibernation \parencite{Par04},
lending credence to the theory that bears are able to maintain bone throughout
hibernation rather than rebuild in the spring and summer.

\section{Discussion}

Sampling bones of killed animals before and after hibernation, while able to
show whether they experience disuse osteoporosis, fails to indicate by what
mechanism this is mediated. Hibernation involves a number of physiological
behaviors that are of interest in considering bone loss. To increase body
temperature and stave off hypothermia, both bears \parencite{Lin04,Toi11}
and smaller mammals \parencite{Daa91,Gei04} are known to shiver on a regular
periodic basis. Following the hypothesis of low-magnitude vibration,
hibernating animals' shivering practices may be a mechanism to stimulate
bone growth and maintenance despite disuse. In contrast, \textcite{McG08CareyDon}
propose a number of neuronal and hormonal mechanisms that may mediate
osteon activity via leptin, parathyroid hormone, or norepinephrine.
However, these associations are at best correlative with no verifiable
biological pathway.

\subsection{Areas for Future Study}

Hibernation being a trait ancestral to mammals implies that the development
of disuse osteoporosis may be an evolutionarily recent trait, appearing
after the loss of hibernation in an ancestral lineage. As many of the
non-hibernating mammals are also those in lineages with larger brains,
\textcite{Hel18} propose hibernation as a constraint on
brain size evolution, using the expensive brain hypothesis. Adapting this
research to examine whether brain size evolution is also connected to the
physiological mechanisms underlying disuse osteoporosis could prove promising.

Even if hibernating mammals do have physiological mechanisms to prevent
osteoporosis from disuse, as there is clear evidence for in some species,
it is still notable that the animals are still operating at an energy
deficit. During hibernation, many animals excrete waste from their metabolic
processes, which contains calcium and other nutrients used to build and
maintain bone \parencite{McG08CareyDon}. In humans, decrease
in BMI within healthy boundaries has no correlated effect on bone strength
\parencite{Kel18}, but this is in a species that continues to consume
food and nutrients during weight loss. Hibernating animals' metabolic
processes, however slowed, is not similarly supplemented with an intake of
calories or nutrients. How mammals that excrete waste products during
hibernation can maintain bone structure is a potential area of further research.

One unexplored area of research is whether hibernating animals experience
disuse osteopenia even outside of the torpid state. Nearly all
limb-immobilization studies have been performed on animals who do not
hibernate, such as laboratory rats \parencite{Wei89} and
dogs \parencite{LiC05}. Performing such experiments on bears presents
many methodological difficulties, but could enlighten theories of how disuse
osteoporosis is prevented. Crucially, whether they experience disuse
osteoporosis outside of a state of torpor grants clues as to whether such a
process is exclusive to the special metabolic state experienced during
hibernation or whether it evolved as a separate process.

\printbibliography

\end{document}
