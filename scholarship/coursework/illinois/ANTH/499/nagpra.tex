% find this class at https://archive.danleonard.us/scholarship/coursework/coursework.cls
\documentclass{../../../coursework}

\title{International Repatriation in the Eurocentric World Order}
\subtitle{}
\author{Daniel}{Glenn}{Leonard}
\newdate{date}{06}{05}{2019}
\date{\displaydate{date}}
\course{ANTH}{499}{Ethics and NAGPRA}{University of Illinois at Urbana-Champaign}
\instructor{Dr}{Laura}{Lynn}{}{Shackelford}
\instructor{Dr}{Jenny}{L.}{}{Davis}

\keywords{}
\addbibresource{nagpra.bib}
%\addbibresource[location=remote]{https://archive.danleonard.us/scholarship/coursework/illinois/ANTH/499/nagpra.bib}
\baseurl{https://archive.danleonard.us/scholarship/coursework/illinois/ANTH/499/nagpra.xhtml}

\begin{document}

\maketitle

\begin{quote}
    \centering\emph{States shall provide redress through effective
    mechanisms, which may include restitution, developed in conjunction with
    indigenous peoples, with respect to their cultural, intellectual,
    religious and spiritual property taken without their free, prior and
    informed consent or in violation of their laws, traditions and customs.}
    
    --~\textcite[Article 11, Section 2]{Uni07}
\end{quote}

\section{Introduction}

On December 13\textsuperscript{th}, 2007, the United Nations ratified the
Declaration on the Rights of Indigenous Peoples (UNDRIP) after twenty-five
years spent languishing in committees. The Declaration asserted the rights of
indigenous peoples to self-determination and ceremonial expression, and
notably delineated the rights to ownership of cultural and spiritual property.
The Declaration is nonbinding on member states, and as such countries are free
to follow or ignore its provisions as they so desire; however, for many civil
rights activists it signifies a watershed moment in indigenous rights
\parencite{Wei12}. The U.N.\ itself describes it as a ``important standard for
the treatment of indigenous peoples and … a significant tool towards
eliminating human rights violations'' \parencite[2]{UniFAQ}. Notable are the
identities of the four states who voted against the Declaration: Australia,
Canada, New Zealand, and the United States, all settler-colonial countries
with large indigenous populations. The U.S.\ delegation in particular cited
qualms with the protections of self-determination and land rights, which it
found incompatible with its own national situation \parencite{Uni071}.

For peoples throughout the world, there exist cultural artifacts and human
remains that were taken from their original locations to museums and state
institutions without the knowledge or input of their owners. While many
settler-colonial states have taken efforts to repatriate artifacts
domestically, the issue of artifacts taken by outside states remains fraught
with conflict and debate over the merits of repatriation. Further complicating
the issue is the different processes of repatriation necessitated by two
different types of claimants: those who exist subnationally within states
versus states themselves. Indigenous peoples of the Americas exist within
settler-colonial states and are far outnumbered in the political structure of
such states. Conversely, there exist postcolonial states which are primarily
inhabited and politically controlled by the indigenous peoples of the region.
Both groups have ``cultural, intellectual, religious and spiritual property
taken without their free, prior and informed consent,'' yet their methods of
repatriation must necessarily differ. This paper investigates how each group
performs repatriation and how former colonial powers respond to requests for
cultural property.

\section{Trapped Within States: Diplomacy from Subnational Indigenous Groups}

For indigenous groups in settler-colonial states, connection to international
diplomacy is restricted through the formalities of state structures. The
existence of indigenous people is recognized by the international community
only as individuals residing within U.N.\ member states, rather than as
nations deserving of recognition as collective bodies \parencite{Sch12}.
\textcite{Car12} notes that the wording of UNDRIP specifies indigenous
peoples' rights to self-determination only through the construction of their
own institutions, yet refuses to afford those same institutions any
recognition within international relations. This conceptual understanding of
indigenous communities as subnational, or as Chief Justice Marshall spelled
out in the roots of federal Indian law ``that of a ward to his guardian''
\parencite[2]{The31} leaves indigenous nations unable to negotiate for
repatriation with states to whom they are not ``dependent.''

\subsection{NAGPRA Abroad}

NAGPRA, as U.S.\ domestic legislation, has no legal effect on museums abroad.
European museums are free to hold, trade, and conduct destructive analysis on
human remains and ceremonial objects that would be illegal within the U.S\@.
However, this does not mean the effects of NAGPRA are not felt abroad.

Owing to its imperial history, most human remains in British museums come from
overseas. A 2003 state survey of over one hundred museums found nearly half
possessed human remains from overseas, 41\% of which came from the Americas
during the period 1500-1947~CE\@. This was estimated to number approximately
1,800 individuals from that time period alone \parencite{Wee03}.

In 2004, following a scandal involving unauthorized possession of children's
organs \parencite[see][]{Bur02}, the U.K.\ Parliament passed the Human Tissue
Act, which created a regulatory authority for the study of human tissues.
Notably for repatriation of Native remains, the ``miscellaneous'' section of
the Act permits the boards of state museums to de-accession ``any human
remains which they reasonably believe to be remains of a person who died less
than one thousand years before the day on which this section comes into force
if it appears to them to be appropriate to do so for any reason, whether or
not relating to their other functions'' \parencite{Hum04}. The Department for
Culture, Media and Sport followed the Human Tissue Act with a report outlining
nonbinding guidance on de-accessioning and repatriating remains from overseas
upon request. Specifically, the guidance instructs museums to ``do no harm''
and to practice ``respect for diverse religious, spiritual and cultural
beliefs and attitudes to remains'' \parencite[14]{Dep05}. \textcite{Gie09}
further provides guidance to U.K.\ museums in negotiating claims from abroad.
Notable is the instruction to provide claimants with itemized inventories of
human remains and associated funerary objects, a requirement of NAGPRA found
nowhere in U.K.\ law yet provided to claimants in a spirit of goodwill. 

\section{States as Claimants of National Heritage}

For colonial empires, the collection of artifacts from conquered peoples
became a cornerstone of imperial identity. Most states existing today were at
some point colonized by a European power, yet few operated as settler-colonial
operations for the physical residence of European people. Those states in
which the indigenous population remain the vast majority of the country and
manage the affairs of the state present a different to international
repatriation when compared to the struggles of subnational indigenous groups.

Unlike colonization of the Americas, European conquest of Africa has left the
population of native Africans nearly intact with little exception. As such, it
is a model for the process of indigenous heritage management in the context of
postcolonial nation-building. \textcite{Shy09} places the establishment of
African national museums in the context of colonial resource extraction,
pointing out that for many colonial administrators, the removal of artifacts
and cultural items was justified on the basis that the colonies had no museums
of their own. In many African nations, therefore, had their origins in an
early twentieth-century desire to preserve some remainder of the nation's
cultural objects.

\subsection{The Creation of ``National Heritage''}

The idea of national heritage is best understood in its relation to the
concept of common human heritage, both of which influence the practice of
museum collection and operation. As \textcite{Tur09} argues, the Enlightenment
idea of cultural artifacts as the common heritage of mankind was the impetus
for the founding of such museums as the British Museum  and the Louvre. In
recent centuries, the idea of cultural artifacts as showing how human
societies differ has led to the idea of national heritage, where individual
nations seek to create national museums showcasing their own individual
history and culture. Both views of heritage and the human condition are
equally valid, but they present differences inherent to the museum systems in
contemporary states. While colonizer states are host to such ``common human
heritage'' museums, the one-way transmission of cultural artifacts
\parencite{Kup14} leaves formerly-colonized states unable to present the same
view, leaving only ``national heritage'' depictions.

States in this creation of heritage have inherent political goals. The ruling
classes are imbued with ideologies that in some cases run counter to the aims
of repatriation. In East Africa, the colonial government utilized
divide-and-rule tactics to mask anti-British sentiment with intertribal
grievances. As the colonial era came to an end, the colonial administrators
trained a new class of indigenous bourgeoisie to ensure Western exploitation
of the country \parencite{Mut79}. In Kenya, the Agĩkũyũ\footnote{The Agĩkũyũ
(s.\ Mũgĩkũyũ) and their language, Gĩgĩkũyũ, are often Romanized ``Kikuyu''}
people are perceived as having received the largest benefit from British rule,
as they play an outside influence in politics, the media, and industry. Most
importantly, under Kenya's first president, the Mũgĩkũyũ Jomo Kenyatta, the
ethnic group settled lands formerly claimed by the country's other ethnic
groups before colonial-era forced resettlement. This ethnic divide has spilled
into the sphere of academia and anthropology. Archaeology outside that of
early human evolution has failed to see the light of day in the country, with
most archaeologists being Europeans following in the footsteps of the Leakeys
\parencite{Kof97}. \textcite{Sch95} has argued that this white-dominated focus
on paleoanthropology is an ideological choice by the government: the ruling
elite is threatened by the land claims of the country's ethnic groups.

\begin{quote}
    Archaeology, if allowed to flourish at a regional level, can easily be
    identified with an attempt to valorize the history … of one ethnic group
    at the perceived expense of others. The state's deep investment in the
    white intellectuals' pursuit of ancient human ancestors has been an ideal
    way to neutralize regional histories … it focuses on a `population' devoid
    of ethnicity -- indeed, devoid of humanness.
    
    \parencite[128-129]{Sch95}
\end{quote}

Thus the practice of repatriation to postcolonial nations 

\subsection{Repatriation as Diplomacy}

The development project, as defined by \textcite{McM04}, was the
nation-state-centric model of development pursued through the Bretton Woods
system from the 1940s to the 1970s. Decolonization was in full swing, and
newly-formed countries were eager to improve their lives outside the bounds of
colonial government. The endpoint for a nation in this era was considered to
be the achievement of Western-style luxury and affluence. Newly-formed states
designed themselves around development, making large public expenditures to
build infrastructure and state-building projects. The states that took this
route adopted Western capitalism, and enormous loans were distributed to
countries across the global South, many in food and most in hard currency.
Industries spawned from Western investment and began exporting cheap products
en masse to the West. Even nations rich in national resources were importing
grains on a large scale from the United States. These imports came at a price
-- large loans from Bretton Woods institutions and Western states would need
to be paid back, but many countries found themselves in debt traps, getting
new loans to pay off the interest on previous ones \parencite{McM04}.

The maintenance of private property and land ownership remained intact through
decolonization nearly everywhere except Cuba, thus failing to address the
deepest effects of colonization on former colonies. Thus it is clear that the
development project of the mid-twentieth century was designed from the outset
to ensure continuous Western profit rather than global equity
\parencite{Fra66}. Much ink has been spilled over analyses of the extractive
and coercive nature of Western finance-capital in these early years of
postcolonial state formation, yet the movement of cultural objects and museum
artifacts can be seen within the same framework.

When confiscated artifacts become national heritage, they also become
bartering chips in the international dealings of states. The Royal Museum for
Central Africa in Belgium boasts nearly half a million African cultural
objects and geological samples, while even the largest museums within Africa
average collections one hundredth that size \parencite{Shy09}. This dichotomy
mirrors the division in accumulated capital between the two global spheres,
allowing each to reinforce one another. As noted in a brief Ethiopian
editorial entitled ``Let's have our treasure back, please'' in a 1999 issue of
\textit{The Economist}, African goods plundered as treasure do occasionally
return to their countries of origin, but with a second purpose: ``odd
treasures have been returned -- usually the less valuable ones -- as gestures,
whenever the British needed to court Ethiopia'' \parencite[66]{The99}. The
British then demonstrate a view of cultural objects simultaneously as heritage
of mankind -- in keeping them in their museums -- and also as trade goods --
in using them as state gifts expecting Ethiopian reciprocation.

The return of cultural objects in this manner should not be considered
repatriation -- in fact, it runs counter to the aims and morality of
repatriation. In reducing cultural patrimony to state gifts or trade goods,
Western nations serve to commodify culture itself. Paul Bator made this
connection explicit in describing nationalist views of art possession: ``The
Elgin marbles are part of England's national patrimony. All such works of art
are part of the national capital: They generate income (by attracting
tourists, etc.) and they can produce social and psychological benefits for a
country and its inhabitants'' \parencite[303]{Bat82}. Thus the neocolonial
project has absorbed the repatriation effort within itself as a manifestation
of capital accumulation, lessening its political expediency and reducing the
goal of restitution for invaluable wrongs into a simple exchange of goods and
services.

\section{Discussion}

Paradoxically, it appears that indigenous groups within settler-colonial
states -- lacking in international recognition and political power -- have an
easier path to repatriation than do indigenous-controlled postcolonial states.
The British Museum, known for its closely-guarded holdings and refusal to
relinquish even the most controversial of objects, has via the Department for
Culture, Media and Sport a framework for returning pre-1947 human remains when
requested by NAGPRA-covered groups. Conversely, the return of confiscated
artifacts to African states is performed within the framework of trade, where
colonial powers expect some form of reimbursement for their relinquishment.

As the peoples of the Global South represent a reservoir of both labor and
consumers unlike the indigenous peoples of settler-colonial states, it is
possible that Western powers' view of Global South artifacts as resources is
politically expedient for further exploitation. The neocolonial project has
simultaneously far more to gain from Africans and far more to lose from their
potential departure from the global economic order than it does from the few
remaining Native Americans, who are already directly within the Western
settler-colonial project. As such, the power of Global South artifacts
manifests most in their power to maintain compliance in the global system of
trade.

\printbibliography

\end{document}
